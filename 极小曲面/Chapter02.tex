\chapter{曲率估计}
\section{Choi-Schoen的结果}
设$\Sigma^{n-1} \subset \M^n$是光滑子流形. $\vec{n}$是 $\Sigma$的单位法向. 设$T$是$(0,2)$型张量, $T=T_{ij}dx^idx^j$. 记号约定如下.
\begin{equation}
    (\nabla^2_{XY}T)(Z,W)=\nabla^2T(Z,W,Y,X).
\end{equation}
在局部坐标下,
\begin{equation}
    \nabla^2_{lk}T_{ij}=(\nabla^2_{\partial l,\partial k}T)(\PI,\PJ)=\nabla^2T(\PI,\PJ, \partial k,\partial l)=T_{ijkl}.
\end{equation}
Codazzi方程.
\begin{equation}
    R^\M(X,Y,W,\vec{n}) = \nabla \II(W,Y,X) - \nabla \II (W,X,Y).
\end{equation}
Gauss方程
\begin{equation}
    R^\M(X,Y,Z,W) = R^\Sigma(X,Y,Z,W)-\II(X,W)\II(Y,Z)+\II(X,Z)\II(Y,W).
\end{equation}
求导交换次序.
\begin{equation}
    \nabla^2_{X,Y}T(Z,W) = \nabla^2_{Y,X}T(Z,W)-T(R(X,Y)Z,W)-T(Z,R(X,Y)W).
\end{equation}
在局部坐标下,
\begin{equation}
    T_{ijkl}=T_{ijlk}-R_{lki}^pT_{pj}-R_{lkj}^pT_{ip}.
\end{equation}
求导交换次序的结果可以参考\cite[定理7.14]{lee}. 需要注意, $T_{ijkl}=(\nabla^2_{lk}T)(\PI,\PJ)$.  
\begin{proposition}
    设$\Sigma^{n-1} \subset \R^n$是光滑曲面. $\nabla, \Delta$为$\Sigma$上的算子. $\II$为其第二基本型, $H$是平均曲率. 则
    \begin{equation} \label{simon1}
        \Delta \II_{ij}=\nabla^2_{ij}H + Hg^{kl}\II_{ik}\II_{jl}-\abs{\II}^2\II_{ij}.
    \end{equation}
\end{proposition}
\begin{proof}
    固定点$p$, 选取$p$点附近的测地坐标, 则有
    \begin{equation}
        \Delta \II_{ij}=\II_{ij;kk}=\II_{ik;jk}=\II_{ik;kj}-R_{kji}^p\II_{pk}-R_{kjk}^p\II_{ip}.
    \end{equation}
    而
    \begin{equation}
        \II_{ik;kj}=\II_{kk;ij}=H_{ij}.
    \end{equation}
    \begin{equation}
        \begin{split}
            R_{kji}^p\II_{pk}+R_{kjk}^p\II_{ip} &= R_{kjip}\II_{pk}+R_{kjkp}\II_{ip}\\
            &=\II_{kp}\II_{ji}\II_{kp}-\II_{ki}\II_{jp}\II_{pk}+\II_{kp}\II_{jk}\II_{ip}-\II_{kk}\II_{jp}\II_{ip} \\
            &=\abs{\II}^2\II_{ij}-Hg^{pq}\II_{jp}\II_{iq}
        \end{split}
    \end{equation}
\end{proof}
\begin{proposition}[Simons等式] \label{simon_equation}
    设$\Sigma^{n-1}$是极小曲面, 则有
    \begin{equation}
        \Delta\abs{\II}^2 = 2(\abs{\nabla \II}^2-\abs{\II}^4).
    \end{equation}
\end{proposition}
\begin{proof}
    将等式\eqref{simon1}写成坐标无关的形式, 又因$\Sigma$是极小曲面, 则有
    \begin{equation}
        \Delta \II = -\abs{\II}^2 \II
    \end{equation}
    由于$\Tr$与联络导数可交换, 则
    \begin{equation}
        \begin{split}
            \Delta \abs{\II}^2 = \Delta (\Tr^2(\II\otimes\II)) &= \Tr^2(\Delta \II \otimes\II + \II \otimes \Delta \II + 2 \Tr(\nabla \II, \nabla \II)) \\
            &=\Tr^2(-2\abs{\II}^2 \II\otimes\II+2\Tr(\nabla \II \otimes \nabla \II)) \\
            &=-2\abs{\II}^4+2\abs{\nabla \II}^2.
        \end{split}
    \end{equation}
\end{proof}
\begin{proposition}[Simons不等式]
    设$\Sigma^{n-1}$是极小曲面, $\nabla,\Delta$为$\Sigma$上的算子. 则
    \begin{equation} \label{simon_inequality}
        \Delta \abs{\II}^2 \ge -2\abs{\II}^4+2(1+\frac{2}{n-1})\abs{\nabla \abs{\II}}^2.
    \end{equation}
\end{proposition}
\begin{proof}
    证明与命题\eqref{kato}类似. 我们首先来计算 $\abs{\nabla \abs{\II}}^2$. 固定点$p$, 取$p$点处的测地坐标使得$\II=diag(\lambda_i)$.
    \begin{equation} \label{s11}
        \begin{split}
            \abs{\nabla \abs{\II}}^2 = \frac{\abs{\inner{\nabla \II}{\II}}^2}{\abs{\II}} &= \frac{\sum_j(\sum_i\lambda_i\II_{iij})^2}{\sum_i\lambda_i^2} \\
            &\le \sum_{ji}\II_{iij}^2
        \end{split}
    \end{equation}
    而
    \begin{equation} \label{s12}
        \begin{split}
            \sum_{ji}\II_{iij}^2 &= \sum_{j\ne i}\II_{iij}^2+ \sum_{i}\II_{iii}^2 \\
            &= \sum_{j\ne i}\II_{iij}^2+ \sum_{i}(-\sum^{n-1}_{j \ne i}\II_{jji})^2\\
            &\le \sum_{j\ne i}\II_{iij}^2+ \sum_{i}(n-2)\sum_{j\ne i}\II_{jji}^2\\
            &\le (n-1)\sum_{j\ne i}\II_{iij}^2\\
            &= \frac{n-1}{2}(\sum_{j\ne i}\II_{iji}^2 +\sum_{j\ne i}\II_{jii}^2)
        \end{split}
    \end{equation}
    \eqref{s12}两侧同乘以$\frac{2}{n-1}$后与\eqref{s11}相加可得,
    \begin{equation}
        \begin{split}
            \abs{\nabla \abs{\II}}^2+\frac{2}{n-1} \abs{\nabla \abs{\II}}^2 &\le \sum_{ji}\II_{iij}^2 + \sum_{j\ne i}\II_{iji}^2 +\sum_{j\ne i}\II_{jii}^2 \\
            &\le \sum_{ijk}\II^2_{ijk}
        \end{split}
    \end{equation}
    再由Simon等式\eqref{simon_equation}即可得到不等式\eqref{simon_inequality}.
\end{proof}
\begin{theorem}\label{choi_schoen}
    设$\Sigma^2 \subset \R^3$是极小曲面.  则存在常数$\epsilon, \rho>0$, 使得 $\forall p \in \Sigma, r_0 < \rho$, 如果
    \begin{enumerate}
        \item $\partial \Sigma \cap B_{r_0}(p)=\emptyset$.
        \item $\int_{\Sigma \cap B_{r_0}} \abs{\II}^2\le \epsilon$.
    \end{enumerate}
    那么$\forall  0 < \sigma \le r_0$及$y \in B_{r_0-\sigma}$, 成立
    \begin{equation}
        \sigma^2 \abs{\II}^2(y) \le \delta, \text{ here }\delta= \frac{\int_{\Sigma \cap B_{r_0}}\abs{\II}^2}{\epsilon}.
    \end{equation}
\end{theorem}
\begin{proof}
    将$B_{r_0}$简写为$B$. 只需要证明$F(x)=d^2(x,\P B)\abs{\II}^2(x) \le \delta$即可. 我们用反证法. 设$F(q)=\max_{x \in B}F(x)$. 注意到在$\partial B$上, $F(x) \eq 0$. 则$q$是内点.  设$F(q) > \delta$, 我们将证明当$\epsilon$足够小时, 会推出一个矛盾. 取 $\sigma$使得
    \begin{equation}
        \sigma^2\abs{\II}^2(q)=\frac{\delta}{4} \le \frac{1}{4} d^2(q,\P B)\abs{\II}^2(q).
    \end{equation}
    因此, 有
    \begin{equation}
        \sigma \le \frac{1}{2}d(q,\P B).
    \end{equation}
    由三角不等式可知,  $\forall y \in B_\sigma(q)$,
    \begin{equation}
        \frac{1}{2} \le \frac{d(y,\P B)}{d(q,\P B)} \le 2.
    \end{equation}
    于是
    \begin{equation}
        \begin{split}
            \sup_{y\in B_\sigma(q)} d^2(q,\P B) \abs{\II}^2(y) \le &4\sup_{y\in B_\sigma(q)}d^2(y,\P B)\abs{\II}^2(y) \\
            =& 4d^2(q,\P B) \abs{\II}^2(q).
        \end{split}
    \end{equation}
    因此,
    \begin{equation}
        \sup_{y \in B_\sigma(q)} \abs{\II}^2(y) \le 4\abs{\II}^2(q)=\frac{\delta}{\sigma^2}.
    \end{equation}
    通过变换$x\to \sigma x$, 上面的不等式变为
    \begin{equation} \label{ii_sup}
        \sup_{y \in B_1(q)} \abs{\II}^2(y) \le 4\abs{\II}^2(q)=\delta \le 1.
    \end{equation}
    由Simons不等式, 我们得到
    \begin{equation}
        \Delta \abs{\II}^2 \ge -2\abs{\II}^4 \ge -2 \abs{\II}^2.
    \end{equation}
    由推论\eqref{sub_harmonic}(取$s=1,\lambda=2$)可知,
    \begin{equation}
        \frac{\delta}{4}=\abs{\II}^2(q) \le C \int_{B_1} \abs{\II}^2 \le C\delta\epsilon.
    \end{equation}
    则当$\epsilon$足够小时, 显然是不可能的.
\end{proof}
\begin{lemma}[一致图像引理] \label{uniform_graph}
    设$\Omega \subset \R^3$是具有光滑边界的开集. 设$\Sigma^2 \subset \Omega$是嵌入的光滑曲面, 并且$\Sigma$在$\Omega$中是闭的. 设$\abs{\II} \le C$. 则$\Sigma$可以局部一致地表示成函数图像的形式. 确切地说, $\forall p \in \Sigma$, $R\in (0,R_0)$. 这里, 
    \begin{equation}
        R_0=\min\{\frac{1}{4C}, \frac{1}{2}d(p,\partial \Omega)\}.
    \end{equation}
    设$D(p,R)$为$T_p\Sigma$上以$p$为中心, 半径为$R$ 圆盘, $W(p,R)=D(p,R)\times \R$. 
    \begin{align}
        &D(p,R)= \{p+\vec{v}\mid v \in T_p\Sigma, \abs{v} < R\}. 
        &W(p,R)= \{q+t\vec{n}(p)\mid q \in D(p,R), t \in \R\}.
    \end{align}
    则存在函数$u: D(p,R)\to \R$ 使得$u(p)=0$, 并且$\Sigma$可以写成$u$的在$D(p,R)$上的图像的形式. 并且, 给定$q \in D(p,R)$,
    \begin{enumerate}
        \item $\abs{u(p)-u(q)} \le 8C\abs{p-q}$.
        \item $\abs{\nabla u}(q) \le 8C\abs{p-q}$.
        \item $\abs{\nabla^2 u}(q) \le 16C$.
    \end{enumerate}
\end{lemma}
\begin{remark}
    一致图像引理不要求曲面是极小的.
\end{remark}
\begin{proof}
    设$\Sigma_{p,r}$是$\Sigma \cap W(p,r)$的包含$p$点的连通分支.  经过旋转平移之后, 设$p =0, T_p\Sigma = \{z=0\}$. 设$R \in \R$满足以下条件, 并且取上界: $\Sigma_{p,R}$可以表示为某个函数$u \in C^\infty(D(p,R))$的图像, $u(0)=0, \nabla u(p)=0$ 且 $\abs{\nabla u} \le 1$. 注意, $\abs{\nabla u} \le 1$等价于$N_3=\inner{\vec{n}}{e_3}=\frac{1}{\sqrt{1+\abs{Du}^2}} \ge \frac{1}{2}$. 现在, 当$R$变得更大一点时, 有两种情况可能发生:
    \begin{enumerate}
        \item 函数$u$可以光滑地延拓到$D(p,R)$的邻域内, 但是会导致$N_3< \frac{1}{2}$. \label{case1}
        \item 函数$u$不能继续延拓, 这表明$\G(u)$碰到了$\Omega$的边界. \label{case2}
    \end{enumerate}
    直接计算可知, 我们有如下的比较关系
    \begin{equation}
        \frac{\abs{\nabla^2 u}^2}{(1+\abs{\nabla u}^2)^3} \le \abs{\II}^2 \le 2 \frac{\abs{\nabla^2 u}^2}{1+\abs{\nabla u}^2}.
    \end{equation}
    在$D(p,R)$上, 由于$\abs{Du} \le 1$, 则
    \begin{equation}
        C^2 \ge \abs{\nabla^2 u} \le 8C^2.
    \end{equation}
    现在, 设情况\eqref{case1}发生, 设$q \in \partial D(p,R)$使得$\abs{\nabla u(q)}=\frac{1}{2}$. 则
    \begin{equation}
        1=\abs{\nabla u(0)-\nabla u(q)} \le \abs{\nabla^2 u}\abs{q}\le 4CR.
    \end{equation}
    因此, $R \ge \frac{1}{4C}$.
    若情况\eqref{case2}发生,  设$Q=(q,u(q))\in \partial \Omega, q \in \partial D(p,R)$. 设$\alpha$是连接 $p,q$的线段, 则
    \begin{equation}
        d(p,Q) \le \sup \sqrt{1+\abs{\nabla u}}l_\alpha \le 2R.
    \end{equation}
    即有$R \ge \frac{1}{2}{d(p,\partial \Omega)}$.  其余性质均可由中值定理得到.
\end{proof}
\begin{corollary} \label{choi_schoen_intrinsic}
    定理\eqref{choi_schoen}中的$B_r$可以换成$B^\Sigma_r$.
\end{corollary}
\begin{proof}
    在定理\eqref{choi_schoen}的证明中, 有了不等式\eqref{ii_sup}(欧氏度量换成内蕴度量后, 直到这里的推导都是正确的), 我们就可以用应用一致图像引理. 由于此时$\abs{\nabla u}$一致有界, 那么$\Sigma$上的内蕴度量与外蕴度量实际上是局部可比较的. 最后在$B_\frac{1}{4}$(引理\eqref{uniform_graph}中, $C=1$)上应用推论\eqref{sub_harmonic}.
\end{proof}
\begin{theorem}
    设$\Sigma^2 \subset \R^3$是定向的稳定极小曲面. 则存在常数$C>0$使得
    \begin{equation}
        \abs{\II}^2(x) \le \frac{C}{d^2_\Sigma(x,\partial \Omega)}.
    \end{equation}
\end{theorem}
\begin{proof}
    记$r_0=d(x,\partial \Sigma)$. 由命题\eqref{mc_area}可知, $\forall 0<r < r_0$, 
    \begin{equation}
        \Area(B^\Sigma_r) \le Cr^2.
    \end{equation}
    (这里如果$B^\Sigma_r$不是嵌入圆盘, 我们就在万有覆盖中来做). 固定$x$, 设$r=d(y,x)$, 定义截断函数
    \begin{equation}
        \eta(y)=\left\{
            \begin{aligned}
                & 1, r\le e^{-N}r_0, \\
                & \frac{\log r_0-\log r}{N}, e^{-N}r_0 < r < r_0, \\
                & 0, r>r_0.
            \end{aligned}
        \right.
    \end{equation}
    由于$\abs{\nabla r}=1$, 则$\abs{\nabla \eta} \le \frac{1}{Nr}$. 将$\eta$代入到稳定性不等式中, 得到
    \begin{equation}
        \begin{split}
            \int_{B^\Sigma_{e^-Nr_0}}\abs{\II}^2 \le \int_\Sigma \eta^2 \abs{\II}^2 &\le \int_\Sigma \abs{\nabla \eta}^2  \\
            &\le \frac{1}{N^2}\sum_{i=-n}^{-1}\int_{B_{e^{i+1}r_0}^\Sigma-B_{e^ir_0}^\Sigma}\frac{1}{r^2} \\
            &\le \frac{4\pi e^2}{3N}.
        \end{split}
    \end{equation}
    $N$取足够大时就可以应用定理\eqref{choi_schoen}(推论\eqref{choi_schoen_intrinsic}).
\end{proof}
\begin{theorem} \label{curvature_estimate_4pi}
    设$C \in (0,8\pi)$. 则存在常数$C_0=C_0(C)$使得: 如果$\Sigma \subset \R^3$是定向极小曲面, 并且
    \begin{equation}
        \int_\Sigma \abs{\II}^2 dv_g \le C.
    \end{equation}
    则$\forall p \in \Sigma$,
    \begin{equation}
        d(p,\P \Sigma) \le C_0.
    \end{equation}
\end{theorem}
\begin{theorem}
    设$\Sigma \subset \R^3$是完备,定向的极小曲面, 且
    \begin{equation}
        \int_\Sigma K^- < +\infty. 
    \end{equation}
    这里, $K^-=-\min(K,0)$, $K$是Gauss曲率. 则有
    \begin{enumerate}
        \item 存在闭黎曼面$\mathcal{S}$及$\{p_1,p_2\cdots p_k\} \subset \mathcal{S}$使得$\Sigma$共形同构于$\mathcal{S}-\{p_i\}$.
        \item 存在整数$m$使得$\int_\Sigma Kdv_g = 4m\pi$.
    \end{enumerate}
\end{theorem}
\begin{proof}
    设$p \in \Sigma$. 记$D_r=D(p,r)=\{x\in \Sigma \mid d_\Sigma(p,x) \le r\}$, 则$\P D_r$是分段光滑曲线. 设$\{\theta^r_i\}$是$\P D$上的非光滑点处的外角, 如是图所示.
    \begin{figure}[ht]
        \centering
        \includegraphics[scale=0.8]{images/angle.png}
        \caption{区域$D_r$的边界.}
        \label{angle}
    \end{figure}
    我们这里选取的定向使得$-\pi \le \theta_i \le 0$.  记$l(r)=l(\P D_r)$. 则由第一变分公式(非光滑点附近需要单独处理, 参考\cite{White}及\cite{Perez}),
    \begin{equation}
        l'(r)=\int_{\P D_r}K_{\P D}ds+2\sum \tan(\frac{\theta_i}{2}).
    \end{equation}
    而由于$\tan(\frac{\theta_i}{2}) \le \frac{\theta_i}{2}$, 及Gauss-Bonnet公式(分段光滑区域上的Gauss-Bonnet公式参考\cite{lee}),
    \begin{equation}
        l'(r)\le \int_{\P D_r}K_{\P D}ds+\sum \theta_i = 2\pi\chi(D_r)-\int_{D_r}Kdv_g.
    \end{equation}
    由于$\mathop{\limsup} l'(r) \ge 0$(若$ \mathop{\limsup l'(r)} \le \alpha <0$, 则存在$r$足够大使得$l(r) < 0$, 这是不可能的), 则有
    \begin{equation} \label{basic_inequality}
        \begin{split}
            0 \le \limsup_{r\to \infty} l'(r) &\le \limsup(2\pi \chi(D_r) - \int_{D_r}Kdv_g) \\
            & \le \limsup (2\pi(2-2g(D_r)-(\P D_r)^\#)-\int_{D_r}K^+dv_g+\int_{D_r}K^-dv_g)
        \end{split}
    \end{equation}
    其中, $g$为$D_r$的亏格, $(\P D_r)^\#$为$D_r$的边界的数目. 上述不等式意味着, 存在常数$C$使得
    \begin{enumerate}
        \item $g(D_r), (\P D_r)^\#, \int_{D_r}K^+dv_g \le \int_\Sigma K^-dv_g +C$. \label{finite}
        \item 存在$r_0>0$使得$\forall r > r_0$, $g(D_r) = g(D_{r_0})$. \label{genus}
    \end{enumerate}
    结论\eqref{finite}由\eqref{basic_inequality}取极限直接得到. 结论\eqref{finite}是因为$g(D_r)$是有上界的, 且$g(D_r)$只取整数值且关于$r$是非减的. 现在, 对于$\Sigma - D_r$,我们证明
    \begin{claim}
        设$r > r_0$, 则$\Sigma - D_r$的每一个紧的连通分支都是单连通的.
    \end{claim}
    设$\Omega$是$\Sigma - D_r$的一个连通分支. 若$\P \Omega$至少包含两个边界, 由于$\P \Omega \subset \P D_r$, 则$g(D_r \cup \Omega) > g(D_r)$, 这与结论\eqref{genus}矛盾. 断言证毕.
    \par 现在, 记$A_r=D_r\cup \{\Sigma - D_r\text{的所有紧连通分支}\}$. 同时取$r_i$使得
    \begin{equation}
        (\P D_{r_i})^\# = (\P D_{r_{i+1}})^\#=c.
    \end{equation}
    $\P D_{r}$总是取整数值并且有上界, 这总是可以取到的. 由于$(\P A_{r_i})^\# \le (\P D_r)^\#$, 可设
    \begin{equation}
        (\P A_{r_i})^\# = (\P A_{r_{i+1}})^\#=c.
    \end{equation}
    此时, $A_{r_i} \subset A_{r_{i+1}}$并且 $A_{r_i}$与$A_{r_{i+1}}$同胚, 则$A_{r_{i+1}}$是由$A_{r_i}$在边界处添加annulus而得到. 在不等式\eqref{basic_inequality}中取极限, 则有
    \begin{equation}
        0 \le 2\pi \chi(\Sigma) - \int_\Sigma K dv_g.
    \end{equation}
    到现在为止, 我们已经证明了$\Sigma$是有限型曲面, 即同胚于闭区曲面去掉有限个点.
    \par 设$E$为$\Sigma$的一个末端 (end, 即$\Sigma-D_{r_0}$的一个连通分支, $r_0$取足够大). 则$E$是annulus. 
    \begin{claim}
        $E$共形同构于$\{1 < \abs{z} < +\infty\}$.
    \end{claim}
    记$\Gamma_r=\P D_r \cap E$. 设$\phi: E \to \{ 1 < \abs{z} < R\}$是共形等价. 我们需要证明$R=+\infty$. 设$\Gamma_r(s)$是其弧长参数化. 则
    \begin{equation}
        2\pi = l_{\S^1} \le l_{\phi\circ \Gamma_r} = \int_{\Gamma_r \abs{d \phi(\Gamma_r)\Gamma'_r}}ds.
    \end{equation}
    由Cauchy不等式, 则有
    \begin{equation}
        \begin{split}
            4\pi^2 \le (\int_{\Gamma_r} \abs{d\phi(\Gamma_r)\Gamma'_r}ds)^2 & \le l_{\Gamma_r}\int_{\Gamma_r}\abs{D\phi(\Gamma_r)}^2ds \\
            & =l(\Gamma_r)\int_{\Gamma_r}\det \phi ds.
        \end{split}
    \end{equation}
    (最后一个等式中, 我们用到了对于全纯函数$\phi$, $\abs{d\phi}^2 = \abs{\phi'}^2= \det \phi$). 由于
    \begin{equation}
        l'(r) \le 2\pi\chi(D_r) - \int_{D_r}Kdv_g \to 2\pi\chi(\Sigma) - \int_\Sigma K dv_g,
    \end{equation}
    则存在常数$C$使得 
    \begin{equation}
        l(r) \le Cr.
    \end{equation}
    综合以上几个不等式, 我们得到
    \begin{equation}
        \frac{4\pi^2}{Cr} \le \frac{4\pi^2}{l(r)} \le \frac{4\pi^2}{l_{\Gamma_r}} \le \int_{\Gamma_r}\det\phi ds.
    \end{equation}
    对$r$积分, 则有
    \begin{equation}
        +\infty = \int^\infty_{r_0} \frac{4\pi^2}{Cr} \le \int^\infty_{r_0}\int_{\Gamma_r} \det\phi dsdr = \Area(\{1<\abs{z}< R\}).
    \end{equation}
    因此, $R=+\infty$.
\end{proof}