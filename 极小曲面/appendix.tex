%\addcontentsline{toc}{chapter}{附录}
\appendix
\appendix
\chapter{附录}

设$\M\subset \MM^{n+k}$是$n$维嵌入子流形. $\M$上的联络记为$\D$, $\MM$上的联络记为$\DD$. 对于每一个点$p \in \M$, $T_p\MM$可以分解为
\begin{equation}
    T_p\MM=T_p\M \oplus  (T_p\M)^\T
\end{equation}
\begin{lemma}
    设$A: \R^{n^2} \to \R$光滑. 设$A^*$是$A$的伴随矩阵. 则$d\det A=\tr(A^*dA)$.
\end{lemma}
\begin{proof}
    由矩阵的Laplace展开, $\det A=A^*_{ij}A_{ij}$. 只需注意到$A^*_{ij}$与$A_{ij}$无关, 则有
    \begin{equation}
        d\det A= \frac{\P \det A}{\P A_{ij}}dA_{ij}=A^*_{ij}dA_{ij}=\tr(A^*dA)
    \end{equation}
\end{proof}
\begin{corollary}
    设$A_t: \R \to \R^{n^2}$连续可微且$A_t$可逆.则有
    \begin{equation}
        \frac{d}{dt}\det A_t=(\det A_t)\tr(A^{-1}_t\frac{d}{dt}A_t)
    \end{equation}
\end{corollary}
\begin{definition}
    设$(\M,g)$是黎曼流形.  称局部坐标$(x^1\cdots, x^n)$是\textit{半测地坐标}, 如果对于任意$x^1\cdots, x^{n-1}$, 曲线$\gamma(t): t \mapsto (x^1\cdots,x^{n-1},t)$是具有单位速度的测地线并且$\gamma(t)$与$x^n$的水平集(level set)正交.
\end{definition}
\begin{lemma}
    设$(\M,g)$是黎曼流形.  设$(x^1\cdots, x^n)$是$\M$中的局部坐标. 则下列陈述等价:
    \begin{enumerate}
        \item $(x^1\cdots, x^n)$是半测地坐标.
        \item $\abs{\P_n}=1$并且对于$\alpha=1\cdots,n-1$, $\inner{\P_\alpha}{\P_n}=0$.
        \item $\abs{\nabla x^n}=1$并且对于$\alpha=1,\cdots,n-1$, $\inner{\nabla x^\alpha}{\nabla x^n}=0$.
        \item $\abs{\nabla x^n}=1$并且对于$\alpha=1,\cdots,n-1$, $x^\alpha$在$\nabla x^n$的每一条积分曲线上是常数.
        \item $\nabla x^n=\P_n$.
    \end{enumerate}
\end{lemma}

\section{联络}
\begin{proposition}
    设$\Sigma \subset M$是子流形.  设$Y \in T_p\Sigma$, $X \in TM$. 则$\nabla_YX$与$X$在$M$中的扩张无关.
\end{proposition}
一些简单的结果.
\begin{enumerate}
    \item 设$X\in T\Sigma$, $N \in T^\perp\Sigma$, 则$\nabla_X N \perp T\Sigma$.
    \item 设$Y \in T_p \Sigma$, $X \in TM$, 则$(\nabla_XY)^\tangent=\nabla_X{Y^\tangent}$.
\end{enumerate}
\section{散度}
\begin{equation}
    \div(X)=\rec{\sqrt{g}}\P_i(X^i\sqrt{g})
\end{equation}
设$\{e_i\}$是一组局部正交标架,则有
\begin{equation}
    \begin{split}
        \div(X)=\tr\nabla X=&\sum_i \inner{e_i}{\nabla_{e_i}X} \\
        =&e_i(X^i)+X^j\Gamma^i_{ij}
    \end{split}
\end{equation}
设$\{e_i\}$是一组局部标架(不一定正交), 设$g_{ij}=\inner{e_i}{e_j}$, 则
\begin{equation}
    \begin{split}
        \div(X)=&g^{ij}\inner{e_i}{\nabla_{e_j}X}
        %=&e_i(X^i)+X^j\Gamma^i_{ij}
    \end{split}
\end{equation}
\section{子流形上的散度}
设$\Sigma \subset M^n$是$k$维子流形. 设$X \in T_\Sigma M$, 即$X$定义在$\Sigma$中并且$\forall p \in \Sigma$, $X(p) \in T_pM$. 将$X$扩张为$M$上的向量场.  设$e_i$是$\Sigma$上的是标准正交标架. 定义$X$的散度为
\begin{equation}
    \div_\Sigma X=\tr(\nabla X^\tangent)=\inner{\nabla_{e_i}X}{e_i}
\end{equation}
\begin{proposition}
    $\div_\Sigma(X)$是良好定义的, 即与$X$的扩张无关, 与$e_i$的选取无关.
\end{proposition}
\section{Hessian矩阵}
\begin{equation}
    \nabla^2u=(\frac{\P^2u}{\P x^i\P x^j}-\Gamma^k_{ij}\frac{\P u}{\P x_k})dx^i\otimes dx^j
\end{equation}
\begin{equation}
    \nabla^2u(X,Y)=X(Y(f))-(\nabla_XY)(f)
\end{equation}
\section{Ricci恒等式}
\begin{equation}
    \nabla_X(T(Y,Z))=\nabla_XT(Y,Z)+T(\nabla_XY,Z)+T(X,\nabla_XZ)
\end{equation}
取$T=\nabla^2T$,则有
\begin{equation}
    \begin{split}
        \nabla_X(\nabla^2T(Y,Z))&=\nabla_X\nabla^2T(Y,Z)+\nabla^2T(\nabla_XY,Z)+\nabla^2T(Y,\nabla_XZ)\\
        &=\nabla^3T(X,Y,Z)+\nabla^2T(\nabla_XY,Z)+\nabla^2T(Y,\nabla_XZ)\\
    \end{split}
\end{equation}
\begin{equation}
    \begin{split}
        \nabla^2_{X,Y}(T(Z))&=\nabla_X\nabla_Y(T(Z)) -\nabla_{\nabla_XY}(T(Z))\\
        &=\nabla_X(\nabla_YT(Z)+T(\nabla_YZ)) -\nabla_{\nabla_XY}(T(Z))\\
        &=\nabla_X\nabla_YT(Z)+\nabla_YT(\nabla_XZ)+\nabla_XT(\nabla_YZ)+T(\nabla_X\nabla_YZ)-\nabla_{\nabla_XY}T(Z)-T(\nabla_{\nabla_XY}Z)
    \end{split}
\end{equation}
\section{Laplace算子}
\begin{equation}
    \begin{split}
        \Laplace u &=\div \nabla u \\
        &=\rec{\sqrt{g}}\frac{\P}{\P x^i}(\sqrt{g}g^{ij}\frac{\P u}{\P x^j}) 
    \end{split}
\end{equation}
\begin{equation}
    \Laplace u=\nabla_{e_i}\nabla_{e_i}u-(\nabla_{e_i}e_i)u
\end{equation}
通常,记$\nabla^2_{XY}=\nabla_X\nabla_Y-\nabla_{\nabla_XY}$. 因此,
\begin{equation}
    \Laplace u=\sum_i\nabla^2_{e_ie_i}u
\end{equation}
类似地, 对于任意张量$T$,我们都可以定义Laplace算子.
\begin{equation}
    \Laplace T=\tr \nabla^2 T
\end{equation}
\subsection{与Laplace算子有关的公式}
\begin{equation}
    \Laplace\inner{X}{Y}=\inner{\Laplace X}{Y}+\inner{X}{\Laplace Y}
\end{equation}