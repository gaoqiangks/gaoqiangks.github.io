\chapter{Plateau问题}
\newcommand{\FG}{{\mathcal{F}_\gamma}}
\newcommand{\AG}{{\mathbb{A}_\gamma}}
\renewcommand{\EG}{{\mathbb{E}_\gamma}}
\renewcommand{\E}{{E}}
\renewcommand{\D}{{\mathbb{D}}}
\newtheorem*{plateauproblem*}{Plateau问题}
本章中, 我们回到最开始的问题:
极小曲面起源于下面的问题:
\begin{plateauproblem*}\label{plateau}
    给定简单闭曲线$\gamma \subset \R^3$, 是否存在以$\gamma$为边界的面积最小的曲面? 即, 寻找曲面$\Sigma, \partial \Sigma = \gamma$, 使得对于任意满足$\partial \Sigma' = \gamma$的曲面$\Sigma'$, 都有
    \begin{equation*}
        \Area(\Sigma) \le \Area(\Sigma')
    \end{equation*}
\end{plateauproblem*}
在本章, 按照Douglas和Rado的方法, 我们将给出Plateau问题的肯定的回答. 关于Douglas在Plateau问题上的工作的简介, 可以参考\cite{WorkofDouglas}.
\par 设$\gamma \subset \R^3$是简单闭曲线. 记$\FG$是满足下列条件的函数的集合:
\begin{enumerate}
    \item $u: \overline{\D} \to \R^3, u\in C(\overline{\D})\cap W_{loc}^{1,2}(\D,\R^3)$.
    \item $u\mid_{\P \D}$是到$\gamma$的单调映射.
    %\item $\abs{u_x}^2=\abs{u_y}^2, \inner{u_x}{u_y}=0$.
\end{enumerate}
\begin{remark}
    称$u$是单调映射, 是指我们将$\gamma$沿任意点剪断看作实轴上的一段线段,  $u$是通常意义下的单调映射.  这里之所以没有要求$u$是同胚, 是因为同胚在求极限后(一致收敛拓扑)不会被保持,  而单调性总是会被保持的. 
\end{remark}
解决Plateau问题的第一选择是:取$u_k \in \FG$使得$\Area(u_k(\D)) \to \AG$, 并证明$\{u_k\}$存在收敛子列. 然而这种思路有两个问题需要解决:
\begin{enumerate}
    \item 我们可以改变$u_k$在任意点$p$附近的值, 得到$\tilde{u}_k$, 使得$\tilde{u}_k$与$\tilde{u}_k(\mathbb{D})$与 $u_k(\D)$相差"细长"的管状区域, 如图. 这样得到的$\tilde{u}_k$仍然是面积最小的列. 但是这样的列无法在通常的意义下收敛. 
    \item 第二个问题是面积泛函是与参数选取无关的, 这意味着选取$\{u_k\}$后, 对于任意的微分同胚$\phi_k: \D \to \D$, $\Area(u_k\circ \phi_k(\D))= \Area(u_k(\D))$. 这种情况下, 我们也很难得到收敛子列.
\end{enumerate}
为了解决上面的问题, 我们将使用能量泛函而不是面积泛函. 当然, 首先我们需要说明的是, 对能量泛函求最小值和对面积泛函求最小值这两个问题是等价的(虽然对于能量泛函, 上述两个问题仍然存在, 但是我们较好的解决该问题的方法).
\begin{theorem} \label{plateau_problem}
    设$\gamma \subset \R^3$是可求长的Jordan曲线. 则存在映射$u \in \FG$使得 $\forall v \in \FG$, 
    \begin{equation}
        \Area(u(\D))\le \Area(v(\D)).
    \end{equation}
\end{theorem}
对于任意$u \in \FG$, 其面积与能量分别定义为
\begin{align}
    &\Area(u)=\int_\D\abs{u_x\wedge u_y} \\
    &\E(u)=\frac{1}{2}\int_\D\abs{u_x}^2+\abs{u_y}^2
\end{align}
简单计算可知
\begin{equation}\label{AleE}
    \Area(u)=\int_\D \sqrt{\abs{u_x}^2\abs{u_y}^2-\inner{u_x}{u_y}^2} \le \frac{1}{2}\int_\D \abs{u_x}^2+\abs{u_y}^2 \le\E(u).
\end{equation}
并且等号成立, 当且仅当$\abs{u_x}=\abs{u_y}, \inner{u_x}{u_y}=0$.
\par 记
\begin{align}
    &\AG=\inf\{\Area(v)\mid v \in \FG\}.\\
    &\EG=\inf \{\E(v)\mid v \in \FG\}.
\end{align}
\begin{definition}
    如果$u \in W^{1,2}(\mathbb{D},\R^3)$几乎处处满足$\inner{u_x}{u_y}=0$并且$\abs{u_x} = \abs{u_y}$, 则称$u$是几乎共形的.
\end{definition}
\begin{lemma}
    $\AG=\EG$, 并且如果$u \in \FG$取到$\EG$, 则$u$是几乎共形的.
\end{lemma}
\begin{proof}
    由不等式\eqref{AleE}可知, $\AG \le \EG$.
    \par 对于反方向的, 设$u\in \FG$ 且$\E(u) \le \EG + \epsilon$. 首先设$u$是浸入, 即$du$处处非退化. 设$(\D,g)$为$u$作用下的拉回度量, 即$g=du^*dx^2$, 由等温坐标的存在性可知, 存在光滑同胚$\phi: \D \to \D$使得$\phi$是$\mathbb{D} \to (\D,g)$ 之间的共形映射, 即$d\phi ^* g=\lambda^2dx^2$. 而$u\circ \phi$是共形浸入, 则有
    \begin{equation}
        \Area(u)=\Area(u\circ \phi)=\E(u\circ \phi) \ge \E(u)-\epsilon.
    \end{equation}
    如果$du$有奇点, 那么我们定义$u^s: \D \to \R^5$, $u^s(x,y)=(u,sx,sy)\in \R^5$. 则$du^s$是非退化的. 像上面一样, 通过$u^s$拉回的度量为$du^*g_{\R^3}+s^2(dx^2+dy^2)$. 显然地, 
    \begin{equation}
        \Area(u^s)=\int_\D \det (du^*g_{\R^3}+s^2I) \ge \Area(u).
    \end{equation}
    \begin{equation}
        \E(u^s\circ\phi)=\int_\D \abs{(u^s\circ \phi)_x}^2+ \abs{(u^s\circ \phi)_y}^2 =\E(u\circ \phi)+s^2\E(\phi).
    \end{equation}
    因此, 当$s$足够小时, 我们有
    \begin{equation}
        \begin{split}
            \Area(u) \le \Area(u^s\circ \phi) &= \E(u^s\circ \phi) =\E(u\circ \phi) +s^2\E(\phi) \\
            &\le \E(u)-\frac{1}{2}\epsilon
        \end{split}
    \end{equation}
\end{proof}
\begin{lemma}
    设$f \in W^{1,2}(\D,\R^3)$, $E(f) \le K$. 设$0 < \delta < 1$, $p \in \mathbb{D}$. 则存在$\rho \in (\delta, \sqrt{\delta})$ 使得$f\mid_{\P B(p,\rho) \cap \D}$是绝对连续的, 且$\forall z_1,z_2 \in \P B(p,\rho) \cap \D$, 成立 
    \begin{equation}
        \abs{f(z_1)- f(z_2)} \le (8\pi)^{\frac{1}{2}}(\log \frac{1}{\delta})^{-\frac{1}{2}}.
    \end{equation}
\end{lemma}
\begin{proof}
    在$p$点处引理极坐标$(r,\theta)$. 由于$f \in W^{1,2}$, 则几乎对于所有的$r$, $f$关于$\theta$是绝对连续的.  则Cauchy不等式, 
    \begin{equation}
        \begin{split}
            \abs{f(z_1)-f(z_2)} &\le \int_{\P B \cap \D} \abs{f_\theta(re^{i\theta})}d\theta \\
            & \le \sqrt{2\pi} (\int_{\P B\cap \D} f^2_\theta d\theta)^{\frac{1}{2}}.
        \end{split}
    \end{equation}
    而由于在极坐标下, $Df=f_r \P_r + \frac{1}{r^2}f_\theta \P_\theta$, 则
    \begin{equation}
        \int_{B \cap \D} (f_r^2 + \frac{1}{r^2}f_\theta^2)r drd\theta \le 2K.
    \end{equation}
    取$\rho$使得$\int_{\P B \cap \D} f^2_\theta(\rho e^{i\theta})$取到(几乎)最小, 则有
    \begin{equation}
        \int^{\sqrt{\delta}}_\delta \int_{\P B \cap \D}\frac{1}{r^2}f^2_\theta(\rho e^{i\theta}) rd\theta dr \le 2K.
    \end{equation}
    因此, 
    \begin{equation}
        \abs{f(z_1)-f(z_2)} \le \sqrt{2\pi} \frac{2K}{ \int^{\sqrt{\delta}}_\delta \frac{1}{r}} = \sqrt{8K\pi}(\log \frac{1}{\delta})^{-\frac{1}{2}}.
    \end{equation}
\end{proof}
\begin{lemma} \label{boundary_compactness}
    一致连续.
\end{lemma}
\begin{proof}%[定理\eqref{plateau}的证明]
    取$u_k \in \FG$ 且$E(u_k) \to \EG$. 用与$u_k$具有相同边界的调和函数替换$u_k$, 仍然记为$u_k$, 由于调和函数是能量最小的, 我们仍然有$E(u_k) \to \EG$. 固定$q_1,q_2,q_3 \in \gamma$及 $p_1,p_2,p_3 \in \P \D$, 取$\phi_k \in Aut(\D)$使得$u_k\circ \phi_k(p_i)=q_i$. 仍将$u_k \circ \phi_k$记作$u_k$. 由于$u_k$是调和的, 则
    \begin{equation}
        \max_{\D}\abs{u_k - u_l} = \max_{\P \D} \abs{u_k-u_l}.
    \end{equation}
    而根据引理\eqref{boundary_compactness}, $u_k\mid_{\P \D}$ 包含一致收敛子列, 则$u_k$包含一致收敛子列. 设$u_k \to u$.
\end{proof}