\chapter{Plateau问题}
\newcommand{\FG}{{\mathcal{F}_\gamma}}
\newcommand{\AG}{{\mathbb{A}_\gamma}}
\renewcommand{\EG}{{\mathbb{E}_\gamma}}
\renewcommand{\E}{{E}}
\renewcommand{\D}{{\mathbb{D}}}
设$\gamma \subset \R^3$是简单闭曲线. 记$\FG$是满足下列条件的函数的集合:
\begin{enumerate}
    \item $u: \overline{D} \to \R^3, u\in C(\overline{D})\cap W_{loc}^{1,2}(D)$.
    \item $u\mid_{\P D}$是到其像的同胚.
    %\item $\abs{u_x}^2=\abs{u_y}^2, \inner{u_x}{u_y}=0$.
\end{enumerate}
\begin{theorem}
    设$\gamma \subset \R^3$是可求长的Jordan曲线. 则存在映射$u \in D_\gamma$使得 $\forall v \in \FG$, 
    \begin{equation}
        \Area(u(D))\le \Area(v(D)).
    \end{equation}
\end{theorem}
对于任意$u \in \FG$, 其面积与能量分别定义为
\begin{align}
    &\Area(u)=\int_\D\abs{u_x\wedge u_y} \\
    &\E(u)=\frac{1}{2}\int_\D\abs{u_x}^2+\abs{u_y}^2
\end{align}
简单计算可知
\begin{equation}\label{AleE}
    \Area(u)=\int_\D \sqrt{\abs{u_x}^2\abs{u_y}^2-\inner{u_x}{u_y}^2} \le \frac{1}{2}\int_\D \abs{u_x}^2+\abs{u_y}^2 \le\E(u).
\end{equation}
并且等号成立, 当且仅当$\abs{u_x}=\abs{u_y}, \inner{u_x}{u_y}=0$.
\par 记
\begin{align}
    &\AG=\inf\{\Area(v)\mid v \in \FG\}.\\
    &\EG=\inf \{\E(v)\mid v \in \FG\}.
\end{align}
\begin{lemma}
    $\AG=\EG$.
\end{lemma}
\begin{proof}
    由不等式\eqref{AleE}可知, $\AG \le \EG$.
    \par 对于反方向的, 设$u\in \FG$ 且$\E(u) \le \EG + \epsilon$. 首先设$u$是浸入, 即$du$处处非退化. 设$(\D,g)$为$u$作用下的拉回度量, 即$g=du^*dx^2$, 由等温坐标的存在性可知, 存在光滑同胚$\phi: \D \to \D$使得$\phi$是$\mathbb{D} \to (\D,g)$ 之间的共形映射, 即$d\phi ^* g=\lambda^2dx^2$. 而$u\circ \phi$是共形浸入, 则有
    \begin{equation}
        \Area(u)=\Area(u\circ \phi)=\E(u\circ \phi) \ge \E(u)-\epsilon.
    \end{equation}
    如果$du$有奇点, 那么我们定义$u^s: \D \to \R^5$, $u^s(x,y)=(u,sx,sy)\in \R^5$. 则$du^s$是非退化的. 像上面一样, 通过$u^s$拉回的度量为$du^*g_{\R^3}+s^2(dx^2+dy^2)$. 显然地, 
    \begin{equation}
        \Area(u^s)=\int_\D \det (du^*g_{\R^3}+s^2I) \ge \Area(u).
    \end{equation}
    \begin{equation}
        \E(u^s\circ\phi)=\int_\D \abs{(u^s\circ \phi)_x}^2+ \abs{(u^s\circ \phi)_y}^2 =\E(u\circ \phi)+s^2\E(\phi).
    \end{equation}
    因此, 当$s$足够小时, 我们有
    \begin{equation}
        \Area(u) \le \Area(u^s\circ \phi) = \E(u^s\circ \phi) =\E(u\circ \phi) +s^2\E(\phi) \le \E(u)-\frac{1}{2}\epsilon
    \end{equation}
\end{proof}