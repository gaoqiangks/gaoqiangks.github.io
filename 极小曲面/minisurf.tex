\PassOptionsToPackage{quiet}{fontspec}
\documentclass[oneside]{ctexbook}
\usepackage{amsthm,amssymb,amsmath}
\usepackage{fancyhdr}
\usepackage{enumerate}
%\usepackage{biblatex}
\usepackage{biblatex}
\usepackage{graphicx}
\usepackage{picinpar}
\usepackage{subfig}
%\usepackage{epstopdf}
\usepackage{epsfig}
\usepackage{etoolbox}
\usepackage{showidx}
\usepackage{imakeidx}
\usepackage[colorlinks=true]{hyperref}
%tipa包用于arc命令
\usepackage{tipa}
\usepackage{listings}%排版代码的包
%\usepackage[yyyymmdd,hhmmss]{datetime}
\setlength{\headheight}{13pt}
\pagestyle{fancy}
%\rfoot{编译时间:  \today\  \currenttime}
%\cfoot{}
%\lfoot{页码:\thepage}
\newcommand{\norm}[1]{{\left | \left |  #1  \right | \right |}}
\newcommand{\inner}[2]{{\left < #1, #2 \right >}}
\newcommand{\abs}[1]{{\left | #1 \right |}}
\newcommand{\parallelsum}{{\mathbin{\!/\mkern-5mu/\!}}} %斜的平行线
\newcommand{\Hn}{{\mathbb{H}^n}}
\newcommand{\Sn}{{\mathbb{S}^n}}
\newcommand{\En}[1]{{\mathbb{R}^{#1}}}
\newcommand{\area}[1]{{\Area{#1}}}
\newcommand{\areap}{{\mathcal{A}_{+}}}
\newcommand{\arean}{{\mathcal{A}_{-}}}
\renewcommand{\div}{{\text{div}}}
\newcommand{\spt}{{\text{spt}}}
\newcommand{\tr}{{\text{tr}}}
\newcommand{\sing}{{\mathcal{S}}}
\newcommand{\proj}{{\text{Proj}}}
\newcommand{\diam}{{\text{diam}}}
\newcommand{\Tr}{{\text{Tr}}}
\newcommand{\XX}{{\vec{X}}}
\newcommand{\cl}[1]{\overline{#1}}
\newcommand{\eq}{{\equiv}}
%\newcommand{\C}{\mathbb{C}}
\newcommand{\R}{{\mathbb{R}}}
\newcommand{\nullset}{{\emptyset}}
\renewcommand{\H}{{\mathbb{H}}}
\newcommand{\N}{{\mathbb{N}}}
%\newcommand{\G}{\mathcal{G}}
\newcommand{\subsub}{{\subset \subset}}
\newcommand{\OR}{{\Omega \times \mathbb{R}}}
\newcommand{\OL}{{\Omega_L}}
\renewcommand{\epsilon}{\varepsilon}
\renewcommand{\phi}{{\varphi}}

\renewcommand{\l}{{\lambda}}
%\renewcommand{\L}{{\Lambda}}
\newcommand{\e}{{\epsilon}}
\newcommand{\E}{{\Epsilon}}
\renewcommand{\d}{{\delta}}
\newcommand{\g}{{\gamma}}
\newcommand{\G}{{\Gamma}}
\renewcommand{\a}{{\alpha}}
\newcommand{\A}{{\Alpha}}
\renewcommand{\b}{{\beta}}
\newcommand{\B}{{\Beta}}
\renewcommand{\u}{{\mu}}
%\newcommand{\U}{{\Mu}}
\renewcommand{\v}{{\nu}}
\newcommand{\V}{{\Vu}}
\newcommand{\x}{{\times}}
\renewcommand{\s}{{\quad}}
\renewcommand{\ss}{{\quad \quad}}
\newcommand{\sss}{{\quad \quad \quad}}
\renewcommand{\L}{\mathcal{L}}
\newcommand{\aij}{{a^{ij}}}
\newcommand{\dij}{{D_{ij}}}
\newcommand{\tto}[1]{{{\xrightarrow{#1}}}}
\renewcommand{\P}{{\partial}}
\newcommand{\PT}{{\partial_t}}
\newcommand{\PTF}{{\partial_t F}}
\newcommand{\PI}{{\partial_i}}
\newcommand{\PJ}{{\partial_j}}
\newcommand{\PIF}{{\partial_i F}}
\newcommand{\PJF}{{\partial_j F}}
\newcommand{\vt}{{d\mu_t}}
\newcommand{\vv}{{d\mu}}
\newcommand{\D}{{\mathbb{D}}}

\newcommand{\GG}{{\mathfrak{G}}}
\newcommand{\BR}{{B_r(x)}}
\newcommand{\GR}{{\GG \times \mathbb{R}}}
\newcommand{\EG}{{\mathfrak{E}}}
\newcommand{\HH}{{\tilde{H}}}
\newcommand{\BVG}{{BV(\GG,\psi)}}
\newcommand{\BVO}{{BV(\Omega)}}
\newcommand{\PO}{{\partial \Omega}}
\newcommand{\overl}[1]{{\overline{#1}}}
\newcommand{\ol}[1]{{\overline{#1}}}
\newcommand{\oo}{{\infty}}
\renewcommand{\O}{\Omega}
\newcommand{\Poincare}{\text{Poincar\'e}}
\newcommand{\Holder}{\text{{H\"older}}}
\newcommand{\reciprocal}[1]{{\frac{1}{#1}}}
\newcommand{\rec}[1]{{\reciprocal{#1}}}
\renewcommand{\e}{\text{e}}
\newcommand{\iin}{\text{ in }}
\newcommand{\supp}{\text{supp}}
\newcommand{\osc}{\mathop{\text{osc}}}
\newcommand{\loc}{\text{loc}}
\newcommand{\SSP}[1]{{\S^+(\lambda,\Lambda,#1)}}
\newcommand{\SSN}[1]{{\S^-(\lambda,\Lambda,#1)}}
\renewcommand{\SS}[1]{{\S(\lambda,\Lambda,#1)}}
\renewcommand{\S}{\mathbb{S}}
\renewcommand{\N}{{N}}
\newcommand{\Hull}{{\text{Conv}}}
\newcommand{\MN}[1]{{\mathcal{M}^-(\lambda,\Lambda,#1)}}
\newcommand{\MP}[1]{{\mathcal{M}^+(\lambda,\Lambda,#1)}}
%\renewcommand{\iff}{\Leftrightarrow}
\renewcommand{\iff}{\Longleftrightarrow}
\newcommand{\semiholder}[1]{{[#1]}}
\newcommand{\lsc}[1]{{\text{LSC}(#1)}}
\newcommand{\usc}[1]{{\text{USC}(#1)}}
\renewcommand{\G}{\text{Graph}}
\newcommand{\CC}{{C^\infty_0}}
\newcommand{\ddt}{\frac{d}{dt}}
\newcommand{\ddtt}{\frac{d^2}{dt^2}}
\newcommand{\ddr}{\frac{d}{dr}}
\newcommand{\ppt}{\frac{\partial}{\partial_t}}
\newcommand{\Area}{\text{Area}}
\newcommand{\MM}{{\tilde{\mathcal{M}}}}
\newcommand{\M}{{\mathcal{M}}}
\newcommand{\tangent}{{\text{T}}}
\newcommand{\Laplace}{{\Delta}}
\newcommand{\T}{{\perp}}
\newcommand{\DD}{{\tilde{\nabla}}}
\newcommand{\barpartial}{{\bar{\partial}}}
\newcommand{\Haus}{{\mathcal{H}}}
\newcommand{\Res}{{\text{Res}}}
\renewcommand{\Re}{{\mathfrak{Re}}}
\renewcommand{\Im}{{\mathfrak{Im}}}
\newcommand{\II}{{\Pi}}
\newcommand{\HLaplace}{{\Delta}}
\newcommand{\CLaplace}{{\underline{\Delta}}}
\newcommand{\TS}{{\tilde{\Sigma}}}
\newcommand{\CV}{{C^\infty_0}}
\renewcommand{\CC}{{C^\infty_c}}
\newcommand{\NT}{{\nabla}}
\newcommand{\arc}[1]{{\texttoptiebar{#1}}}
%\newcommand{\arc}[1]{{
%  \setbox9=\hbox{#1}%
%  \ooalign{\resizebox{\wd9}{\height}{\texttoptiebar{\phantom{A}}}\cr#1}}}

\newtheorem{theorem}{定理}[chapter]
\newtheorem*{theorem*}{定理}
\newtheorem{assumptions}{假设}
\newtheorem{lemma}[theorem]{引理}
\newtheorem{proposition}[theorem]{命题}
\newtheorem{exercise}[theorem]{习题}
\newtheorem{corollary}[theorem]{推论}
\newtheorem{definition}[theorem]{定义}
\newtheorem{question}[theorem]{问题}
\newtheorem*{question*}{问题}
\newtheorem{example}[theorem]{例}
\newtheorem*{example*}{例}
\newtheorem*{proposition*}{命题}
\theoremstyle{plain}
\newtheorem{remark}[theorem]{注}
%\newtheorem{claim}[theorem]{断言}
%\newtheorem*{claim*}{断言}

\newif\ifcomment
\def\comment{}
\def\begincomment{\ifdef{\comment}\noexpand\begingroup}
\def\endcomment{\noexpand\endgroup}

\newenvironment{subproof}[1][\proofname]{%
  \renewcommand{\qedsymbol}{$\diamond$}%
  \begin{proof}[#1]%
}{%
  \end{proof}%
}

\usepackage{enumitem,xparse}
\newlist{Claim}{description}{2}
\setlist[Claim]{labelindent=2em, leftmargin=*}
\newif\ifInsideClaim
\newcounter{claim}[theorem]
%\newcounter{cclaim}[claim]
\renewcommand\theclaim{\arabic{claim}}
%\renewcommand\thecclaim{\arabic{claim}.\arabic{cclaim}}
\let\originalqedsymbol\qedsymbol
\newenvironment{claim}{%
	% disable qed symbol if there is no star
	%\let\qedsymbol\relax%
	%\let\qedsymbol$\diamond$
	%\ifInsideClaim% we have a nested environment
	%\refstepcounter{cclaim}%
	%\let\theclaimcounter\thecclaim%
	%\else%
	\refstepcounter{claim}%
	\let\theclaimcounter\theclaim%
	%\InsideClaimtrue%
	%\fi%
	\Claim\item[\textbf{断言 \theclaimcounter:}]%
}{\endClaim\InsideClaimfalse\let\qedsymbol\originalqedsymbol}



\DeclareSourcemap{
  \maps[datatype=bibtex]{
    \map{
       \step[fieldset=URL, null]
       \step[fieldset=DOI, null]
       \step[fieldset=MRREVIERER, null]
       \step[fieldset=ISBN, null]
       \step[fieldset=ISSN, null]
       \step[fieldset=SERIES, null]
    }
  }
}

%\renewcommand\appendix{\par
%    \setcounter{section}{0}
%    \setcounter{subsection}{0}
%    \gdef\thesection{附录 \Alph{section}}}
%
\addbibresource{minisurf.bib}

\newif\ifshow
%\showtrue % comment out to hide answers

\newif\ifsee
\seetrue

\fancyfoot[RO]{\today}
%\fancyfoot[LE]{\today}

%\defbibheading{bibliography}[参考文献]{\chapter*{#1}}
\begin{document}
\title{课程:极小曲面}
%\author{主讲:高强}
\author{draft version}
\date{修改日期: {\today}}
%\date{2022年秋季学期}
\maketitle
\tableofcontents
%\input{Chapter01}
%\chapter{De Giorgi-Moser理论}
\section{希尔伯特第19问题}
设$L:\R^n \to \R$是光滑函数且是一致凸的. $\Omega \subset \R^n$是有界区域.  固定函数$g \in C(\overline{\O})$.  设
\begin{equation}
    \G=\{u \in H^1\mid u|_{\P \O} =g\}
\end{equation}
$\forall u \in \G$, 考虑下面的泛函
\begin{equation}
    \e(u)=\int_\O L(Du)dx
\end{equation}
\begin{question}
    若泛函$\e$在$u_0$处取到最小值,$u_0$是否是光滑的?
\end{question}
\begin{example}
    \s 
    \begin{enumerate}
        \item $L(x)=\abs{x}^2$,则$\e(u)=\int_\O \abs{Du}^2$,称为Dirichlet能量,是研究调和函数的基础工具.
        \item $L(x)=\sqrt{1+\abs{x}^2}$,则 $\e(u)=\int_\O\sqrt{1+\abs{Du}^2}$,称为面积泛函.因此此时$\e(u)$恰好是函数$u$的图像的面积.
    \end{enumerate}
\end{example}
一些简单的计算可以将上面的变分问题转化为PDE问题.
\begin{proposition}
    若泛函$\e$在函数$u$处取到最小值,则$u$是下列方程的弱解
    \begin{equation}
        \div(DL(Du))=0.
    \end{equation}
\end{proposition}
\begin{proof}
    $\forall \phi \in C^\infty_0(\Omega)$,考虑$u$的变分$e_t=e(u+t\phi)$. 由于$\e$在$t=0$处取到最小值 ,则$\frac{d}{dt}e_t|_{t=0}=0$.而
    \begin{equation}
        e_t=\int_\O L(Du+tD\phi)=\int_\O L(Du)+t\int_\O DL(Du)D\phi+o(t)
    \end{equation}
    因此$\frac{d}{dt}e_t|_{t=0}=\int_\O DL(Du)D\phi=0 \s \forall \phi\in C^\infty_0(\O)$成立.证毕.
\end{proof}
\begin{remark}
    泛函的一阶变分为0所要满足的方程称为Euler-Lagrange方程.显然,这是泛函取到局部最小值的必要非充分条件.
\end{remark}
现在, 们取定$\vec{n} \in \mathbb{S}^{n-1}$,并对$\div(DL(Du))=0$求偏导$D_{\vec{n}}$.
\begin{equation}
    \begin{split}
        D_{\vec{n}}(DL(Du))
        &=D_{\vec{n}}(\frac{\partial L}{\partial x_1}(Du),\cdots,\frac{\partial L}{\partial x_n}(Du))\\
        &=(\frac{\partial^2 L}{\partial x_1 \partial x_i} D_{\vec{n}}D_iu,\cdots,\frac{\partial^2 L}{\partial x_n \partial x_i} D_{\vec{n}}D_iu)
    \end{split}
\end{equation}
因此,$D_{\vec{n}}\div(DL(Du))=\div(D^2L(Du)DD_{\vec{n}}u)$. 记$\aij=\frac{\partial^2L}{\partial x_i \partial x_j}(Du)$, $v=D_{\vec{n}}u$,则上面的方程成为 
\begin{equation}
    D_i(\aij D_j v)=0
\end{equation}
\subsection{齐次方程弱解的\texorpdfstring{\Holder}{Holder}正则性}
我们考虑齐次方程
\begin{equation} \label{heq}
    \L u= D_i(\aij D_j u)=0
\end{equation}
\begin{theorem} \label{hholder}
    设$\aij$可测且满足一致椭圆条件.设$u \in H^1(\O)$是方程\eqref{heq}的弱解.则存在$\alpha \in (0,1)$使得$u \in C^\alpha(\O)$,且$\forall \O' \subsub \O$ ,有
    \begin{equation}
        \norm{u}_{C^\alpha(\O')} \le C\norm{u}_{L^2(\O)}
    \end{equation}
    其中,$\alpha=\alpha(n,\lambda,\Lambda)$, $C=C(n,\lambda,\Lambda,\O',\O)$.
\end{theorem}
\begin{definition}
    若$\forall \phi \ge 0$, $\phi \in H^1_0$, 成立$\int \aij D_iu D_j \phi \le 0$,则称$u$是方程\eqref{heq}的弱下解. \\
    若$\forall \phi \le 0$, $\phi \in H^1_0$, 成立$\int \aij D_iu D_j \phi \ge 0$,则称$u$是方程\eqref{heq}的弱上解.
\end{definition}
\begin{lemma}[Caccioppoli不等式]\label{caccioppoli}
    设$u \ge 0$是方程\eqref{heq}在$B_1$中的弱下解,则$\forall \phi \in C^\infty_0(B_1)$,成立
    \begin{equation}
        \int_{B_1}\abs{D(u\phi)}^2 \le C\norm{D\phi}^2_{L^\infty}\int_{B_1 \cap \text{supp}\phi}u^2
    \end{equation}
    其中,$C=C(n,\lambda,\Lambda)$.
\end{lemma}
\begin{proof}
    证明过程与不等式\eqref{eq_caccio}相同,略.
\end{proof}
\begin{lemma}
    设$u$是方程\eqref{heq}的弱下解.记$u^+=\max(u,0)$.则$u^+$也是方程\eqref{heq}的弱下解.
\end{lemma}
\begin{proof}
    记$f(x)=\max(x,0)$.则$u^+=f(u)$. 显然,$f$是单增,非负的凸函数. 设$f_\epsilon$是$f$的磨光, 则$f_\epsilon$也是单增,非负的凸函数.我们首先证明$f_\epsilon \circ u$是弱下解. 为了记号方便,记$F(x)=f_\epsilon(x)$.则 $\forall \phi \ge 0$,
    \begin{equation}
        \begin{split}
            \int \aij D_i(F\circ u) D_j \phi 
            &= \int \aij F'(u)D_i(u)D_j\phi\\
            &=\int \aij D_iu(D_j(F'(u)\phi)-F''(u)D_ju\phi)\\
            &\le 0
        \end{split}
    \end{equation}
    令$\epsilon \to 0$.则由勒贝格控制收敛定理可知
    \begin{equation}
        \int \aij D_i(f\circ u) D_j\phi \le 0 \s \forall \phi \ge 0
    \end{equation}
    引理2证毕.
\end{proof}
现在我们可以证明弱解的局部有界性.
\begin{lemma}[局部有界性]
    设$u$是方程\eqref{heq}的弱下解,则
    \begin{equation}
        \norm{u^+}_{L^\infty(B_\rec{2})} \le C\norm{u^+}_{L^2(B_1)}
    \end{equation}
    其中,$C=C(n,\lambda,\Lambda)$.
\end{lemma}
\begin{proof}
    由于给$u$乘上任意正常数不会改变不等式.我们证明下面的等价命题:
    \begin{proposition*}
        存在$\delta = \delta(n,\lambda,\Lambda)>0$使得若 $\norm{u^+}_{L^2(B_1)} \le \delta$,则$\norm{u^+}_{L^\infty(B_\rec{2})} \le 1$. 
    \end{proposition*}
    设$k  \in \mathbb{N}^+$. 记$B_k=\{\abs{x} \le \rec{2}+\rec{2^k}\}$. 取$\eta_k \in C^\infty_0(B_1)$ 使得
    \begin{equation}
        \left\{
            \begin{aligned}
                &\eta_k\eq 1 \s \iin B_k \\
                &\eta_k \eq 0 \s \iin B_{k-1}-B_k\\
                &\abs{D\eta_k} \le {2}\cdot {2^k}
            \end{aligned}
        \right.
    \end{equation}
    设$u_k=(u-(1-\rec{2^k}))^+$, $a_k=\int_{B_k}u_k^2$.
    \begin{claim}
        若$a_1$足够小,则$k \to \infty$时,$a_k \to 0$. 即,$\int_{B_\rec{2}}\abs{(u-1)^+}^2 =0$.
    \end{claim}
    显然,若此断言成立,则引理得证. 现在,我们来证明存在$M,\lambda >1$使得$a_k$满足如下的递推不等式
    \begin{equation} \label{iter_eq}
        a_k \le M^{k-1}a^\lambda_{k-1}
    \end{equation}
    \begin{equation}
        a_k=\int_{B_k}u^2_k \le (\int_{B_k}u_k^{2^*})^\frac{2}{2^*} \abs{\supp u^2_k \cap B_k}^\frac{2}{n}
    \end{equation}
    而由Sobolev不等式及Caccioppoli不等式,有 
    \begin{equation}
        \begin{split}
            (\int_{B_k}u_k^{2^*})^\frac{2}{2^*}
            &\le (\int_{B_{k-1}}\abs{\eta_k u_k}^{2^*})^\frac{2}{2^*}\\
            &\le C\int_{B_{k-1}}\abs{D(\eta_k u_k)}^2\\
            &\le C2^{2k}\int_{B_{k-1}}u_k^2 \\
            &\le C2^{2k}\int_{B_{k-1}}u_{k-1}^2=C2^{2k}a_{k-1}
        \end{split}
    \end{equation}
    简单计算可知,$u_k>0$当且仅当$u_{k-1}> \rec{2}$. 则由切比雪夫不等式可知
    \begin{equation}
        \begin{split}
            \abs{\supp u^2_k \cap B_k}&=\abs{\{u_k>0\}\cap B_k} \\
            &\le \abs{\{u_{k-1}> \rec{2}\} \cap B_{k-1}} \\
            &\le 2^{2k}\int_{B_{k-1}}u^2_{k-1} =2^{2k}a_{k-1}
        \end{split}
    \end{equation}
    因此,
    \begin{equation}
        a_k \le C2^{2k}a_{k-1}(2^{2k}a_{k-1})^\frac{2}{n} 
    \end{equation}
    取$\lambda=1+\frac{2}{n}$, $M=2\max\{C,3^{2+\frac{4}{n}}\}$即可得到不等式\eqref{iter_eq}.  现在,我们有$a_k \le M^{k-1}a_{k-1}^\lambda$,递推可知
    \begin{equation}
        \begin{split}
            a_{k+1} &\le M^ka_k^\lambda\\
            &\le M^k(M^{k-1}a_{k-1}^\lambda)^\lambda =M^kM^{(k-1)\lambda}a_{k-1}^{\lambda^2}\\
            &\le M^k M^{(k-1)\lambda}({{M^{k-2} a_{k-2}^\lambda}})^{\lambda^2}=M^kM^{(k-1)\lambda}M^{(k-2)\lambda^2}a^{\lambda^3}_{k-2}\\
            &\le M^kM^{(k-1)\lambda}M{(k-2)\lambda^2}\cdots M^{2\lambda^{k-2}}M^{\lambda^{k-1}}a_1^{\lambda^k}\\
            &=M^{\frac{k-\frac{\lambda-\lambda^k}{1-\lambda}}{1-\lambda}}a_1^{\lambda^k}
        \end{split}
    \end{equation}
    因此,只要取$a_1$足够小,则$a_k \to 0$.
\end{proof}
\begin{lemma}[振荡衰减]
    设$u$是方程\eqref{heq}在$B_2$中的弱解. 记
    \begin{equation}
        \osc_{B_r} u=\sup_{B_r} u - \inf_{B_r} u. 
    \end{equation}
    则存在$\theta \in (0,1)$使得 
    \begin{equation}
        \osc_{B_\rec{2}} u \le \theta \osc_{B_2}u. 
    \end{equation}
    其中,$\theta=\theta(n,\lambda,\Lambda)$.
\end{lemma}
\begin{proposition}\label{p1}
    设$w \in H^1(B_1)$. 记$A=\{w \le 0\}\cap B_1$. $D=\{w\ge \rec{w}\} \cap B_1$. $E=\{0<w<\rec{2}\} \cap B_1$. 则存在常数$C=C(n)$使得
    \begin{equation}
        \abs{A}^2\abs{D}^2 \le C\abs{E}\int_{B_1}\abs{Dw}^2
    \end{equation}
\end{proposition}
\begin{proof}
    记$\tilde{w}(x)=\left\{
        \begin{aligned}
            &0 \s x \in A \\
            &w(x) \s x \in E \\
            &\rec{2} \s x\in D
        \end{aligned}
    \right.$. $\tilde{w}_{B_1}=\frac{1}{\abs{B_1}}\int_{B_1}\tilde{w}$. 则有
    \begin{equation}
        \begin{split}
            \abs{A}\abs{D} &\le 2\int_A\int_D\abs{\tilde{w}(x)-\tilde{w}(y)}dxdy\\
            &\le 2\int_{B_1}\int_{B_1}\abs{\tilde{w}(x)-\tilde{w}_{B_1}} +\abs{\tilde{w}(y)-\tilde{w}_{B_1}} \\
            &=4\int_{B_1}\abs{\tilde{w}(x)-\tilde{w}_{B_1}} dx \\
            &\le C\int_{B_1}\abs{D\tilde{w}}dx \\
            &=C\int_E\abs{Dw}dx \le C(\int_E\abs{Dw}^2)^\rec{2}\abs{E}^\rec{2}
        \end{split}
    \end{equation}
\end{proof}
\begin{proposition}
    设$u$是方程\eqref{heq}在$B_2$中的弱下解且$u \le 1$. 设$u$满足
    \begin{equation}
        \abs{\{u \le 0\}\cap B_1} \ge \mu >0
    \end{equation}
    则$\sup_{B_\rec{2}}u \le 1-\gamma$. 其中,$\gamma=\gamma(n,\lambda,\Lambda,\mu) \in (0,1)$.
\end{proposition}
\begin{proof}
    记$w_k=w^k(u-(1-\rec{2}))^+$.则
    \begin{equation}
        \abs{\{w_k \le 0\} \cap B_1} =\abs{\{u \le 1-\rec{2^k}\}\cap B_1} \ge \abs{\{u \le 0\}\cap B_1} \ge \mu
    \end{equation}
    现在,假设存在$\delta >0$ 使得$\forall k$, 成立$\int_{B_1}w_k^2 \ge \delta >0$,则有
    \begin{equation}
        \abs{\{w_k \ge \rec{2}\}\cap B_1} = \abs{\{w_{k+1}\ge 0\} \cap B_1} \ge \int_{B_1}w^2_{k+1} \ge \delta >0
    \end{equation}
    应用命题\eqref{p1},可知
    \begin{equation}
        \abs{\{0<w_k <\rec{2}\}} \ge C(n)\rec{\norm{Dw_k}_{L^2}}\mu^2\delta^2
    \end{equation}
    而由Caccioppoli不等式,
    \begin{equation}
        \int_{B_1}\abs{Dw_k}^2 \le C\int_{B_2}w_k^2 \le C
    \end{equation}
    因此, $\abs{\{0<w_k<\rec{2}\}}$ 关于$k$有一致下界.然而简单计算可知,$\{0<w_k < \rec{w}\}$是互不相交的,因此假设不成立. 即,$\forall \delta >0$, $\exists k_0$使得$\int_{B_1}w_{k_0}^2 \le \delta$. 由于$w_k$也是方程\eqref{heq}的弱下解,则由局部有界性估计,可知
    \begin{equation}
        \norm{w_{k_0}}_{L^\infty(B_\rec{2})} \le C\int_{B_1}w_{k_0}^2 \le C\delta
    \end{equation}
    取$\delta$足够小使得$C\delta \le \rec{2}$,则有 $\norm{w_{k_0}}_{L^\infty(B_\rec{2})} \le \rec{2}$.即
    \begin{equation}
        2^{k_0}(u-(1-\rec{2^{k_0}}))^+ \le \rec{2}
    \end{equation}
    则$u \le 1-\rec{2^{k_0+1}}$.
\end{proof}
\begin{proof}[振荡衰减的证明]
    显然地,通过$u \to \frac{u-N}{\sup_{B_2}-N} (N < \sup_{B_2}u)$的代换可知,若$\abs{\{u \le N\}\cap B_1} \ge \mu >0$,则$\sup_{B_\rec{2}} \le (1-\gamma)(\sup u-N)+N$. 
    同样的,由于$-u$也是弱下解,则若$\abs{\{-u \le N\}\cap B_1} \ge \mu >0$, 则$\sup_{B_\rec{2}}(-u) \le (1-\gamma)(\sup(-u)-N)+N$.  现在,取$N=\rec{2}(\sup_{B_2}u+\inf_{B_2}u)$. \\
    若$\abs{\{u \le N\}\cap B_1} \ge \rec{2}\abs{B_1}$,则
    \begin{equation}
        \begin{split}
            \sup_{B_\rec{2}}-\inf_{B_\rec{2}}u &\le (1-\gamma)(\sup_{B_2}u-\rec{2}(\sup_{B_2}u+\inf_{B_2}u))+\rec{2}(\sup_{B_2}+\inf_{B_2}u)-\inf_{B_2}u\\
            &=(1-\gamma)\rec{2}\osc_{B_2}u+\rec{2}\osc_{B_2}u =(1-\frac{\gamma}{2})\osc_{B_2}u
        \end{split}
    \end{equation}
    否则,取$\tilde{{N}}=\rec{2}(\sup_{B_2}(-u)+\inf_{B_2}(-u))=-N$即可.
\end{proof}
\begin{proof}[定理\eqref{hholder}的证明]
    记$w(x)=\frac{u(x)}{2\norm{u}_{L^\infty}(B_1)}$. 则$\abs{w} \le \rec{2}$. 设$x,y \in B_{\rec{4}}$.通过两移及两倍伸缩,我们可以假设$y=0$且$\abs{x} \le \rec{2}$. 设$\rec{2^{k-1}} \le \abs{x} < \rec{2^k}$. 则
    \begin{equation}
        \abs{w(x)-w(y)} \le \abs{w(x)-w(0)} \le \osc_{B_\rec{2^k}}w\le(1-\theta)^k\osc_{B_1}w\le (1-\theta)^k
    \end{equation}
    令$(1-\theta)^k=(\rec{2^\alpha})^k$, 即$\alpha=-\log_2(1-\theta)$,则有
    \begin{equation}
        \abs{w(x)-w(y)} \le \rec{2^k}^\alpha \le C\abs{x}^\alpha.
    \end{equation}
    证毕.
\end{proof}
\subsection{希尔伯特第19问题的可解性}
\begin{theorem}
    设$L:\R^n \to \R$是光滑函数,且是一致凸的($D^2L$满足一致椭圆条件).设$ \O \subset \R^n$是有界Lipschitz区域.设$g \in H^1(\O)$.则泛函
    \begin{equation}
        \e(u)=\int_\O L(Du)dx
    \end{equation}
    在$\{u \in H^1(\O), u|_{\partial \O}=g|_{\partial \O}\}$中有唯一的最小值点$u$,且$u \in C^1(\O)$.
\end{theorem}
\begin{lemma}[下半连续性] \label{lower_semi}
    设$\O \subset \R^n$是有界Lipschitz区域. 设$F(x,z,p): \R^n \times \R \times \R^n \to \R$是光滑的,且满足
    \begin{enumerate}
        \item $F \ge 0$.
        \item $F$关于$p$是凸的.
        \item $F(x,z,p) \le C(1+\abs{p}^q)$, $q \in [1,+\infty)$, $C>0$.
    \end{enumerate}
    对于$u \in w^{1,q}$,定义泛函$e(u)=\int_\O F(x,u,Du)dx$.则泛函$\e$关于$w^{1,q}_\loc$的弱拓扑是下半连续的.即,若$u_k \in w^{1,q}_\loc(\O)$且弱收敛到$u \in w^{1,q}_\loc$,则
    \begin{equation}
        e(u) \le \liminf e(u_k).
    \end{equation}
\end{lemma}
\begin{proof}
    由于$w^{1,q}_\loc \subset w^{1,1}_\loc$, 我们可以假设$q=1$. 通过选取$u_k$的子列,假设$u_k \to u \s a.e.$并且$u_k \to u \iin w^{1,1}_\loc$.\\
    固定$\O' \subsub \O$及$\epsilon >0$. 则由Egorov定理及Lusin定理,存在紧集$K \subset \O'$使得
    \begin{enumerate}
        \item $\abs{\O'-K} \le \epsilon$.
        \item $u_k \to u$在$K$上是一致的.
        \item $u|_K$, $Du|_{K}$是连续的.
        \item $\int_K F(x,u,Du) \ge \int_{\O'}F(x,u,Du) - \epsilon$.
    \end{enumerate}
    由$F$关于$p$的凸性,我们有
    \begin{equation}
        F(x,z,p) \ge F(x,z,p_0)+D_p F(x,z,p_0)(p-p_0)
    \end{equation}
    于是,
    \begin{equation}
        \begin{split}
            \int_KF(x,u_k,Du_k)dx \ge& \int_KF(x,u_k,Du)+\int_K\inner{D_pF(x,u_k,Du)}{Du_k-Du}\\
            =&\int_KF(x,u_k,Du)+\int_K\inner{D_pF(x,u,Du)}{Du_k-Du}\\ 
            &+\int_K\inner{D_pF(x,u_k,Du)-D_pF(x,u,Du)}{Du_k-Du}
        \end{split}
    \end{equation}
    \begin{enumerate}
        \item 由于$u_k\to u \iin w^{1,1}_\loc$且$D_pF(x,u,Du)$在$K$上有界,则右(II)$\to 0$.
        \item 由于$u_k \to u$一致且$D_pF$连续,而$Du_k-Du$在$L^1$上有界,则右(III)$\to 0$.
    \end{enumerate}
    令$k \to \infty$,则有
    \begin{equation}
        \liminf \int_K F(x,u_k,Du_k) \ge \lim \int_K F(x,u_k,Du)\ge \int_\O F(x,u,Du)-\epsilon
    \end{equation}
    而由于$\epsilon$是任意小的,引理证毕.
\end{proof}
\begin{lemma}\label{holderh7}
    设$\alpha \in (0,1]$, $\norm{u}_{L^\infty(B_1)} \le M$, 且$\forall h \in B_1$,有
    \begin{equation}
        \norm{\frac{u(x+h)-u(x)}{\abs{h}^\alpha}}_{C^\beta({B_{1-\abs{h}}})} \le M
    \end{equation}
    另外,设$\alpha +\beta$不是整数,则$u \in C^{\alpha+\beta}(B_1)$.
\end{lemma}
\begin{proof}
    解的唯一性由$L$的凸性易证。下面证明最小值点的存在性。由$L$的一致凸性,存在$\lambda, \Lambda>0$使得$\lambda \abs{x}^2 \le \inner{D^2Lx}{x} \le \Lambda \abs{x}^2$.记$h(x)=L(x)-\rec{4}\lambda\abs{x}^2$,显然,$D^2h=D^2L-\rec{2}\lambda I \ge \rec{2}\lambda I$. 于是$h$是一致凸的.因此存在常数$C$使得$L(x)-\rec{4}\lambda\abs{x}^2 \ge C$. 同样地,记$H(x)=\Lambda\abs{x}^2 - L(x)$,则$D^2H=2\Lambda I-D^2L \ge \Lambda I$.于是 $h$是一致凸的,则存在$C_2$使得$\Lambda \abs{x}^2-L(x) \ge C_2$. 即有
    \begin{equation}
        \rec{4}\lambda \abs{x}^2 +C_1 \le L(x) \le \Lambda \abs{x}^2+C_2
    \end{equation}
    记
    \begin{equation}
        e_0=\inf\{ e(w)| w\in H^1(\O), w|_{\partial \O}=g|_{\partial \O} \}
    \end{equation}
    显然,$e_0 \in (-\infty,+\infty)$. 设$u_k \in H^1(\O), u_k|_{\partial \O}=g$ 且$e(u_k) \to e_0$.
    \begin{claim}
        ${u_k}$在$H^1(\O)$中有界.
    \end{claim}
    由于$u_k|_{\partial \O}=g|_{\partial \O}$,则$u_k -g \in H^1_0(\O)$.由Poincare不等式 
    \begin{equation}
        \int_\O \abs{u_k-g}^2 \le C\int_\O \abs{Du_k-Dg}^2
    \end{equation}
    则有:
    \begin{equation}
        \int_\O \abs{u_k}^2 \le C(\int_\O \abs{Du_k}^2+\int_\O \abs{Dg}^2 +\int_O g^2)
    \end{equation}
    而由于 $\abs{x}^2 \le C(L(x)+1)$,则有
    \begin{equation}
        \int_\O \abs{Du_k}^ \le C(\int_\O L(Du_k)+\abs{\O}) \le C(e_0+1+\abs{\O})
    \end{equation}
    于是,$\int_\O \abs{u^k}^2$有界,断言证毕.\\
    由于Hilbert空间中的有界集是弱紧的,则存在$\{u_k\}$的子列,仍记为$\{u_k\}$使得$u_k \to u$(弱收敛). 由引理\eqref{lower_semi}可知,$e(u) \le \liminf e(u_k)$,于是 $u$就是所要求的最小值点. 而我们已知$u$一定是方程
    \begin{equation}
        \int_\O DL(Du)D\phi=0 \s \forall \phi \in H^1_0(\O)
    \end{equation}
    记$\O_h=\{x\mid x\in\O, d(x,\partial \O) >h\}$.显然,$\forall x \in \O_h$, $\forall \phi \in H^1_0(\O)$,
    \begin{equation}
        \int_\O \inner{DL(Du(x+h))-Dl(Du(x))}{D\phi}=0
    \end{equation}
    另外,由于
    \begin{equation}
        f(x)-f(y)=\int^1_0 \inner{Df(tx+(1-t)y)}{x-y}dt
    \end{equation}
    则有:
    \begin{equation}
        \begin{split}
            &DL(Du(x+h))-DL(Du(x)) \\
           =& \int^1_0 D^2l(tDu(x+h)+(1-t)Du(x))(Du(x+h)-Du(x))dt
        \end{split}
    \end{equation}
    记$A(x)=[\aij]=\int^1_0 D^2L(tDu(x+h)-(1-t)Du(x))(Du(x+h)-Du(x))dt$, 则
    \begin{equation}
        \int \aij(Du(x+h)-Du)_iD_j\phi=0 \s \forall \phi \in H^1_0(\O_h)
    \end{equation}
    令$v_h(x)=\frac{u(x+h)-u(x)}{h}$,则$v_h$满足方程
    \begin{equation}
        \int \aij D_i v_h D_j\phi=0
    \end{equation}
    由De Giorgi估计可知
    \begin{equation}
        \norm{v_h}_{C^\alpha(\O')} \le C\norm{Du}_{L^2(\O)}
    \end{equation}
    由引理\eqref{holderh7}可知,$ u \in C^{1,\alpha}$.
\end{proof}
\section{局部有界性}
从现在开始,我们考虑方程
\begin{equation} \label{eq2}
    \int_{B_1} \aij D_iu D_j \phi + cu\phi = \int_{B_1}f\phi
\end{equation}
\begin{theorem}\label{local_boundedness}
    设$u$是方程\eqref{eq2}的弱下解,即$\forall \phi \in H^1_0(B_1), \phi \ge 0$成立
    \begin{equation}
        \int_{B_1} \aij D_i D_j \phi + cu\phi \le \int_{B_1}f\phi
    \end{equation}
    这里,$\aij$满足$(\lambda,\Lambda)$一致椭圆条件, $c,f\in L^1(B_1)$且$q > \frac{n}{2}, \norm{c}_{L^q(B_1)} \le \Lambda$. 则 $u^+ \in L^\infty_{loc}$且 $\forall \theta \in (0,1), p >0$,成立
    \begin{equation}
        \sup_{B_\theta}u^+ \le c(n,\lambda,\Lambda,p,q)\{(\rec{1-\theta})^{\frac{n }{p}}\norm{u^+}_{L^p(B_1)}+\norm{f}_{L^q(B_1)}\}
    \end{equation}
\end{theorem}
\subsection{De Giorgi方法}
\begin{proof}[局部有界性的De Giorgi迭代证明]
    首先假设$\theta=\rec{2},p=2$. 一般情况将通过一个绅缩变换来得到. 记$u_k=(u-k)^+$.则
    \begin{equation}
        Du_k=\left\{
            \begin{aligned}
                &Du \s u>k \\
                &0 \s u \le k
            \end{aligned}
        \right.
    \end{equation}
    取$\eta \in C^\infty_0(B_1)$. 取$\phi=u_k\eta^2$代入到方程中,则有
    \begin{equation} \label{minequa1}
        \int \aij D_iu(D_ju_k)\eta^2 \le -2 \int \aij D_iuD_j\eta u_k \eta - \int cuu_k\eta^2+\int fu_k\eta^2
    \end{equation}
    根据一致椭圆性质及{\Holder}不等式,我们分别有
    \begin{equation} \label{m1}
        \text{左} = \int \aij D_i u_k D_j u_k \eta^2 \ge \lambda\int \abs{Du_k}^2\eta^2
    \end{equation}
    \begin{equation}\label{m2}
        \begin{split}
            \rec{2}\text{右(I)} \le & \int \abs{\aij D_i u_k D_j \eta u_k \eta} \\
            \le & \Lambda \int\abs{Du_k}\abs{\eta} \abs{D\eta} u_k \\
            \le & \Lambda (\epsilon \int \abs{Du_k}^2\eta^2 + \rec{\epsilon}\int \abs{D\eta}^2u_k^2)
        \end{split}
    \end{equation}
    \begin{equation} \label{m3}
        \begin{split}
            \abs{\text{右(II)}}  \le & \int \abs{c}(u_k+k)u_k\eta^k \\
            \le  & C(\int\abs{c}u^2_k \eta^2+k^2\int \abs{c}\eta^2)
        \end{split}
    \end{equation}
    在\eqref{m2}中取$\epsilon$使得 $2\Lambda \epsilon \le \rec{2}\lambda$,并将\eqref{m1},\eqref{m2},\eqref{m3}代入到不等式\eqref{minequa1}中,则有
    \begin{equation}
        \int \abs{Du_k}^2\eta^2 \le C(\int \abs{D\eta}^2u_k^2+\int \abs{c}u_k^2 \eta^2 + k^2 \int \abs{c}\eta^2+\int \abs{f}u_k\eta^2)
    \end{equation}
    再由
    \begin{equation}
        \int \abs{D(u_k\eta)}^2 \le 2(\int\abs{Du_k}^2\eta^2+\int u_k^2\abs{D\eta}^2)
    \end{equation}
    可得到Caccioppoli不等式,
    \begin{equation} \label{cacc}
        \int \abs{D(u_k\eta)}^2 \le C(\int \abs{D\eta}^2u_k^2+\int \abs{c}u_k^2\eta^2 + k^2\int \abs{c}\eta^2+ \int \abs{f}u_k\eta^2)
    \end{equation}
    由于$k \to \infty$时,$\abs{\{u_k\eta \ne 0\}} \to 0$. 从现在开始,我们总是假设$k$足够大,也就是$\abs{\{u_k\eta \ne 0\}}$足够小. 现在,分别估计不等式\eqref{cacc}中的右侧各项.   由{\Holder}不等式及Sobolev不等式可知:
    \begin{equation}
        \begin{split}
            \text{右(II)} =& \int \abs{c}{u_k^2}\eta^2 \\
            \le & (\int \abs{c}^q)^\rec{q}( \int \abs{u_k\eta}^{2^*})^\frac{2}{2^*} \abs{\{ u_k \eta \ne 0\}}^{1-\frac{2 }{2^*}-\rec{q}} \\
            \le & (\int \abs{c}^q)^\rec{q} \int  \abs{D(u_k\eta)}^2 \abs{\{ u_k \eta \ne 0\}}^{\frac{2}{n}-\rec{q}}
        \end{split}
    \end{equation}
    \begin{equation}
        \text{右(III)}= k^2 \int \abs{c}\eta^2 \le k^2(\int \abs{c}^q)^\rec{q}\abs{\{u_k\eta \ne 0\}}^{1-\rec{q}}
    \end{equation}
    \begin{equation}
        \begin{split}
            \text{右(IV)} = \int \abs{f}u_k\eta^2 \le & \int \abs{f}u_k\eta \\
            \le & (\int \abs{f}^q)^\rec{q} (\int \abs{u_k\eta}^{2^*})^\rec{2^*} \abs{\{u_k\eta \ne 0\}}^{1-\rec{2^*}  -\rec{q}} \\
            \le & c(\int \abs{f}^q)^\rec{q}( \int \abs{D(u_k\eta)}^2)^\rec{2}\abs{\{u_k\eta \ne 0\}}^{1-\rec{2^*}-\rec{q}} \\
            \le &\delta \int \abs{D(u_k\eta)}^2+ \frac{C}{\delta}(\int \abs{f}^q)^\rec{q}\abs{\{u_k\eta \ne 0\}}^{1+\frac{2}{n}-\frac{2}{q}}
        \end{split}
    \end{equation}
    缩合以上估计,我们有
    \begin{equation} \label{t263}
        \begin{split}
            &\int \abs{D(u_k\eta)}^2  \\
            \le & C(\int u_k^2\abs{D\eta}^2 + \norm{c}_{L^q} \int \abs{D(u_k\eta)}^2 \abs{\{u_k\eta \ne 0\}}^{\frac{2 }{n}-\rec{q}} \\
            &+k^2\norm{c}_{L^q}\abs{\{u_k\eta\ne 0\}}^{1-\rec{q}} + \delta \int \abs{D(u_k\eta)}^2 + C(n,\delta)\norm{f}_{L^q}^2\abs{\{u_k\eta \ne 0\}}^{1+\frac{2 }{n}-\frac{2 }{q}})
        \end{split}
    \end{equation}
    现在, 可以选取 $k$足够大及$\delta$足够小, 使得不等式\eqref{t263}右侧含有$\int\abs{D(u_k\eta)}^2$的项可以被左侧约去.  而由于
    \begin{equation}
        \abs{\{u>k\}} \le \rec{k} \int (u-k)^+ \le C\rec{k}(\int \abs{u_k}^2)^\rec{2}
    \end{equation}
    可设$k>k_0=C\abs{u^+}_{L^2}$.于是,取$\delta$足够及$k>k_0$,则有
    \begin{equation} \label{meq5}
        \begin{split}
            &\int \abs{D(u_k\eta)}^2  \\
            \le &C(\int u_k^2 \abs{D\eta}^2+k^2\norm{c}_{L^q}\abs{u_k\eta\ne 0}^{q-\rec{q}}+\norm{f}^2_{L^q}\abs{u_k\eta\ne 0}^{1+\frac{2}{n}-\frac{2 }{q}}) \\
            \le & C(\int u_k^2\abs{D\eta}^2+(k^2+\norm{f}^2_{L^q})\abs{u_k\eta \ne 0}^{1-\rec{q}})
        \end{split}
    \end{equation}
    现在,我们利用不等式\eqref{meq5}来估计 $\int \abs{u_k \eta}^2$. 由Holder不等式及Sobolev不等式可知
    \begin{equation} \label{meq6}
        \begin{split}
            \int\abs{u_k \eta}^2 \le & (\int \abs{u_k\eta}^{2^*})^{\frac{2}{2^*}}\abs{\{u_k \eta \ne 0\}}^{1-\frac{2 }{2^*}} \\
            \le & C \int \abs{D(u_k\eta)}^2 \abs{\{u_k \eta \ne 0 \}}^{1-\frac{2 }{2^*}} \\
            \le & C(\int u_k^2 \abs{D\eta}^2 \abs{u_k\eta \ne 0}^\frac{2}{n} + (k+\norm{f}_{L^q})^2 \abs{u_k \eta \ne 0}^{1-\rec{q}+\frac{2 }{n}}) \\
            \le & C(\int \abs{u_k D\eta}^2\abs{u_k\eta\ne 0}^\epsilon + (k+\norm{f}_{L^q})^2\abs{\{u_k \eta \ne 0\}}^{1+\epsilon})
        \end{split}
    \end{equation} 
    这里, $ \abs{u_k\eta \ne 0}, \epsilon$都取足够小.  $\forall  0 < r < R < 1$, 取$ \eta \in C^\infty_0(B_R)$使得 
    \begin{equation}
        \left\{
        \begin{aligned}
            & \eta(x) \eq 1, x \in B_r \\
            & \eta(x) \eq 0, x \in B_R^c\\
            & \abs{D\eta} \le \frac{2}{R-r}
        \end{aligned}
        \right.
    \end{equation}
    记$A(k,r)=\{u>k\} \cap B_r$. 显然, $\supp u_k \cap B_r = A(k,r)$. 由不等式\eqref{meq6}可知, 当$k>k_0$时,有
    \begin{equation}\label{ss}
        \begin{split}
            &\int_{A(k,r)} (u-k)^2 \\
            \le & C( \rec{\abs{R-r}^2} \abs{A(k,R)}^\epsilon\int_{A(k,R)}(u-k)^2 + (k+\norm{f}_{L^q})^2\abs{A(k,R)}^{1+\epsilon})
        \end{split}
    \end{equation}
    为了证明$u^+$在$B_{\rec{2}}$上的有界性,只需证明存在$k$使得 $\int_{A(k,\rec{2})} (u-k)^2=0$即可.  $\forall h > k > k_0$及$ 0 < r < 1$,显然有
    \begin{equation}
        \int_{A(h,r)} (u-h)^2 \le \int_{A(k,r)}(u-k)^2
    \end{equation}
    以及
    \begin{equation}
        \begin{split}
            \abs{A(h,r)} = &\abs{B_r \cap \{u-h>0\}} \\
            = &\abs{B_r \cap \{u-k > h-k\}} \\
            \le  & \abs{B_r \cap \{(u-k)^2 > (h-k)^2\}}\\
            \le & \rec{(h-k)^2}\int_{A(k,r)}(u-k)^2
        \end{split}
    \end{equation}
    于是, $\forall \rec{2} < r < R <1$,有
    \begin{equation} \label{s1}
        \begin{split}
            \abs{A(h,R)}^\epsilon \int_{A(h,R)}(u-h)^2 \le & (\rec{(h-k)^2}\int_{A(k,R)}(u-k)^2)^\epsilon\int_{A(k,R)} (u-k)^2\\
            = &\rec{(h-k)^{2\epsilon}} (\int_{A(k,R)}(u-k)^2)^{1+\epsilon}
        \end{split}
    \end{equation}
    \begin{equation} \label{s2}
        \abs{A(h,R)}^{1+\epsilon} \le (\rec{(h-k)^2})\int_{A(k,R)}(u-k)^2)^{1+\epsilon}
    \end{equation}
    将不等式$\eqref{s1},\eqref{s2}$代入到  \eqref{ss}中,则$\forall h > k > k_0$, 有
    \begin{equation}
        \begin{split}
            &\int_{A(h,r)}(u-h)^2\\
            \le & C(\rec{(R-r)^2}\rec{(h-k)^{2\epsilon}}(\int_{A(k,R)}(u-k)^2)^{1+\epsilon}+(k+\norm{f}_{L^q})^2 \\
            +&(\rec{(h-k)^2}\int_{A(k,R)}(u-k)^2)^{1+\epsilon}) \\
            =& C(\int_{A(k,R)}(u-k)^2)^{1+\epsilon}\rec{(h-k)^{2\epsilon}}(\rec{(R-r)^2}+(k+\norm{f}_{L^q})^2\rec{(h-k)^2})
        \end{split}
    \end{equation}
    %记 $\phi(h,r)=(\int_{B_r}u_h^2)^\rec{2}$.则
    记 $\phi(h,r)=(\int_{A(h,r)}(u-h)^2)^\rec{2}$.则
    \begin{equation}\label{msharp}
        \phi(h,r) \le C\phi(k,R)^{1+\epsilon}\rec{(h-k)^\epsilon}(\rec{R-r}+(k+\norm{f}_{L^q})\rec{h-k})
    \end{equation}
    现在,设$L$是待定常数.记
    \begin{equation}
        k_i=k_0+L(1-\rec{2^i}), r_i=\rec{2}+\rec{2^i}
    \end{equation}
    取$h-k_i,k=k_{i-1}, r=r_i,R=r_{i-1}$,代入到\eqref{msharp}中,则有
    \begin{equation}\label{mssharp}
        \begin{split}
            \phi(k_i,r_i) \le & C\phi(k_{i-1},r_{i-1})^{1+\epsilon}(\frac{2^i}{L})^\epsilon(2^i+(k_{i-1}+\norm{f}_{L^q})\frac{2^i}{L}) \\
            \le & C\phi(k_{i-1},r_{i-1})^{1+\epsilon}(2^i)^{1+\epsilon}\rec{L^{1+\epsilon}}(k_0+L+\norm{f}_{L^q})
        \end{split}
    \end{equation}
    \begin{claim*}
        当$L$足够大时,存在$A>1$使得$\phi(k_i,r_i) \le \rec{A^i}\phi(k_0,r_0)$.
    \end{claim*}
    我们用数学归纳法. $i=0$时断言是平凡的.设断言对$i-1$成立.则
    \begin{equation}
        \begin{split}
            \phi(k_i,r_i) \le & C(\rec{A^{i-1}}\phi(k_0,r_0))^{1+\epsilon}(2^i)^{1+\epsilon}\rec{L^{1+\epsilon}}(k_0+L+\norm{f}_{L^q}) \\
            =&C\rec{A^i}\phi(k_0,r_0)(\frac{2^{1+\epsilon}}{A^\epsilon})^i(A^{1+\epsilon}\frac{\phi^\epsilon(k_0,r_0)}{L^{1+\epsilon}}(k_0+L+\norm{f}_{L^q}))
        \end{split}
    \end{equation}
    为了使得归纳法成立, 取$A$使得$A^\epsilon=2^{1+\epsilon}$. 显然,$A>1$. 同时选取$L$使得
    \begin{equation}
        CA^{1+\epsilon}\frac{\phi^{\epsilon}(k_0,r_0)}{L^{1+\epsilon}}(k_0+L+\norm{f}_{L^q}) \le 1
    \end{equation}
    这里,可以取$L=C(k_0+\norm{f}_{L^q})\approx C(\norm{u^+}_{L^2}+\norm{f}_{L^q})$.于是,断言证毕.
    \par 令$i \to \infty$,则有$\phi(k_i,r_i) \to 0$.即,$\phi(k_0+L,\rec{2})=0$. 证毕.
\end{proof}
\subsection{Moser方法}
\begin{proof}[局部有界性的Moser迭代证明]
\newcommand{\um}{{\bar{u}_m}}
\newcommand{\ub}{{\bar{u}}}
    设
    \begin{equation}
        \bar{u}=u^++k(\text{其中, }k=\norm{f}_{L^q}), \s \bar{u}_m=\min\{\bar{u},k+m\}. 
    \end{equation}
    即
    \begin{equation}
        \um=\left\{
            \begin{aligned}
                & k \s u \le 0 \\
                &u+k \s 0 < u < m \\
                &k+m \s u\ge m
            \end{aligned}
        \right.
    \end{equation}
    显然,$u \le 0$或$u \ge m$时,$D\um =0$. $0<u<m$时,$D\um=Du$. $u>0$时, $D\ub=Du$. $D\um \ne 0$时, $\ub =\um $且$Du=D\um$. 
    取$\phi=\eta^2(\um^\beta\ub-k^{\beta+1})$. 显然, $\phi \in H^1_0$,  $\phi \ge 0$且$\supp\phi \subset\{u>0\}$. 则
    \begin{equation}
        \begin{split}
            D\phi= & \eta^2(\um^\beta D\ub + \beta \um^{\beta-1}\ub D\um)+2\eta(\um^\beta\ub-k^{\beta+1})D\eta \\
            =& \eta^2\um^\beta(D\ub+\beta D\um)+2\eta(\um^\beta\ub-k^{\beta+1})D\eta
        \end{split}
    \end{equation}
    取$\phi$作为测试函数,则有
    \begin{equation}
        \begin{split}
            &\int \aij D_iu D_j \phi \\
            =& \int \aij D_iu(D_j\ub +\beta D_j\um)\eta^2\um^\beta + \aij D_iu D_j \eta 2\eta(\um^\beta\ub -k^{\beta+1}) \\
            =& \int \aij D_i\ub D_j\ub \eta^2 \um^\beta+ \aij D_i\um D_j\um \beta \eta^2 \um^\beta + \aij \D_i\ub D_j\eta 2\eta(\um^\beta\ub -k^{\beta+1}) \\
            \ge & \lambda\int \abs{D\ub}^2\eta^2 \um^\beta+\lambda\beta \int \abs{D\um}^2\eta^2\um^\beta-\Lambda \int \abs{D\ub}\abs{D\eta}2\eta\um^\beta\ub\\
            \ge & \lambda \int \eta^2 \um^\beta(\abs{D\ub}+\beta\abs{D\um}^2)-2\Lambda\int(\epsilon\abs{D\ub}^2\eta^2 \um^\beta+\rec{\epsilon}\abs{D\eta}\um^\beta\ub^2) \\
            & \s\s\s (\text{取$\epsilon$使得$2\Lambda\epsilon=\rec{2}\lambda$}) \\
            \ge & \frac{\lambda}{2}\int \eta^2 \um^\beta(\abs{D\ub}^2+\beta\abs{D\um}^2)-C(\lambda,\Lambda)\int\abs{D\eta}^2\um^\beta\ub^2
        \end{split}
    \end{equation}
    另一方面,我们有
    \begin{equation}
        \begin{split}
            &-\int cu\phi + f \phi \\
            \le & \int \abs{c}\abs{u}\eta^2(\um^\beta\ub -k^{\beta+1})+\abs{f}\eta^2(\um^\beta\ub - k^{\beta+1}) \\
            \le & \int \abs{c}\eta^2 \um^\beta \ub^2 + \frac{\abs{f}}{k}\eta^2 \um^\beta \ub ^2 \\
            &\s \s (\text{记$c_0=\abs{c}+\frac{\abs{f}}{k}$, 显然,$\norm{c_0}_{L^q}\le 1+\Lambda$}) \\
            =&\int c_0\eta^2\um^\beta \ub^2
        \end{split}
    \end{equation}
    则由$\int \aij D_iuD_j\phi \le -\int cu\phi + f\phi$可知
    \begin{equation} \label{mosers}
        \int \eta^2 \um^\beta (\abs{D\ub}^2+\beta \abs{D\um}^2) \le C(\int \abs{D\eta}^2\um^\beta\ub^2+\int c_0\eta^2\um^\beta\ub^2)
    \end{equation}
\newcommand{\umb}{{\bar{u}_m^{\frac{\beta}{2}}}}
    记 $w=\ub_m^{\frac{\beta}{2}}\ub$,则
    \begin{equation}
        Dw=\umb D\ub+\frac{\beta}{2}\ub_m^{\frac{\beta}{2}-1}\ub D\um 
    \end{equation}
    \begin{equation}
        \begin{split}
            \abs{Dw}^2 \le & C(\um^\beta\abs{D\ub}^2+ \beta^2 \um^\beta \abs{D\um}^2) \\
            \le & C(1+\beta)\um^\beta(\abs{D\ub}^2+\beta\abs{D\um}^2)
        \end{split}
    \end{equation}
    再由不等式 \eqref{mosers}可知
    \begin{equation}
        \begin{split}
            \int \abs{Dw}^2\eta^2 \le  &C(1+\beta)\int \eta^2 \um^\beta(\abs{D\ub}^2+\beta\abs{D\um}^2) \\
            \le & C(1+\beta)(\int \abs{D\eta}^2w^2+c_0\eta^2w^2)
        \end{split}
    \end{equation}
    又因为 $\int \abs{D(\eta w)}^2 \le 2(\int \abs{Dw}^2\eta^2 + w^2\abs{D\eta}^2)$,则
    \begin{equation}\label{mosersharp}
        \int\abs{D(\eta w)}^2 \le C(1+\beta)(\int \abs{D\eta}^2 w^2 + c_0\eta^2 w^2)
    \end{equation}
    而由$\Holder$不等式可知
    \begin{equation}
        \begin{split}
            \int c_0\eta^2 w^2 \le & (\int \abs{c_0}^q)^\rec{q}(\int (\eta w)^{2q'})^\rec{q'} \\
            \le & (1+\Lambda)(\int (\eta w)^{2q'})^\rec{q'}
        \end{split}
    \end{equation}
    根据定理中的假设, $q > \frac{n}{2}$,则有$2< 2q' = \frac{2q}{q-1} < \frac{2n }{n-2} =2^*$. 设$\theta$满足
    \begin{equation}
        \frac{2q}{q-1}=(1-\theta)2+\theta\frac{2n}{n-2}
    \end{equation}
    即,$\theta=\frac{n-2}{2(q-1)}$. 则有
    \begin{equation}
        \begin{split}
            \int (\eta w)^{\frac{2q}{q-1}} =  & \int (\eta w)^{(1-\theta)2} (\eta w)^{\theta \frac{2n}{n-2}} \\
            \le & (\int (\eta w)^2)^{1-\theta}(\int (\eta w)^{2^*})^\theta
        \end{split}
    \end{equation}
    即有
    \begin{equation}
        \begin{split}
            {(\int (\eta w)^{\frac{2q }{q-1}})}^\frac{q-1}{q} \le & {(\int (\eta w))}^\frac{2q-n}{2q} {(\int (\eta w)^{2^*})}^\frac{n-2}{2q} \\
            \le & C(\int (\eta w)^2)^\frac{2q-n}{2q}{(\int \abs{D(\eta w)}^2)}^\frac{n }{2q} \\
            \le & C(\epsilon\int \abs{D(\eta w)}^2+ (\rec{\epsilon})^\frac{n }{2q-n}\int (\eta w)^2)
        \end{split}
    \end{equation}
    选取$\epsilon$足够小, $ \epsilon \le \frac{\epsilon_0}{1+\beta}$,代入到\eqref{mosersharp}中, 取$\alpha=1+\frac{n}{2q-n}$. 则有
    \begin{equation}
        \begin{split}
            \int \abs{D(\eta w)}^2 \le & C((1+\beta)(\int \abs{D\eta}^2w^2) + (1+\beta)^{\frac{n }{2q-n}} \int w^2 \eta^2) \\
            \le & C(1+\beta)^\alpha ( \int w^2(\abs{D\eta}^2+\eta^2))
        \end{split}
    \end{equation}
    记$\chi = \frac{2^*}{2}>1$, 由Sobolev不等式
    \begin{equation}
        (\int (\eta w)^{2\chi })^\rec{\chi} \le C \int \abs{D(\eta w)}^2 \le C(1+\beta)^\alpha\int w^2(\eta^2 +\abs{D\eta}^2)
    \end{equation}
    $\forall 0< r < R <1$, 取$B_r, B_R$上的截断函数$\eta$,则有 
    \begin{equation}
        (\int _{B_r}w^{2\chi})^\rec{\chi} \le C\frac{(1+\beta)^\alpha}{(R-r)^2} \int _{B_R}w^2
    \end{equation}
    而$w=\umb \ub$,则
    \begin{equation}
        (\int _{B_r}\ub^{2\chi}\um^{\beta \chi })^\rec{\chi} \le C \frac{(1+\beta)^\alpha}{(R-r)^2}\int_{B_R} \ub^2\um^\beta
    \end{equation}
    令$m \to \infty$,则有 
    \begin{equation}
        (\int_{B_r}\ub^{(2+\beta)\chi})^\rec{\chi} \le C \frac{(1+\beta)^\alpha}{(R-r)^2}\int_{B_R}\ub^{\beta+2}
    \end{equation}
    记$A_0=2, A_{i+1}=\chi A_i$, 令$r=r_{i+1}=1+\rec{2^{i+1}}, R=r_i=1+\rec{2^i}$.则有
    \begin{equation}
        \begin{split}
            \norm{\ub}_{L^{A_{i+1}}(B_{r_{i+1}})} \le & (C(A_i-1)^\alpha 2^{2i})^{\rec{A_i}}\norm{\ub}_{L^{A_i}(B_{r_i})} \\
            \le &(C\chi^{i\alpha}2^{2i})^\rec{A_i} \norm{\ub}_{L^{A_i}(B_{r_i})}
        \end{split}
    \end{equation}
    将上面的不等式迭代,则有
    \begin{equation}
        \begin{split}
            \norm{\ub}_{L^{A_{i+1}}(B_{r_{i+1}})}\le & \Pi_{j=1}^i(C\chi^{j\alpha}2^{2j})^\rec{\chi^j} \norm{\ub}_{L^2(B_1)} \\
            =&\Pi_{j=1}^iC^\rec{\chi^j}\chi^{\Sigma^i_{j=1}\frac{i\alpha}{\chi^i}}2^{\Sigma_{j=1}^i\frac{2j}{\chi^j}} \norm{\ub}_{L^2(B_1)}
        \end{split}
    \end{equation}
    显然,$i \to \infty$时收敛. 令$i \to \infty$即可.
\end{proof}
\begin{proof}[局部有界性$p,\theta$任意时的证明]
    前面已经完成了定理\eqref{local_boundedness}在$p=2,\theta=\rec{2}$时的证明. 一般情况可以由一个伸缩变换来得到. $\forall R \le 1$,设$u$是方程\eqref{meq}在$B_R$中的弱下解.定义$\tilde{u}(x)= u(Rx)$ . 显然, $D_i \tilde{u}(x) = RD_iu(Rx)$.根据弱解的定义 
    \begin{equation}
        \int_{B_R} \aij D_iu D_j\phi + cu\phi \le \int_{B_R}f\phi \forall \phi \in H^1_0{B_R}, \phi \ge 0
    \end{equation}
    作变量代换$x\to Rx$后,则变为
    \begin{equation}
        \int_{B_1} \aij(Rx) D_i(Rx)D_j\phi(Rx) + c(Rx)u(Rx)\phi(Rx) \le \int_{B_1}f(Rx)\phi(Rx)
    \end{equation}
    记$\tilde{a}^{ij}=\aij(Rx), \tilde{c}(x)=c(Rx), \tilde{f}(x)=f(Rx),\tilde{\phi}(x)\phi(Rx)$. 则
    \begin{equation}
        \rec{R^2}\int_{B_1}\tilde{a}^{ij}D_i\tilde{u}D_j\tilde{\phi}+\int_{B_1}\tilde{c}\tilde{u}\tilde{\phi} \le R^2\int_{B_1}\tilde{f}\tilde{\phi}
    \end{equation}
    显然,$\tilde{a}^{ij}$满足 $\lambda,\Lambda$一致椭圆条件.$\norm{R^2\tilde{c}}_{L^q} \le \Lambda$. 于是有 
    \begin{equation}
        \sup_{B_\rec{2}}\tilde{u}^+ \le C(\norm{\tilde{u}^+}_{L^2(B_1)}+R^2\norm{\tilde{f}}_{L^q(B_1)})
    \end{equation}
    即有
    \begin{equation}
        \sup_{B_\rec{2}}{u}^+ \le C(R^{-\frac{n}{2}}\norm{{u}^+}_{L^2(B_R)}+R^{2-\frac{n}{q}}\norm{\tilde{f}}_{L^q(B_R)})
    \end{equation}
    若$p \ge 2$,则由Holder不等式,
    \begin{equation} \label{local_boundedness_s}
        \sup_{B_\rec{2}}{u}^+ \le C(R^{-\frac{n}{p}}\norm{{u}^+}_{L^p(B_R)}+R^{2-\frac{n}{q}}\norm{\tilde{f}}_{L^q(B_R)})
    \end{equation}
    $\forall \theta \in (0,1)$,可设$\theta \in (\rec{4},1)$.因为$\theta \in (0,\frac{3}{4})$时,用$\sup_{B_\theta}u^+ \le \sup_{B_\rec{2}}u^+$即可. 现在,固定$y \in B_\theta$.则$B_{1-\theta}(y) \subset B_1$. 在$B_{1-\theta}(y)$上应用不等式\eqref{local_boundedness_s},则得到
    \begin{equation}
        \sup_{B_{1-\theta}}{u}^+ \le C(\rec{{(1-\theta)}}^{\frac{n}{p}}\norm{{u}^+}_{L^p(B_1)}+\norm{\tilde{f}})
    \end{equation}
    取有限个$u\in B_{\theta}$使得$\cup B_{1-\theta}(y_i)$覆盖$B_\theta$即可.  这样就完成了定理\eqref{local_boundedness}在$p \ge 2, \theta \in (0,1)$时的证明.
    显然,$p \in (0,2)$时无法使用Holder不等式.
    \begin{equation}
        \norm{u^+}^2_{L^2(B_R)} \le \norm{u^+}^{2-p}_{L^\infty(B_R)} \norm{u^+}^p_{L^p(B_R)}
    \end{equation}
    而
    \begin{equation}
        \begin{split}
            &\sup_{B_{\theta R}}u^+  \\
            \le & C(\rec{((1-\theta)R)^\frac{n}{2}} \norm{u^+}^{1-\frac{p}{2}}_{L_p(B_R)}\norm{u}_{L^p(B_R)}^\frac{p}{2}+\norm{f}_{L^q}) \\
            \le & C(\epsilon\rec{((1-\theta)R)^\frac{n}{2}} \norm{u^+}^{1-\frac{p}{2}}_{L_p(B_R)}+\rec{\epsilon}^\frac{2-p}{p}\norm{u}_{L^p(B_R)}+\norm{f}_{L^q}) 
        \end{split}
    \end{equation}
    在上面的不等式中,取$\epsilon$使得$C\epsilon\rec{((1-\theta)R)^\frac{n}{2}}=\rec{2}$. 则有
    \begin{equation}
        \sup_{B_{\theta R}}u^+ \le \rec{2}\norm{u^+}_{L^\infty(B_R)}+ C(\rec{((1-\theta)R)^\frac{n}{2}} \norm{u^+}^{1-\frac{p}{2}}_{L_p(B_R)}+\norm{f}_{L^q}) 
    \end{equation}
    定义$f(t)=\norm{u^+}_{L^\infty(B_t)}$, 则$\forall r < R <1$,有(取$\theta=\frac{r}{R})$,
    \begin{equation}
        f(r) \le \rec{2}f(R)+C\rec{(R-r)^\frac{n}{p}}\norm{u^+}_{L^p}+C\norm{f}_{L^q}
    \end{equation}
    应用引理\eqref{lb_iter}即可.
\end{proof}
\begin{lemma} \label{lb_iter}
    设$f(t) \ge 0$在$[\tau_0,\tau_1]$上有界. $\tau_0 \ge 0$. 设存在$\theta \in (0,1)$及 $\alpha >0$使得$\forall \tau_0 \le t < s \le \tau_1$, $f$满足
    \begin{equation}
        f(t) \le \theta f(s) + \frac{A}{(s-t)^\alpha}+B
    \end{equation}
    则
    \begin{equation}
        f(t) \le C(\theta,\alpha)(\frac{A}{(s-t)^\alpha}+B).
    \end{equation}
\end{lemma}
\begin{proof}
    将区间$[t,s]$等比分割. 固定$\tau \in (0,1)$,取$t_i$使得 $t_{i+1}-t_i=\tau(t_i-t_{i-1})$. 由于$\Sigma(t_{i+1}-t_i)=s-t$,则$t_{i+1}-t_i=(1-\tau)\tau^i(s-t)$.
    \begin{equation}
        \begin{split}
            f(t) \le & \theta f(t_1) + \frac{A}{(t_1-t)^\alpha}+B \\
            \le & \theta^2 f(t_2)+\frac{A }{(t_1-t)^\alpha}+B+\theta(\frac{A}{(t_2-t_1)^\alpha}+B)\cdots \\
            \le & \theta^{N+1}f(t_{N+1})+\Sigma \frac{A\theta^{i-1}}{(t_i-t_{i-1})^\alpha} + \Sigma B \theta^{i-1}\\
            \le & \theta^{N+1}f(t_{N+1})+A\rec{(s-t)^\alpha}\rec{(1-\tau)^\alpha}\frac{\frac{\theta}{\tau^\alpha}}{1-(\frac{\theta}{\tau^\alpha})^N}+B\frac{1-\theta}{1-\theta^N}
        \end{split}
    \end{equation}
    取$\tau$使得 $\frac{\theta}{\tau^\alpha} < 1$.  令$N \to \infty$即可.
\end{proof}
当非齐次项$f$满足较低的可积性条件时,不一定有$u \in L^\infty_\loc$. 但是用Moser迭代的技巧,我们仍然可以得到$u$的某些可积性信息.
\begin{theorem}
    设 $c \in L^{\frac{n }{2}}(B_1), f \in L^q(B_1), q \in [\frac{2n}{n+2},\frac{n}{2})$. $\aij$满足$(\lambda,\Lambda)$一致椭圆条件. 设$u \in H^1(B_1)$是方程\eqref{eq}
    \begin{equation}
        \aij D_iu D_j\phi + cu\phi \le \int f\phi \forall \phi \in H^1_0(\O), \phi \ge 0
    \end{equation}
    则$u^+\in L^{q'}_{\loc}. \rec{q'}=\rec{q}-\frac{2}{n}$. 且有
    \begin{equation}
        \norm{u^+}_{L^\rec{q'}-\rec{n}}(B_\rec{2}) \le C(\norm{u^+}_{L^2(B_1)}+\norm{f}_{L^q}(B_1))
    \end{equation}
    其中, $C=C(n,\lambda,\Lambda,q,\epsilon(k))$. $\epsilon(k)=(\int_{\abs{c}>k}\abs{c}^\frac{n}{2})^\frac{n}{2}$(这里指的是依赖于$k\to \infty$时, $\epsilon(k) \to 0$的速度)
\end{theorem}
\begin{proof}
\newcommand{\um}{{\bar{u}_m}}
\newcommand{\ub}{{\bar{u}}}
    设$m >0$,记 $\ub=u^+$, $\um=\min\{\ub,m\}$. 记$\phi=\eta^2 \um^\beta \ub \in H^1_0(B)$. $\supp \phi \subset \{u >0\}$. $D\um\ne 0$时,$\um=\ub$且$D\um = D\ub = Du$.
    \begin{equation}
        \begin{split}
            D\phi= &\eta^2 \beta \um^{\beta-1}\ub D\um+\eta^2\um^\beta D\ub + 2\um^\beta \ub \eta D\eta\\
            =& \eta^2 D\um^\beta(\beta D\um+D\ub)+2\um^\beta\ub \eta D\eta
        \end{split}
    \end{equation}
    则有
    \begin{equation}
        \begin{split}
            &\int \aij D_iuD_j\phi  \\
            =& \int \eta^2 \um^\beta (\beta\aij D_iuD_j\um+\aij D_iuD_j\ub)+2\um\ub\eta\aij D_iuD_j\eta \\
            \le & \int \eta^2 \um^\beta(\lambda\beta\abs{D\um}^2+\lambda\abs{D{\ub}}^2)-2\Lambda\um^\beta\ub\eta\abs{D\ub}D\eta\\
            \le & \int \eta^2 \um^\beta(\lambda\beta\abs{D\um}^2+\lambda\abs{D{\ub}}^2)-2\Lambda(\epsilon\eta^2\um^\beta \abs{D\ub}^2+\rec{\epsilon}\um^\beta\ub^2\abs{D\eta}^2)\\
            \le & C(\int \eta^2\um^\beta(\beta\abs{D\um}^2+\abs{D\ub}^2)-\um^\beta\ub^2\abs{D\eta}^2)
        \end{split}
    \end{equation}
    于是,有
    \begin{equation} \label{llllstar}
        \begin{split}
            \int \eta^2 \um^\beta(\beta\abs{\um}^2+\abs{Du}^2) \le  C(\int\um^\beta\ub^2\abs{D\eta}^2+\abs{cu\phi}+\abs{f\phi}) 
        \end{split}
    \end{equation}
    令$w=\um^{\frac{\beta}{2}}\ub$. 则$Dw=\frac{\beta}{2}\um^{\frac{\beta}{2}-1}\ub D\um+\ub^{\frac{\beta}{2}}D\ub$.
    在$u >0$上,有
    \begin{equation}
        \abs{Dw}^2 \le C(1+\beta)(\beta\um^\beta\abs{\um}^2+\ub^\beta\abs{D\ub}^2).
    \end{equation}
    由\eqref{llllstar},有
    \begin{equation}
        \int \eta^2\abs{Dw}^2 \le C(1+\beta)(\int \um^\beta\ub^2\abs{D\eta}^2+\abs{cu\phi}+\abs{f\phi})
    \end{equation}
    又因 $\abs{D(\eta w)}^2 \le 2(\eta^2\abs{Dw}^2+\abs{D\eta}^2w^2)$, $\abs{D\eta}^2w^2=\um^2\ub^2\abs{D\eta}^2$,则有
    \begin{equation}
        \int \abs{D(\eta w)}^2 \le C(1+\beta)(\int \abs{D\eta}^2w^2+\abs{cu\phi}+\abs{f\phi})
    \end{equation}
    由Soboleb不等式,
    \begin{equation}\label{llllsharp}
        (\int (\eta w)^{2\chi})^\rec{\chi} \le \int \abs{D\eta w}^2 \le C(1+\beta)(\int \abs{D\eta}^2w^2+\abs{cu\phi}+\abs{f\phi})
    \end{equation}
    而
    \begin{equation}
        \begin{split}
            \int \abs{cu\phi} = & \int \abs{c}\eta^2\um^\beta \ub^2 \\
            \le & k \int_{\abs{c}\le k}\eta^2 \um^\beta \ub^2 + \int_{\abs{c}>k}\abs{c}\eta^2\um\ub^2 \\
            \le& k \int_{\abs{c}\le k} \eta^2 \um^\beta \ub^2+(\int_{\abs{c}>k}\abs{c}^\frac{n}{2})^\frac{2}{n}(\int (\eta^2\um\ub^2)^\chi)^\rec{\chi} \\
            =& k \int (\eta w)^2 + \epsilon(k)(\int (\eta w)^{2\chi})^\rec{\chi}
        \end{split}
    \end{equation}
    而$\epsilon(k) \to 0$. 因此当$k$足够大时,由不等式\eqref{llllsharp}可知
    \begin{equation}\label{llllplus}
        (\int (\eta w)^{2\chi})^\rec{\chi} \le C(1+\beta)(\int w^2(\eta^2+\abs{D\eta}^2)+\abs{f\phi})
    \end{equation}
    而$f\phi=f\eta\um^\beta\ub$. 现在其它项都是$\um^\beta u^2$的形式,我们需要比较$\um^\beta \um$与$\um^\beta\ub^2$这两项.显然有
    \begin{equation}
        \um^\beta \ub \le \um^{\beta-\epsilon}\ub^{1+\epsilon}
    \end{equation}
    取$\epsilon$使得$\frac{\beta-\epsilon}{1+\epsilon}=\frac{\beta}{2}$,即$\epsilon=\frac{\beta}{\beta+2}$. 则有
    \begin{equation}
        \um^\beta\ub \le (\um^\beta \ub^2)^{\rec{\beta}(\beta-\epsilon)}=(\um^\beta \ub^2)^\frac{\beta+1}{\beta+2}
    \end{equation}
    则由Holder不等式
    \begin{equation}
        \begin{split}
            \int \abs{f}\eta^2 \um^\beta \ub \le & \int \abs{f}(\eta^2 \um^\beta \ub^2)^\frac{\beta+1}{\beta+2}(\eta^2)^\rec{\beta+2} \\
            \le & (\int\abs{f}^q)^\rec{q}(\int (\eta^2 \um^\beta \ub^2)\chi)^{\rec{\chi}\frac{\beta+1}{\beta+2}} \abs{\supp\eta}^{1-\rec{q}-\frac{\beta+1}{(\beta+2)\chi}} \\
            & \s \s (\text{如果有}1-\rec{q}-\frac{\beta+1}{(\beta+2)\chi}\ge 0, \text{即}\beta+2 \le \frac{q(n-2)}{n-2q}) \\
            \le &\epsilon (\int (\eta^2 \um^\beta \ub^2)^\chi)^\rec{\chi}+C(\epsilon\beta)\norm{f}^{\beta+2}_{L^q}
        \end{split}
    \end{equation}
    上面的不等式要求$\abs{\supp\eta}^{1-\rec{q}-\frac{\beta+1}{(\beta+2)\chi}}$有界,但由于$\abs{\supp\eta}$很小,因此$\beta+2 \le \frac{q(n-2)}{n-2q}$的条件不能少.(后面总是假设$\beta$满足这个条件). 现在,由不等式\eqref{llllplus},有
    \begin{equation}
        (\int (\eta w)^{2\chi})^\rec{\chi} \le C(\int w^2(\eta^2+\abs{D\eta}^2)+\norm{f}^{\beta+2}_{L^q})
    \end{equation}
    记$\chi=\beta+2$, 则
    \begin{equation}
        2 \le \lambda \le \frac{q(n-2)}{n-2q} = \frac{q'}{\chi}
    \end{equation}
    选取$B_r,B_R$上的截断函数$\eta$,则
    \begin{equation}
        (\int_{B_r}w^{2\chi})^\rec{\chi} \le C(\rec{(R-r)^2}\int_{B_R}w^2+\norm{f}_{L^q}^\lambda)
    \end{equation}
    而$w=\um^\frac{\beta}{2}\ub$,令$m \to \infty$,则有 
    \begin{equation} \label{llllx}
        (\int_{B_r}\ub^{\chi\lambda})\rec{\chi\lambda} \le C(\rec{(R-r)^\frac{2}{\lambda}}(\int_{B_R}\ub^\lambda)^\rec{\lambda}+\norm{f}_{L^q})
    \end{equation}
    由于$q \in [\frac{2n}{n+2},\frac{n}{2})$,存在$k$使得$2\chi^{k-1}\le \frac{q(n-2)}{n-2q} < 2\chi^k$. 从$\beta=0$,即$\lambda=2$开始,经过$k$次迭代后,则有
    \begin{equation}
        \norm{\ub}_{L^{2\chi^k}(B_{\frac{3}{4}})}\le C(\norm{\ub}_{L^2}+\norm{f}_{L^q})
    \end{equation}
    而$\frac{q'}{\chi} < 2\chi^k$,则
    \begin{equation}
        \norm{\ub}_{L^\frac{q'}{\chi}(B_\frac{3}{4})}\le C(\norm{\ub}_{L^2}+\norm{f}_{L^q})
    \end{equation}
    再由不等式\eqref{llllx},代入$\lambda=q',R=\frac{3}{4}, r=\rec{2}$,则有
    \begin{equation}
        \norm{\ub}_{L^{q'}(B_\rec{2})} \le C(\norm{\ub}_{L^{\frac{q'}{\chi}}}+\norm{f}_{L^q}) \le C(\norm{\ub}_{L^2}+\norm{f}_{L^q})
    \end{equation}
\end{proof}
\section{\texorpdfstring{\Holder}{Holder}连续性}
$c=f=0$的情况在定理\eqref{hholder}中已经证明过.现在利用齐次方程的结论来证明方程\eqref{eq}弱解的Holder连续性.
\begin{theorem}\label{holder}
    设$\aij$满足一致椭圆条件. $c \in L^n(B_1), f\in L^q(B_1)(q > \frac{n}{2})$. 设$u$是方程\eqref{eq}的弱解
    \begin{equation}
        \int \aij D_iu D_j\phi + cu\phi = \int f \phi \forall  H^1_0(B_1)
    \end{equation}
    则存在$\alpha \in (0,1)$使得$u \in C^\alpha(B_1)$. $\alpha=(n,q,\lambda,\Lambda,\norm{c}_{L^n}$. 且存在 $R_0=R_0(q,\lambda,\Lambda,\norm{c}_{L^n})$使得$\forall  x\in B_\rec{2}, r \le R_0$,成立 
    \begin{equation}
        \int_{B_r(x)}\abs{Du}^2 \le Cr^{n-2+2\alpha}(\norm{f}^2_{L^q}+\norm{f}^2_{H^1})
    \end{equation}
    其中,$C=C(n,\lambda,\Lambda,\norm{c}_{L^n})$.
\end{theorem}
\begin{lemma} \lemma{hccc}
    设$u \in H^1(B_R)$是方程$D_i(\aij D_j u)=0$的弱解.则存在$\alpha \in (0,1)$使得$\forall r < R$, 
    \begin{equation}
        \int_{B_r}\abs{Du}^2 \le C{(\frac{r}{R})}^{n-2+2\alpha}\int_{B_R}\abs{Du}^2
    \end{equation}
\end{lemma}
\begin{proof}
    不失一般性,可设$r \in (0,\rec{4}R)$. 否则结论是平凡的. 另外,显然$u-u_{B_R}$也是方程$D_i(\aij D_ju)=0$的弱解,因此,可设$\int_{B_R}u=0$. 由齐次方程弱解的正则性,
    \begin{equation}
        \norm{u}_{C^\alpha(B_\rec{2})} \le C(n,\lambda,\Lambda) \norm{u}_{L^2(B_1)}
    \end{equation}
    作变量代替$x \to Rx$,则有 $\forall x \in B_{\frac{R}{2}}$,
    \begin{equation}
        \abs{u(x)-u(0)}^2 \le C\rec{R^{n+2\alpha}}\abs{x}^{2\alpha}\norm{u}_{L^2(B_R)} \le C\rec{R^{n+2\alpha-2}}\abs{x}^{2\alpha} \norm{Du}^2_{L^2(B_R)}
    \end{equation}
    现在,$\forall r < \rec{4}R$,取$B_r, B_{2r}$上的截断函数$\eta \in C^\infty_0$,取$\phi=\eta^2(u-u(0))$则 
    \begin{equation}
        \int \aij D_iu(2\eta(u-u(0))D\eta-\eta^2 Du)=0
    \end{equation}
    \begin{equation}
        \int_{B_2r}\abs{Du}^2\eta^2 \le\int_{B_2r}\abs{u-u(0)}^2\abs{D\eta}^2
    \end{equation}
    \begin{equation}
        \begin{split}
            \int_{B_r}\abs{Du}^2 \le & C\rec{r^2} \rec{R^{n-2+2\alpha}} \sup_{B_{2r}}\abs{u(x)-u(0)}\abs{B_{2r}}\norm{Du}_{L^2(B_R)} \\
            =&C\frac{r}{R}^{n-2+2\alpha}\int_{B_R}\abs{Du}^2
        \end{split}
    \end{equation}
\end{proof}
\begin{proof}[定理\eqref{holder}的证明]
    设$w$是方程
    \begin{equation}
        \left\{
            \begin{aligned}
                & D_j(\aij D_iw)=0  \s x \in B_R&
                & w=u \s x \in \P B_R
            \end{aligned}
        \right.
    \end{equation}
    在$B_R \subset B_1$上的弱解.  由引理\eqref{hccc}得
    \begin{equation}\label{hhhstar}
        \begin{split}
            \int_{B_r}\abs{Du}^2 \le& C(\int_{B_r}\abs{Du-Dw}2+\int_{B_r}\abs{Dw}^2 \\
            \le &(\int_{B_R}\abs{Du-Dw}^2+(\frac{r}{R})^{n-2+2\alpha'}\int_{B_R}\abs{Dw}^2 \\
            \le & C(\int_{B_R}\abs{Du-Dw}+(\frac{r}{R})^{n-2+2\alpha'}(\int_{B_R}\abs{Du}^2+\abs{D(u-w)}^2))\\
            \le & C(\int_{B_R}\abs{Du-Dw}^2+(\frac{r}{R})^{n-2+2\alpha'}\int_{B_R}\abs{Du}^2)
        \end{split}
    \end{equation}
    根据$u,w$的定义,有
    \begin{equation}
        \int \aij D_i(u-w)D_j\phi + cu\phi = \int f\phi
    \end{equation}
    记$v=u-w \in H^1_0(B_R)$,并取$v$作为测试函数,则
    \begin{equation} \label{hsharp}
        \int \abs{Dv}^2 \int \abs{cuv}+\abs{fv}
    \end{equation}
    而由Holder不等式及 由Sobolev不等式,
    \begin{equation}
        \begin{split}
            \int cuv \le & \norm{c}_{L^n} \norm{v}_{L^{2^*}} \norm{{u}}_{L^2} \\
            \le & C\norm{c}_{L^n}\norm{Dv}_{L^2}\norm{u}_{L^2}
        \end{split}
    \end{equation}
    \begin{equation}
        \begin{split}
            \int fv \le & \norm{f}_{L^{\frac{n+2}{2n}}}\norm{v}_{L^{2^*}} \\
            \le & CR^{\frac{n+2}{2}-\frac{n}{q}}\norm{f}_{L^q}\norm{Dv}_{L^2}
        \end{split}
    \end{equation}
    因此,不等式\eqref{hsharp}变为
    \begin{equation}
        \int\abs{Dv}^2 \le C(\norm{c}_{L^n}^2\norm{u}_{L^2(B_R)}^2+\norm{f}^2_{L^q}R^{n+2-\frac{2n}{q}})
    \end{equation}
    由不等式\eqref{hhhstar}可知(记$\alpha''=2-\frac{n}{q}>0$)
    \begin{equation} \label{hstarp}
        \begin{split}
            &\int_{B_r}\abs{Du}^2  \\
            \le & C((\frac{r}{R})^{n-2+2\alpha'}\int_{B_R}\abs{Du}^2 +\norm{c}^2_{L^n}\norm{u}^2_{L^2(B_R)}+\norm{f}^2_{L^q}R^{n-2+2\alpha''})
        \end{split}
    \end{equation}
    由于$R <1$, 我们可以缩小 $\alpha''$使得$\alpha'' < \alpha'$. 而 $\int_{B_R}\abs{Du}^2 \le \norm{u}^2_{H^1}$. 由引理xxx
    \begin{equation}
        \int_{B_R}u^2 \le C\norm{u}^2_{H^1}R^2
    \end{equation}
    将上面的结果代回到\label{hstarp}中,则有 
    \begin{equation} 
        \begin{split}
            &\int_{B_r}\abs{Du}^2  \\
            \le & C((\frac{r}{R})^{n-2+2\alpha'}\int_{B_R}\abs{Du}^2 +\norm{u}^2_{H^1}R^2+\norm{f}^2_{L^q}R^{n-2+2\alpha''})
        \end{split}
    \end{equation}
    由迭代引理(3.4)可知
    \begin{equation} 
        \begin{split}
            &\int_{B_r}\abs{Du}^2  \\
            \le & C((\frac{r}{R})^{n-2+2\alpha}\int_{B_R}\abs{Du}^2 +(\norm{u}^2_{H^1}+\norm{f}^2_{L^q})r^2)
        \end{split}
    \end{equation}
    再由引理3.3,
    \begin{equation}
        \int_{B_r} \le C(\norm{u}^2_{H^1}+\norm{f}^2_{L^q})r^4\cdots
    \end{equation}
    再代回到\eqref{hstarp}中,由引理3.4知$\cdots$. 一直进行下去,直到应用完引理 3.3后,得到的是恰好比$n-2+2\alpha''$小的偶数(这个数或者是$n-2$,或者是$n-3$). 无论如何,再次应用一次引理3.4,引理3.3以后,都有
    \begin{equation}
        \int_{B_r}u^2 \le C(\norm{u}^2_{H^1}+\norm{f}^2_{L^q})r^{n-1}
    \end{equation}
    代入到\eqref{hstarp}中,则当$R$足够小时,
    \begin{equation}
        \int_{B_r}\abs{Du}^2 \le C((\frac{r}{R})^{n-2+2\alpha'}\int_{B_R}\abs{Du}^2 + (\norm{u}^2_{H^1}+\norm{f}^2_{L^q}R^{n-1}))
    \end{equation}
    显然,这意叶着存在$\alpha < \alpha'$使得
    \begin{equation}
        \int_{B_r}\abs{Du}^2 \le C((\frac{r}{R})^{n-2+2\alpha'}\int_{B_R}\abs{Du}^2 + (\norm{u}^2_{H^1}+\norm{f}^2_{L^q}R^{n-2+2\alpha}))
    \end{equation}
    由引理3.4可知当$r$足够小时,
    \begin{equation}
        \int_{B_r}\abs{Du}^2 \le C(\norm{u}^2_{H^1} + \norm{f}^2_{L^q})r^{n-2+2\alpha'}
    \end{equation}
    由此可知,$u \in C^\alpha$.
\end{proof}
\section{Harnack不等式}
\begin{theorem}[弱Harnack不等式] \label{weak_harnack}
    设$u \in H^1(B_1)$是方程\eqref{eq}的非负上解,即有
    \begin{equation}
        \int \aij D_iuD_j\phi \ge \int f \phi \forall \phi \in H^1_0, \phi \ge 0
    \end{equation}
    设$f \in L^q, q > \frac{n}{2}$,则$0<p < \frac{n}{n-2}$及$0 < r < R <1$,有
    \begin{equation}
        \inf_{B_r}u + \norm{f}_{L^q(B_1)} \ge C(\int_{B_R}\abs{u}^p)^\rec{p}
    \end{equation}
    其中, $C=C(n,p,q,\lambda,r,R)$.
\end{theorem}
\begin{proof}
\newcommand{\ub}{{\bar{u}}}
    首先证明存在$p_0$使得定理中的结论成立. 记$\ub=u+k,(k=\norm{f}_{L^q})$.选取$\ub^{-2}\eta$为测试函数($\eta \in H^1_0,\eta \ge 0$,上解的Moser迭代通常选取$u^\beta \eta, \beta <0$为测试函数).
    \begin{equation}
        D(\ub^{-2}\eta)=\ub^{-2}D\eta-2\eta \ub^{-3}D\ub
    \end{equation}
    则有
    \begin{equation}
        \int \aij D_iu \ub D_j\eta \ub^{-2}-\aij D_i\ub D_j\ub 2 \eta \ub^{-3} -f\ub^{-2}\eta \ge 0
    \end{equation}
    记$u=\ub^{-1}, \tilde{f}=f\ub^{-1}$,则$Dv=-\ub^{-2}D\ub$. 于是有
    \begin{equation}
        \int \aij D_iv D_j\eta + \tilde{f}v\eta \le 0
    \end{equation}
    即,$v$是齐次方程$D_j(\aij D_iv)+\tilde{f}v=0$的下解,且$\norm{\tilde{f}}_{L^q} \le 1$. 则有弱下解的局部有界性(定理\eqref{local_boundedness}),有
    \begin{equation}
        \sup_{B_r}v \le C\rec{(R-r)^\frac{p}{n}}\norm{v}_{L^p(B_R)}, C=C(n,\lambda,\Lambda,p,q)
    \end{equation}
    则有
    \begin{equation}
        \sup_{B_r}\ub^{-p} \le C\int_{B_R}\ub^{-p}, C=C(n,\lambda,\Lambda,p,q,r,R)
    \end{equation}
    如果存在$p_0$使得$\forall B\subset B_R$,成立
    \begin{equation}\label{tobeproved}
        \int_B \ub^{-p_0}\int_B \ub^{p_0} \le C
    \end{equation}
    那么就有
    \begin{equation}
        \sup_{B_r} \ub^{-p_0} \le C \int_{B_R}\ub^{-p_0} \le C\rec{\int_{B_R}\ub^{p_0}}
    \end{equation}
    即有,
    \begin{equation}
        \int_{B_r} \ub \ge C(\int_{B_R}\ub^{p_0})^\rec{p_0}
    \end{equation}
    现在,我们需要证明\eqref{tobeproved}. 而
    \begin{equation}
        \begin{split}
            &\int_B{\ub^{p_0}}\int_B\ub^{-p_0}  \\
            = & \int_B\exp(p_0(\log\ub-(\log\ub)_B))\exp(p_0(\log\ub)_B)\int_B\exp(-p_0)(-p_0\log\ub) \\
            =& \int_B \exp(p_0(\log\ub -(\log \ub )_B))\int_B \exp(-p_0(\log\ub-(\log\ub)_B)) \\
            \le & (\int_B \exp\abs{p_0(\log\ub-(\log\ub)_T)})^2
        \end{split}
    \end{equation}
    记$w=\log\ub-(\log\ub)_{B_r}$. 现在证明 $\int_{B_r}\exp(p_0\abs{w}) \le C(n,q,\lambda,\Lambda,r)$. 取$\phi=\ub^{-q}\eta$作为测试函数. $D\eta=\ub^{-1}D\eta-\ub^{-2}\eta D\ub$,则有
    \begin{equation}
        \aij D_i D_j\eta \ub^{-1}-\aij D_iuD_j\ub \ub^{-2}\eta \ge \int f\eta \ub^{-1}
    \end{equation}
    而 $Dw=\ub^{-1}D\ub$.记$\tilde{f}=f\ub^{-1}$,则有
    \begin{equation}
        \int \aij D_iwD_jw \eta \le \int \aij D_iw D_j\eta + \int \tilde{f}\eta
    \end{equation}
    用$\eta^2$替换$\eta$,则有
    \begin{equation}
        \begin{split}
            \int \abs{Dw}^2\eta^2 \le &C(\int \abs{Dw}\abs{D\eta}\eta+ \int \tilde{f}\eta^2) \\
            \le & C(\int \epsilon\abs{Dw}^2\eta^2 +\rec{\epsilon}\abs{D\eta}^2+\int \tilde{f}\eta^2)
        \end{split}
    \end{equation}
    又因为
    \begin{equation}
        \int \tilde{f}\eta^2 \le (\int \abs{\tilde{f}}^\frac{{n}}{2})\frac{2}{n}(\int \eta^{2^*})^\frac{2}{2^*} \le C\norm{\tilde{f}}_{L^q}\norm{D\eta}^2_{L^2}
    \end{equation}
    则有 
    \begin{equation}
        \int \abs{Dw}^2\eta^2 \le C\int \abs{D\eta}^2 \forall \eta \in C^\infty_0, \eta \ge 0
    \end{equation}
    $\forall B_{2r} \subset B_1$,选取$B_{r},B_{2r}$上的截断函数$\eta$,则有
    \begin{equation}
        \begin{split}
            \rec{\abs{B_r}} \int_{B_r}\abs{w-w_{B_r}} \le & \rec{\abs{B_r}^\rec{2}}(\int_{B_r}\abs{w-w_{B_r}})^\rec{2} \\
            \le & Cr^{-\frac{n}{2}}r(\int_{B_r}\abs{Dw}^2)^\rec{2} \\
            \le C
        \end{split}
    \end{equation}
    因此,$w \in BMO$. 由John-Nirenberg定理,存在$p_0$使得
    \begin{equation}
        \int_B \exp(p_0\abs{w}) \le C
    \end{equation}
    因此,存在$p_0$使得弱Harnack不等式成立.对于$p < p_0$,显然用Holder不等式即可.
    \begin{equation}
        \int_{B_r}\ub \ge C(\int_{B_R}\ub^{p_0})^\rec{p_0} \ge C(\int_{B_R}\ub^p)^\rec{p}
    \end{equation}
    而对于$p>p_0$,我们需要反向的Holder不等式,在这正是Moser迭代所能提供的. 现在选取$\ub^{-\beta}\eta^2$作为测试函数,$\beta \in (0,1)$.
    \begin{equation}
        D\eta = -\beta\ub^{-(\beta+1)}\eta^2D\ub+2\ub^{-\beta}\eta D\eta
    \end{equation}
    \begin{equation}
        \int \aij D_iu D_j\eta 2\ub^{-\beta}\eta - \aij D_iuD_j\ub \beta \ub^{-(\beta+1)}\eta^2 \ge \int f\ub^{-\beta}\eta^2
    \end{equation}
    则有
    \begin{equation}
        \begin{split}
            &\beta \int \abs{D\ub}^2\ub^{-(\beta+1)}\eta^2 \\
            \le & C(\epsilon\int\abs{D\ub}^2\ub^{-(\beta+1)}+\rec{\epsilon}\int\ub^{1-\beta}\abs{D\eta}^2+\int \frac{\abs{f }}{k}\ub^{1-\beta}\eta^2)
        \end{split}
    \end{equation}
    取$\epsilon$使得$C\epsilon=\rec{2}\beta$.记 $\tilde{f}=\frac{f }{k}$,则有
    \begin{equation}
        \int \abs{D\ub}^2\ub^{-(\beta+1)}\eta^2 \le C(\rec{\beta^2}\int \ub^{1-\beta}\abs{D\eta}^2 +\rec{\beta}\int\tilde{f}\ub^{1-\beta}\eta^2)
    \end{equation}
    记$w=\ub^{\frac{1-\beta }{2}}$,则$\abs{Dw}^2 =\frac{(1-\beta)^2}{2}\ub^{-(\beta+1)}\abs{D\ub}^2$.于是有
    \begin{equation} \label{wsharp}
        \rec{(1-\beta)^2}\int \abs{Dw}^2\eta^2 \le C(\rec{\beta}^2\int w^2 \abs{D\eta}^2 +\rec{\beta}\int \tilde{f}w^2\eta^2)
    \end{equation}
    而
    \begin{equation}
        \int \tilde{f}\eta^2 w^2 \le (\int \tilde{f}^q)^\rec{q} (\int (\eta w)^{2q'})^\rec{q'}(q'=\frac{q}{q-1})
    \end{equation}
    而由于$q> \frac{n}{2}$,则$2< 2q' < 2^* = \frac{2n}{n-2}$. 类似于定理\eqref{local_boundedness},有
    \begin{equation}
        \begin{split}
            (\int (\eta w)^{2q'})^\rec{q} \le & \epsilon\norm{\eta w}_{L^{2^*}}+C(\rec{\epsilon})^\frac{n}{2q-n}\norm{\eta w}_{L^2} \\
            \le & \epsilon\norm{D(\eta w)}_{L^2}+C(\rec{\epsilon})^\frac{n }{2q-n}\norm{\eta w}_{L^2}
        \end{split}
    \end{equation}
    代入到\eqref{wsharp}中,则存在$\alpha >0$使得
    \begin{equation}
        \int \abs{Dw}^2\eta^2 \le C \rec{\beta^\alpha}(\int w^2\abs{D\eta}^2+\int \eta^2 w^2)
    \end{equation}
    进而有
    \begin{equation}
        \int \abs{D(\eta w)}^2 \le C\rec{\beta^\alpha}(\int w^2(\eta^2+\abs{D\eta}^2))
    \end{equation}
    记$\chi=\frac{2^*}{2}$,由Sobolev不等式,
    \begin{equation}
        (\int(\eta w)^{2\chi})^\rec{\chi} \le C \rec{\beta^\alpha}\int w^2(\eta^2 \abs{D\eta}^2)
    \end{equation}
    选取$B_r,B_R$上的截断函数$\eta$,则有
    \begin{equation}
        (\int_{B_r}w^{2\chi})\rec{\chi} \le C\rec{\beta^\alpha}\rec{(R-r)^2}\int_{B_R}w^2
    \end{equation}
    将$w=\ub^{\frac{1-\beta}{2}}$入,记$\lambda=1-\beta$,则有
    \begin{equation}
        (\int_{B_r}\ub^{\chi \lambda})^\rec{\lambda\chi} \le C\rec{(1-\lambda)^\frac{\alpha}{\lambda}}\rec{(R-r)^\frac{2}{\lambda}}(\int_{B_R}\ub^\lambda)^\rec{\lambda}
    \end{equation}
    需要注意的是,这里不等式成立的条件是$\lambda=1-\beta \in (0,1)$,因此不能无限迭代.对于任意 $p_0< p < \frac{n}{n-2}$,选取$\beta \le p_0$.则经过有限次迭代后,有 
    \begin{equation}
        \begin{split}
            (\int_{B_r}\ub^p)^\rec{p} \le & C(\int_{B_r}\ub^{N\chi})^\rec{\chi^N\lambda}  \\
            \le & C(n,p,q,\lambda,r,R)(\int_{B_R}\ub^\beta)\rec{\beta} \\
            \le & C(\int_{B_R}\ub^{p_0})^\rec{p_0}
        \end{split}
    \end{equation}
\end{proof}
\begin{theorem}[Harnack不等式]
    设$u \in H^1(\O)$是方程\eqref{meq}的非负弱解,即有
    \begin{equation}
        \int \aij D_iuD_j\phi=\int f\phi \forall \phi \in H^1_0(\O)
    \end{equation}
    其中, $f \in L^q(\O), q > \frac{n}{2}$. 则对于任意满足$B_{2R}\subset \O$的$B_R$,成立
    \begin{equation}
        \max_{\frac{R}{2}}u \le C(\min_{B_{\frac{R }{2}}}u+R^{2-\frac{n}{q}}\norm{f}_{L^q(B_R)})
    \end{equation}
\end{theorem}
\begin{proof}
    应用局部有界性和弱Harnack不等式即可.
\end{proof}
\begin{corollary}
    设$u \in H^1(\O)$是方程\eqref{meq}的弱解.设$f \in L^q, q>\frac{n}{2}$.
    \begin{equation}
        \int\aij D_iuD_j\phi = \int f\phi \forall \phi \in H^1_0(\O)
    \end{equation}
    则存在$\alpha \in (0,1)$使得$u \in C^\alpha$,且 $\forall B_R\subset \O$及 $x,y \in B_{\frac{R}{2}}$,有 
    \begin{equation}
        \abs{u(x)-u(y)} \le C\abs{\frac{x-y}{R}}^\alpha(\rec{R^{\frac{n}{2}}}\norm{u}_{L^2(B_R)}+R^{n-\frac{q}{2}}_{L^q{B_R}})
    \end{equation}
    其中,$C=C(n,\lambda,\Lambda,q)$.
\end{corollary}
\begin{proof}
    只需证明$R=1$时即可,一般情况下做一个伸缩变换即可. 记$M_r=\sup_{B_r}u, m_r=\inf_{B_r}u, w(r)=M_r-m_r=\osc_{B_r}u$. 显然,只需证明$\forall r \le \rec{2}, w(r) \le Cr^\alpha(\norm{u}_{L^2}(B_1)+\norm{f}_{L^q}(B_1))$即可. 将Harnack不等式分别应用在$M_r-u \ge 0$及$u-m_r \ge 0$上,则有
    \begin{align}
        &M_r-m_{\frac{r}{2}} \le C(M_r-M_{\frac{r}{2}}+r^{2-\frac{n}{q}\norm{f}_{L^q(B_r)}}) \\
        &M_{\frac{r}{2}} - m_r \le C(m_{\frac{r}{2}}-m_r+r^{2-\frac{n}{q}\norm{f}_{L^q(B_r)}})
    \end{align}
    两式相加,则有
    \begin{equation}
        w(r)+w(\frac{r}{2}) \le C(w(r)-w(\frac{r}{2})+r^{2-\frac{n}{q}}\norm{f}_{L^q(B_r)})
    \end{equation}
    则有$w(\frac{r}{2}) \le \lambda w(r)+cr^{2-\frac{n}{q}}\norm{f}_{L^q(B_r)} \s \lambda \in (0,1)$. 在引理\eqref{lemma419}中,选择$\tau=\rec{2}, \sigma(r)=Cr^{r-\frac{n}{q}},R=\rec{2}$, 选择$\mu$使得$\alpha=(1-\mu)\frac{\ln \lambda}{\lambda 2} < \mu(2-\frac{q}{n})$,则有
    \begin{equation}
        w(r)\le Cr^\alpha(w(\rec{2})+\norm{f}_{L^q(B_1)})
    \end{equation}
    在由弱解的局部有界性
    \begin{equation}
        w(\rec{2}) \le \norm{u}_{L^2}+\norm{f}_{L^q}
    \end{equation}
    证毕.
\end{proof}
\begin{lemma}\label{lemma419}
    设$w(r), \sigma(r)$是定义在$[0,R]$上的非减函数.且存在$\tau,\lambda <1$使得
    \begin{equation}
        w(\tau r) \lambda \lambda w(r)+\sigma(r)
    \end{equation}
    则$\forall \mu \in (0,1)$,存在 $C=C(\lambda,\tau)$及$\alpha=\alpha(\lambda,\tau,\mu)$使得
    \begin{equation}
        w(r)\le C((\frac{r}{R})^\alpha w(R)+\sigma(r^\mu R^{1-\mu}))
    \end{equation}
\end{lemma}
\begin{proof}
    因定$r_1 < R$. $\forall r < r_1$,由于$\sigma$非减,则
    \begin{equation}
        w(\tau r) \le \lambda w(r)+\sigma(r_1)
    \end{equation}
    迭代上面的少地式,则有
    \begin{equation}
        \begin{split}
            w(\tau^k r) \le& \lambda w(\tau^{k-1}r) +\sigma(r_1) \\
            \le & \lambda^2 w(\tau^{k-2}r)+\lambda\sigma(r_1)+\sigma(r_1)\\
            &\s\cdots\\
            \le & \lambda^kw(r)+\frac{1-\lambda^k}{1-\lambda}\sigma(r_1)\\
            \le &\lambda^kw(R)+\rec{1-\lambda}\sigma(r_1)
        \end{split}
    \end{equation}
    $\forall r \le r_1$,选择$k$使得$\tau^{k+1}r_1< r \le \tau^kr_1$,则有
    \begin{equation}
        w(r) \le w(\tau^k r) \le \lambda^kw(R)+\rec{1-\lambda}\sigma(r_1)
    \end{equation}
    而$k \le \frac{\ln(\frac{r}{r_1})}{\ln \tau} < k+1$,则 $\lambda^k \le \lambda ^{\frac{\ln \frac{r}{r_1}}{\ln \tau}}$. 于是有
    \begin{equation}
        w(r) \le \lambda^{\frac{\ln \frac{r}{r_1}}{\ln \tau}} +\rec{1-\lambda}\sigma(r_1)
    \end{equation}
    取$r_1=r^\mu R^{1-\mu}$,则
    \begin{equation}
        \lambda ^\frac{\ln \frac{r}{r_1}}{\ln \tau}=\lambda^(\ln(\frac{r}{R})^{1-\mu})^\rec{\ln \tau}=(\frac{r}{R})^{(1-\mu)\frac{\ln \lambda}{\tau}}
    \end{equation}
    \begin{equation}
        w(r) \le (\frac{r}{R})^{(1-\mu)\frac{\ln \lambda}{\ln \tau}}w(R)+\rec{1-\lambda}\sigma(r^\mu R^{1-\mu})
    \end{equation}
\end{proof}
\begin{corollary}[Liouville定理]
    设$u$是方程$\int_{\R^n}\aij D_iuD_j\phi=0$在$\R^n$中的弱解.$\aij$满足一致椭圆条件.若$u$有界,则$u$是常数.
\end{corollary}
\begin{proof}
    记$w(r)=\sup_{B_r}u-\inf_{B_r}u$. 则存在$\lambda \in (0,1)$使得$w(r) \le \lambda w(2r)$.于是$w(r) \le \lambda^kw(2^kr)$. 令$k \to \infty$即可.
\end{proof}
\chapter{Plateau问题}
\newcommand{\FG}{{\mathcal{F}_\gamma}}
\newcommand{\AG}{{\mathbb{A}_\gamma}}
\renewcommand{\EG}{{\mathbb{E}_\gamma}}
\renewcommand{\E}{{E}}
\renewcommand{\D}{{\mathbb{D}}}
\newtheorem*{plateauproblem*}{Plateau问题}
本章中, 我们回到最开始的问题:
极小曲面起源于下面的问题:
\begin{plateauproblem*}\label{plateau}
    给定简单闭曲线$\gamma \subset \R^3$, 是否存在以$\gamma$为边界的面积最小的曲面? 即, 寻找曲面$\Sigma, \partial \Sigma = \gamma$, 使得对于任意满足$\partial \Sigma' = \gamma$的曲面$\Sigma'$, 都有
    \begin{equation*}
        \Area(\Sigma) \le \Area(\Sigma')
    \end{equation*}
\end{plateauproblem*}
在本章, 按照Douglas和Rado的方法, 我们将给出Plateau问题的肯定的回答. 关于Douglas在Plateau问题上的工作的简介, 可以参考\cite{WorkofDouglas}.
\par 设$\gamma \subset \R^3$是简单闭曲线. 记$\FG$是满足下列条件的函数的集合:
\begin{enumerate}
    \item $u: \overline{\D} \to \R^3, u\in C(\overline{\D})\cap W_{loc}^{1,2}(\D,\R^3)$.
    \item $u\mid_{\P \D}$是到$\gamma$的单调映射.
    %\item $\abs{u_x}^2=\abs{u_y}^2, \inner{u_x}{u_y}=0$.
\end{enumerate}
\begin{remark}
    称$u$是单调映射, 是指我们将$\gamma$沿任意点剪断看作实轴上的一段线段,  $u$是通常意义下的单调映射.  这里之所以没有要求$u$是同胚, 是因为同胚在求极限后(一致收敛拓扑)不会被保持,  而单调性总是会被保持的. 
\end{remark}
解决Plateau问题的第一选择是:取$u_k \in \FG$使得$\Area(u_k(\D)) \to \AG$, 并证明$\{u_k\}$存在收敛子列. 然而这种思路有两个问题需要解决:
\begin{enumerate}
    \item 我们可以改变$u_k$在任意点$p$附近的值, 得到$\tilde{u}_k$, 使得$\tilde{u}_k$与$\tilde{u}_k(\mathbb{D})$与 $u_k(\D)$相差"细长"的管状区域, 如图. 这样得到的$\tilde{u}_k$仍然是面积最小的列. 但是这样的列无法在通常的意义下收敛. 
    \item 第二个问题是面积泛函是与参数选取无关的, 这意味着选取$\{u_k\}$后, 对于任意的微分同胚$\phi_k: \D \to \D$, $\Area(u_k\circ \phi_k(\D))= \Area(u_k(\D))$. 这种情况下, 我们也很难得到收敛子列.
\end{enumerate}
为了解决上面的问题, 我们将使用能量泛函而不是面积泛函. 当然, 首先我们需要说明的是, 对能量泛函求最小值和对面积泛函求最小值这两个问题是等价的(虽然对于能量泛函, 上述两个问题仍然存在, 但是我们较好的解决该问题的方法).
\begin{theorem} \label{plateau_problem}
    设$\gamma \subset \R^3$是可求长的Jordan曲线. 则存在映射$u \in \FG$使得 $\forall v \in \FG$, 
    \begin{equation}
        \Area(u(\D))\le \Area(v(\D)).
    \end{equation}
\end{theorem}
对于任意$u \in \FG$, 其面积与能量分别定义为
\begin{align}
    &\Area(u)=\int_\D\abs{u_x\wedge u_y} \\
    &\E(u)=\frac{1}{2}\int_\D\abs{u_x}^2+\abs{u_y}^2
\end{align}
简单计算可知
\begin{equation}\label{AleE}
    \Area(u)=\int_\D \sqrt{\abs{u_x}^2\abs{u_y}^2-\inner{u_x}{u_y}^2} \le \frac{1}{2}\int_\D \abs{u_x}^2+\abs{u_y}^2 \le\E(u).
\end{equation}
并且等号成立, 当且仅当$\abs{u_x}=\abs{u_y}, \inner{u_x}{u_y}=0$.
\par 记
\begin{align}
    &\AG=\inf\{\Area(v)\mid v \in \FG\}.\\
    &\EG=\inf \{\E(v)\mid v \in \FG\}.
\end{align}
\begin{definition}
    如果$u \in W^{1,2}(\mathbb{D},\R^3)$几乎处处满足$\inner{u_x}{u_y}=0$并且$\abs{u_x} = \abs{u_y}$, 则称$u$是几乎共形的.
\end{definition}
\begin{lemma}
    $\AG=\EG$, 并且如果$u \in \FG$取到$\EG$, 则$u$是几乎共形的.
\end{lemma}
\begin{proof}
    由不等式\eqref{AleE}可知, $\AG \le \EG$.
    \par 对于反方向的, 设$u\in \FG$ 且$\E(u) \le \EG + \epsilon$. 首先设$u$是浸入, 即$du$处处非退化. 设$(\D,g)$为$u$作用下的拉回度量, 即$g=du^*dx^2$, 由等温坐标的存在性可知, 存在光滑同胚$\phi: \D \to \D$使得$\phi$是$\mathbb{D} \to (\D,g)$ 之间的共形映射, 即$d\phi ^* g=\lambda^2dx^2$. 而$u\circ \phi$是共形浸入, 则有
    \begin{equation}
        \Area(u)=\Area(u\circ \phi)=\E(u\circ \phi) \ge \E(u)-\epsilon.
    \end{equation}
    如果$du$有奇点, 那么我们定义$u^s: \D \to \R^5$, $u^s(x,y)=(u,sx,sy)\in \R^5$. 则$du^s$是非退化的. 像上面一样, 通过$u^s$拉回的度量为$du^*g_{\R^3}+s^2(dx^2+dy^2)$. 显然地, 
    \begin{equation}
        \Area(u^s)=\int_\D \det (du^*g_{\R^3}+s^2I) \ge \Area(u).
    \end{equation}
    \begin{equation}
        \E(u^s\circ\phi)=\int_\D \abs{(u^s\circ \phi)_x}^2+ \abs{(u^s\circ \phi)_y}^2 =\E(u\circ \phi)+s^2\E(\phi).
    \end{equation}
    因此, 当$s$足够小时, 我们有
    \begin{equation}
        \begin{split}
            \Area(u) \le \Area(u^s\circ \phi) &= \E(u^s\circ \phi) =\E(u\circ \phi) +s^2\E(\phi) \\
            &\le \E(u)-\frac{1}{2}\epsilon
        \end{split}
    \end{equation}
\end{proof}
\begin{lemma}
    设$f \in W^{1,2}(\D,\R^3)$, $E(f) \le K$. 设$0 < \delta < 1$, $p \in \mathbb{D}$. 则存在$\rho \in (\delta, \sqrt{\delta})$ 使得$f\mid_{\P B(p,\rho) \cap \D}$是绝对连续的, 且$\forall z_1,z_2 \in \P B(p,\rho) \cap \D$, 成立 
    \begin{equation}
        \abs{f(z_1)- f(z_2)} \le (8\pi)^{\frac{1}{2}}(\log \frac{1}{\delta})^{-\frac{1}{2}}.
    \end{equation}
\end{lemma}
\begin{proof}
    在$p$点处引理极坐标$(r,\theta)$. 由于$f \in W^{1,2}$, 则几乎对于所有的$r$, $f$关于$\theta$是绝对连续的.  则Cauchy不等式, 
    \begin{equation}
        \begin{split}
            \abs{f(z_1)-f(z_2)} &\le \int_{\P B \cap \D} \abs{f_\theta(re^{i\theta})}d\theta \\
            & \le \sqrt{2\pi} (\int_{\P B\cap \D} f^2_\theta d\theta)^{\frac{1}{2}}.
        \end{split}
    \end{equation}
    而由于在极坐标下, $Df=f_r \P_r + \frac{1}{r^2}f_\theta \P_\theta$, 则
    \begin{equation}
        \int_{B \cap \D} (f_r^2 + \frac{1}{r^2}f_\theta^2)r drd\theta \le 2K.
    \end{equation}
    取$\rho$使得$\int_{\P B \cap \D} f^2_\theta(\rho e^{i\theta})$取到(几乎)最小, 则有
    \begin{equation}
        \int^{\sqrt{\delta}}_\delta \int_{\P B \cap \D}\frac{1}{r^2}f^2_\theta(\rho e^{i\theta}) rd\theta dr \le 2K.
    \end{equation}
    因此, 
    \begin{equation}
        \abs{f(z_1)-f(z_2)} \le \sqrt{2\pi} \frac{2K}{ \int^{\sqrt{\delta}}_\delta \frac{1}{r}} = \sqrt{8K\pi}(\log \frac{1}{\delta})^{-\frac{1}{2}}.
    \end{equation}
\end{proof}
\begin{lemma} \label{boundary_compactness}
    一致连续.
\end{lemma}
\begin{proof}%[定理\eqref{plateau}的证明]
    取$u_k \in \FG$ 且$E(u_k) \to \EG$. 用与$u_k$具有相同边界的调和函数替换$u_k$, 仍然记为$u_k$, 由于调和函数是能量最小的, 我们仍然有$E(u_k) \to \EG$. 固定$q_1,q_2,q_3 \in \gamma$及 $p_1,p_2,p_3 \in \P \D$, 取$\phi_k \in Aut(\D)$使得$u_k\circ \phi_k(p_i)=q_i$. 仍将$u_k \circ \phi_k$记作$u_k$. 由于$u_k$是调和的, 则
    \begin{equation}
        \max_{\D}\abs{u_k - u_l} = \max_{\P \D} \abs{u_k-u_l}.
    \end{equation}
    而根据引理\eqref{boundary_compactness}, $u_k\mid_{\P \D}$ 包含一致收敛子列, 则$u_k$包含一致收敛子列. 设$u_k \to u$.
\end{proof}
%\chapter{存在性方法}
这一章中我们介绍解下面的Dirichlet问题的一些方法. 每一种方法都有它的特点.
\par 设$\O\subset \R^n$是一个有界区域. 考虑下面的边界值问题
\begin{equation}
    \left\{
        \begin{aligned}
            &\L u(x) = f(x) \s x \in \O \\
            &u(x)=g(x) \s x \in \P \O
        \end{aligned}
    \right.
\end{equation}
在具体的问题中,$\L$是一个给定的偏微分算子,  $g(x)$是在$\P \O$上给定的一个函数, $g(x)$和区域$\O$通常要满足一定的正则性. 比如,连续函数, Lipschitz区域或者$C^k$区域等. 
\ifdefined \noperron
\else
\section{Perron方法}
\subsection{经典解}
Perron方法是一种非常"初等"的方法,它的基础是最大值原理, 方程的局部可解性和梯度的内估计. Perron方法的一个特点是它将Dirichlet问题的的两方面分开来看, 即是一方面是函数在内部是满足方程, 另一方面是函数在边界上是否取到相应的边界值. 本节我们以Laplace方程为例, 考虑下面的Dirichlet问题.
\begin{equation} \label{perron}
    \left\{
        \begin{aligned}
            &\Delta u(x) = 0 \s x \in \O \\
            &u(x)=\phi(x) \s x \in \P \O
        \end{aligned}
    \right.
\end{equation}
\begin{definition}
    函数$u \in C(\O)$被称为下调和的,如果对于任意球$B \subsub \O$以及任意在$\P \O$上满足 $u|_{\P B} \le h|_{\P B}$的调和函数$h$,  在$B$中都有$u \le h$.
\end{definition}
上面的定义中的$h$可以由Poission积分给出.
\begin{theorem}
    设$\phi$是$\P B_r(0)$上的连续函数. 定义
    \begin{equation}
        u(x)=\left\{
            \begin{aligned}
                &\frac{r^2-\abs{x}^2}{n\omega_n r}\int_{\P B}\frac{\phi(y)}{\abs{x-y}^n}dS_y \s x \in B \\
                &\phi(x) \s x \in \P B
            \end{aligned}
        \right.
    \end{equation}
    那么,  $u\in C^2(B)\cap C(\overline{B})$并且在$B$中满足$\Delta u=0$.
\end{theorem}
通过调和函数的均值性质,还可以给出以下的等价定义.
\begin{definition}
    称$u \in C(\O)$是下调和函数,如果对于任意球$B_r(x_0) \subsub \O$, 都有
    \begin{equation}
        u(x_0) \le \rec{\abs{\P B_r(x_0)}} \int_{\P B_r(x_0)}u(x)dS
    \end{equation}
    \begin{equation}
        (\text{或者 }u(x_0) \le \rec{\abs{B_r(x_0)}} \int_{B_r(x_0)}u(x)dx)
    \end{equation}
\end{definition}
同样的方式,可以定义上调和函数.
\begin{proposition}
    设$\O$是连通区域且$u$是$\O$中的下调和函数. 如果$u$在$\O$内部取到局部最大值,那么$u$是常数.
\end{proposition}
\begin{corollary}
    设$u,v$分别是$\O$中的下调和及上调和函数. 若$u|_{\P \O} \le v|_{\P \O}$, 则$u \le v$.
\end{corollary}
\begin{definition}
    设$u$是$\O$中的下调和函数, $B \subsub \O$. 设$\bar{u}(x)$在$B$中是调和函数在$B$的边界上满足$\bar{u}|_{\P B}= u|_{\P B}$. 定义
    \begin{equation}
        U(x)=
        \left \{
            \begin{aligned}
                &\bar{u}(x) \s x \in B\\
                &u(x) \s x \in \O-B
            \end{aligned}
        \right.
    \end{equation}
    称$U$为函数$u$的调和提升.
\end{definition}
\begin{proposition}
    若$u$是下调和函数, 函数$U$是$u$的调和提升.  则$U$是下调和函数.
\end{proposition}
\begin{proof}
    设$B$为调和提升的定义中的球. 设$B' \subsub \O$且$h$是$B'$中的调和函数且$h|_{\P B'} \ge U|_{\P B'}$.  显然地, 我们只要考虑$B \cap  B' \ne \emptyset$的情况即可. 
    \par 首先注意到在$\P B'$上, $h \ge U \ge u$. 则在$B'$中, $h > u$.  因此在$B\cap B'$的两部分边界上,均有$h \ge \bar{u}$. 而$h,\bar{u}$都是 $B\cap B'$中的调和函数, 则在$B \cap B'$上,也有$h \ge \bar{u} = U$.
\end{proof}
\begin{proposition}
    若$u,v$是下调和函数, 则$\max(u,v)$是下调和函数.
\end{proposition}
现在,设$\O$是有界区域, $\phi$是$\P \O$上的一个连续函数. 定义
\begin{equation}
    S_\phi=\{ u \in C(\O)\mid u\text{下调和且 }u|_{\P \O} \le \phi\}
\end{equation}
称$S_\phi$中的函数为$\phi$-下调和函数.
\begin{theorem}
    函数$u_\phi(x)=\sup_{v\in S_\phi}v(x)$在$\O$中是调和函数.
\end{theorem}
\begin{proof}
    我们首先说明$u_\phi$是良好定义的. 记
    \begin{equation}
        m=\min_{\P \O} \phi\s  M=\max_{\P \O} \phi
    \end{equation}
    显然地, $m \in S_\phi$.  因此, $S_\phi$非空. 由于是常数函数$M$是调和函数且在$\P \O$上,  $\forall v \in S_\phi$, $v \le \phi(x) \le M$. 因此,则根据最大值原理, 在$\O$中, $v\le M$. 因此, $u_\phi \le M$. 则$u_\phi$是良好定义的.
    \par 现在, 固定$B=B_r(x_0) \subsub \O$. 我们证明在$B$中$u_\phi$是调和的. 根据定义,存在$v_k\in S_\phi$使得
    \begin{equation}
        \lim_{k \to +\infty} v_k(x_0)=u_\phi(x_0).
    \end{equation}
    设$V_k$是$v_k$相对于$B$的调和提升. 因为$V_k\in S_\phi$, 显然地, 我们有
    \begin{equation}
        v_k(x_0) \le V_k(x_0) \le u_\phi(x_0)
    \end{equation}
    于是$V_k(x_0) \to u_\phi(x_0)$.  根据调和函数的内估计,我们有
    \begin{equation}
        \sup_{B_{\frac{r}{2}}}\abs{D^\alpha u} \le C(n,r,\alpha)\sup_B u.  
    \end{equation}
    则根据Alzera-Ascoli定理, 在$B_{r}$上, $V_k$中包含内闭一致收敛子列.  因此, 我们可以假设存在函数$\tilde{v}(x)$使得, 在$B_r(x_0)$中, $V_k\tto{\text{一致收敛}}\tilde{v}(x_0)$且$\tilde{v}(x_0)=u_\phi(x_0)$.
    \par 现在,我们证明下面的断言. 显然, 如果下面的断言成立, 就可以说明$u$是调和函数.
    \begin{claim}
        在$B_r(x_0)$中,有$\tilde{v}=u_\phi$.
    \end{claim}
    \par \textit{反证法.} 设存在$x_1 \in B_r(x_0)$ 使得$\tilde{v}(x_1) < u_\phi(x_0)$. 根据$u_\phi$的定义, 存在$w_k \in S_\phi$且$w_k(x_1) \to u_\phi(x_1)$. 不失一般性,我们可以假令$w_k \ge v_k$. 因为假如我们用$\max(w_k,v_k)$来代替$w_k$, 所得到的仍然是这样的序列.  记$W_k$为$w_k$的调和提升, 同样的,我们可以假设在$B_r(x_0)$中, $W_k$内闭一致收敛到某个函数$\tilde{w}(x)$且$\tilde{w}(x_1)=u_\phi(x_1)$. 现在,我们有
    \begin{align}
        &\tilde{v}(x) \le \tilde{w}(x) \le u_\phi(x) \s x \in B_r(x_0)\\
        &\tilde{v}(x_0)=\tilde{w}(x_0)=u_\phi(x_0) \\
        &\tilde{v}(x_1) < \tilde{w}(x_1)=u_\phi(x_1)
    \end{align}
    于是, $B_r(x_0)$中的调和函数$\tilde{v}-\tilde{w}$在内点$x_0$处取到局部最大值,则$\tilde{v}-\tilde{w}$是常数, 这显然与上面的不等式是矛盾的.
\end{proof}
现在, 虽然我们得到了调和函数$u_\phi$, 但是$u_\phi$的边界性质是不确定的. 边界上的连续性一般通过\textit{闸函数}来研究. 通常情况下, $u_\phi$是否能取到边界值$\phi$依赖于边界$\O$的几何性质.
\begin{definition}
    设$\O$是有界区域. 设$\zeta \in \P \O$. 称函数$w\in C(\overline{\O})$为Laplace算子$\Delta$相对于$\O$在点$\zeta$处的一个闸函数, 如果$w$满足 
    \begin{enumerate}
        \item $w$在$\O$中是上调和的.
        \item 在 $\overline{\O}-\zeta$中, $w >0$且 $w(\zeta)=0$.
    \end{enumerate}
\end{definition}
如果一个边界点上存在闸函数,就称该点为\textit{正则点}(相对于算子$\Delta$及区域$\O$).
\begin{lemma} \label{boundary}
    设$\zeta \in \P \O$是正则点, 则 
    \begin{equation}
        \lim_{\O \ni x \to \zeta} u_\phi(x)=\phi(\zeta).
    \end{equation}
\end{lemma}
\begin{proof}
    不失一般性,可将$\phi$连续延拓到$\O$上. 设$M=\sup \abs{\phi}$. 设$w(x)$是$\zeta$处的闸函数. 由于$\phi(x)$连续,则 $\forall \epsilon >0$, 存在$\delta>0$使得$\abs{x-\zeta} < \delta$时, $\abs{\phi(x)-\phi(\zeta)} < \delta$. 另外,在$B_\delta(\zeta)$外部, $w >0$, 因此必有严格正的最小值.  选择$K$使得 $Kw(x) > 2M$.  显然有
    \begin{equation}
        \abs{\phi(x)-\phi(\zeta)} \le \epsilon+Kw(x)
    \end{equation}
    则有
    \begin{equation}
        \phi(\zeta)-\epsilon-Kw(x) \le \phi(x)
    \end{equation}
    而由于$w$是上调和的,则$\phi(\zeta)-\epsilon-Kw(x) \in S_\phi$. 因此,有
    \begin{equation} \label{perron_sub}
        u_\phi(x) \ge \phi(\zeta)-\epsilon-Kw(x)
    \end{equation}
    另外, $\forall v \in S_\phi$及 $x \in \P \O$,有
    \begin{equation} 
        v(x)\le \phi(x) \le \phi(\zeta)+\epsilon+Kw(x)
    \end{equation}
    而 $v(x)$是下调和函数, $w$是上调和函数. 因此 $\forall x \in \O$,
    \begin{equation} \label{perron_sup_v}
        v(x)\le \phi(\zeta)+\epsilon+Kw(x)
    \end{equation}
    \begin{equation} \label{perron_sup}
        u(x)\le  \phi(\zeta)+\epsilon+Kw(x)
    \end{equation}
    不等式\eqref{perron_sub}, \eqref{perron_sup}即为
    \begin{equation}
        \abs{u_\phi(x)-\phi(\zeta)} \le \epsilon+Kw(x)
    \end{equation}
    令$x \to \zeta, \epsilon\to 0$即可.
\end{proof}
\begin{theorem}
    设$\O$是有界区域. 则Dirichlet问题\eqref{perron}对于任意的连续边界值$\phi \in C(\P \O)$是可解的,当且仅当$\O$的边界点都是正则点.
\end{theorem}
\begin{proof}
    如果每个点都是正则点, 那么由Perron方法得到调和函数$u_\phi$, 再由引理\eqref{boundary}得到$u_\phi|_{\partial \O}=\phi$即可. 
    \par 反之,设$\zeta \in \P \O$, $\phi(x)=\abs{x-\zeta}$. 由边界值$\phi(x)$解Dirichlet问题\eqref{perron}, 得到调和函数$u$. 显然$u$满足闸函数的定义.
\end{proof}
\begin{definition}
    设$\O$是有界区域. 若对于每一个点$\zeta \in \P \O$, 存在球$B_r(y) \subset \R^n-\O$使得$\overline{B}_r(y) \cap \overline{\O}=\zeta$, 则称区域$\O$满足外部球条件.
\end{definition}
\begin{theorem}
    设$\O$是有界区域且满足外部球条件.  则Dirichlet问题\eqref{perron}对于任意的连续边界值$\phi \in C(\P \O)$是可解的.
\end{theorem}
\begin{proof}
    设$\zeta \in \P \O$且存在球$B=B_r(y)$满足$\overline{B}\cap \overline{\O}=\zeta$. 定义函数
    \begin{equation}
        w(x)=r^{2-n}-\abs{x-y}^{2-n} \s n \ge 2
    \end{equation}
    可验证$w$是一个闸函数.
\end{proof}
其它可解性的例子. 
\begin{proposition}\label{regular_2d}
    设$\O \subset \R^2$是平面上的有界区域. 设对于任意一个点$\zeta \in \P \O$, 都存在一条弧$\gamma: [0,1] \to \R^2-\O$使得$\gamma((0,1)) \subset \R^2-\overline{\O}$且$\gamma(0) \in \P \O$, 则Dirichlet问题\eqref{perron}总是可解的.
\end{proposition}
\begin{proof}
    设$\zeta \in \P \O$. 根据假设, 存在一条弧$\gamma$将$\zeta$的足够小的邻域$B_r(\zeta)$沿$\zeta$点切开, 此时函数$-\rec{\log (z-\zeta)}$在$B_r(\zeta)-\gamma$中是良好定义的.取其实部 
    \begin{equation}
        w(z)=-\Re\rec{\log (z-\zeta)}=-\frac{\log r}{\log^2r+\theta^2}
    \end{equation}
    这里,$r=\abs{(z-\zeta)}, \theta=\arg(z-\zeta)$. 全纯函数的实部都是调和的, 所以$w(z)$是调和函数. 当$r<r_0$取比较小时,上面的函数满足闸函数的定义.
\end{proof}
\begin{example*}[非正则点]
    记$\R^3$中的点为$(x,y,z)$. 设$\rho^2=y^2+z^2$. 对于$(x,y,z) \notin [0,1]\times \{0\} \times \{0\}$, 定义函数
    \begin{equation}
        u(x,y,z)=\int^1_0 \frac{t}{\sqrt{(t-x)^2+\rho^2}}dt=v(x,\rho)-2x\ln \rho
    \end{equation}
    其中, 
    \begin{equation}
        \begin{split}
            v(x,\rho)= &\sqrt{(1-x)^2+\rho^2}-\sqrt{x^2+\rho^2} \\
            &+x\ln\abs{(1-x+\sqrt{(1-x)^2+\rho^2})(x+\sqrt{x^2+\rho^2})}
        \end{split}
    \end{equation}
    固定$\epsilon\in (0,1), \delta >1$. 记$\O_\epsilon=\{u(x,y,z) >\epsilon\}$, $\O_\delta=\{u(x,y,z) < 1+\delta\}$. 取$\O=\O_\epsilon\cap \O_\delta$. 易证, $\O_\epsilon$是包含$0$点的开集.
    \par 直接计算或者根据位势理论, $u$是调和函数. 对于区域$\O$, 显然它有两部分边界, $\P \O_\epsilon$与 $\P\O_\delta$.  在$\P \O_\epsilon$上, $u$是连续的且$u \eq \epsilon$. 而在$\P \O_\delta$上, $u$在除了$0$点的地方连续且等于$1+\delta$.
    $(u=\epsilon$部分的边界仅仅是为了取$\O$有界,因为$\abs{(x,y,z)} \to \infty$时, $u \to 0$.  重要的是$u=1+\delta$的这一部分边界).
    \par 我们将说明$0$点是一个非正则点. 先来观察函数$u$在$0$点附近的性质. 简单计算易知
    \begin{equation}
        \lim_{x>0, (x,\rho)\to 0} v(x,\rho)=1
    \end{equation}
    但是第二项$2x\ln \rho$的极限严重依赖于$(x,y,z)\to 0$的方式.  比如,$\forall k >0$, 如果沿着 $y^2+z^2=\rho^2=x^k$收敛到$0$点,则有$u(x,y,z) \to 1$.  而如果沿着$\rho=\exp(-\frac{c}{2x}), x>0, c>\delta$, 则$u(x,y,z) \to 1+c$.  因此, 当$(x,y,z)$足够小时,如果$y^2+z^2=\rho^2=x^k$, 则$(x,y,z) \in \O$.  如果$\rho=\exp(-\frac{c}{2x})$,则$(x,y,z) \in \O^c$.  这也意味着$(0,0,0) \in \P \O$.  
    \par 定义$\phi|_{\P \O_\epsilon}=\epsilon,  \phi|_{\P \O_\delta} =1+\delta$. 这时,由Perron方法得到的$u_\phi=u$. 参考[xxxxx]. 然而当$x <0$时, 
    \begin{equation}
        u(x,0,0)=\int^1_0\frac{t}{t-x}dt=1+x\log(1-x)-x\log\abs{x}
    \end{equation}
    则有$\lim_{x\to 0^-}u(x,0,0)=1 \ne 1+\delta$.
\end{example*}
\subsection{粘性解}
本节中, 我们总假设$\aij, f \in C(\overline{\O})$且$\aij$满足$(\lambda,\Lambda)$一致椭圆条件. 我们考虑下列方程在粘性解意义下的Dirichlet问题
\begin{equation} \label{vis_d}
    \left\{
        \begin{aligned}
            &\L u=\aij(x) \dij u(x)=f(x) \s x \in \O\\
            &u(x)=\phi(x) \s x \in \PO
        \end{aligned}
    \right.
\end{equation}
\par 在本节中,对于粘性上解/下解, 我们只要求它们是上半连续/下半连续的. 我们记$\lsc{\O}, \usc{\O}$分别为$\O$中下半连续/上半连续的函数的集合. 在后文中,我们用$u^*, u_*$分别表示
\begin{equation}
    u^*(x)=\limsup_{y\to x}u(y)
\end{equation}
\begin{equation}
    u_*(x)=\liminf_{y\to x}u(y)
\end{equation}
\begin{definition} \label{def_viscosity_sup}
    称$u \in \lsc{\O}$是方程\eqref{vis_d}的粘性上解, 如果$\forall x_0 \in \O$及 $\phi \in C^2(\O)$, 假如$u-\phi$在$x_0$处取到局部最小值, 则$\L \phi(x_0) \le f(x_0)$.  
\end{definition}
\begin{definition} \label{def_viscosity_sub}
    称$u \in \usc{\O}$是方程\eqref{vis_d}的粘性下解, 如果$\forall x_0 \in \O$及 $\phi \in C^2(\O)$, 假如$u-\phi$在$x_0$处取到局部最大值, 则$\L \phi(x_0) \le f(x_0)$.  
\end{definition}
\begin{remark}
    对于粘性下解,由于$u \in \usc{\O}$, 因此谈论``$u-\phi$在$x_0$处取到局部最大值是有意义的''.
\end{remark}
\begin{theorem}[粘性解的比较原理] \label{vis_comp}
    设$u \in \usc{\O}, v \in \lsc{\O}$分别是方程\eqref{vis_d}的粘性下解与粘性上解. 设在$\P \O$上, $u^* \le v_*$. 则在$\O$中, $u \le v$.
\end{theorem}
比较原理的证明比较长, 我们直接承认这个结果.
\begin{theorem} \label{vis_e}
    设$f \in \usc{\O}, g \in \lsc{\O}$分别是方程\eqref{vis_d}的粘性下解与粘性上解. 设$f \le g$且在$\P \O$, 有$f=g=\phi$. 则方程\eqref{vis_d}存在粘性解$u$且$f \le u \le g$.
\end{theorem}
\begin{lemma} \label{vis_l1}
    设$X \subset \usc{\O}$且$\forall u_i \in X$, $u_i$是方程\eqref{vis_d}的粘性下解. 设
    \begin{equation}
        u(x)=\sup_{u_i\in X}u_i(x)
    \end{equation}
    若 $\forall x \in \O$, $u^*(x) < +\infty$, 则$u^*$是方程\eqref{vis_d}的粘性下解.
\end{lemma}
\begin{proof}
    $\forall x \in \O$及$x_k \to x$, $u_k \in X$, 显然有
    \begin{equation}
        \limsup u_k(x_k) \le \limsup u^*(x_k) \le u^*(x)
    \end{equation}
    \par 固定$x_0$, 由定义可知, 存在$\{u_k\} \subset X$及$x_k \to x_0$使得$u_k(x_k) \to u^*(x_0)$.  设$\phi \in C^2(\O)$, 在$B_r(x_0)$中, $\phi(x) \ge u^*(x)$,且在$x_0$处等号成立.  
    \begin{claim}
        存在$\O \ni \tilde{x}_k \to x_0$及$\phi_k \in C^2(\O)$使得
        \begin{enumerate}
            \item $\phi_k \ge u_k$且在$\tilde{x}_k$处等号成立.
            \item $D\phi_k(\tilde{x}_k) \to D\phi(x_0)$, $D^2\phi_k(\tilde{x}_k) \to D^2\phi(x_0)$且$u_k(x_0) \to u^*(\tilde{x}_k)$.
        \end{enumerate}
    \end{claim}
    现在我们证明该断言. 显然地, $\phi(x)+\abs{x-x_0}^4 \ge u^*(x)$且在$x_0$处,等号成立. 另外, 在$B_r(x_0)-\{x_0\}$上, $\phi(x)+\abs{x-x_0}^4 >u^*$(严格不等号成立).   因此, 不失一般性, 我们用$\phi+\abs{x-x_0}^4$代替$\phi$. 设在$B_r(x_0)-\{x_0\}$上, 有
    \begin{equation} \label{strict_in}
        \phi(x) > u^*(x)
    \end{equation}
    \par 选择$\tilde{x}_k \in \overline{B_r(x_0)}$使得$u_k -\phi$在$\tilde{x}_k$处取到最大值. 取$\tilde{x}_k$的子列, 使得$\tilde{x}_k \to \tilde{x}_0 \in \overline{B}_r(x_0)$. 则有
    \begin{equation}
        \begin{split}
            (u^*-\phi)(x_0)=&\limsup (u_k-\phi)(x_k) \\
            \le & \limsup(u_k-\phi)(\tilde{x}_k) \\
            \le & u^*(\tilde{x}_0)-\phi(\tilde{x}_0)
        \end{split}
    \end{equation}
    因此, $\tilde{x}_0$也是$u^*-\phi$在$\overline{B}_r(x_0)$中的最大值. 而根据 \eqref{strict_in}, $u^*-\phi$只有唯一的最大值点$x_0$. 因此, $\tilde{x}_0=x_0$. 则当$k$足够大时,  有$x_k \in B_r(x_0)$(而不是在$\P B_r(x_0)$上).
    \par 记$\phi_k(x)=\phi(x)+\max_{B_r(x_0)}(u_k-\phi)$. 则$u_k(x) \le \phi_k(x)$且在 $\tilde{x}_k$处, 等号成立. 由于$\tilde{x}_k \to x_0$,  显然有
    \begin{equation}
        D\phi_k(\tilde{x}_k) \to D\phi(x_0), \s D^2\phi_k(x_k) \to D^2\phi(x_0)
    \end{equation}
    又因
    \begin{equation}
        u_k(x_k) -\phi(x_k) \le u_k(\tilde{x}_k) - \phi(\tilde{x}_k)
    \end{equation}
    两侧取极限, 则有
    \begin{equation}
        \begin{split}
            u^*(x_0)-\phi(x_0) \le& \limsup u_k(\tilde{x}_k)-\phi(x_0) \\
            \le & u^*(x_0)-\phi(x_0)
        \end{split}
    \end{equation}
    因此, $u_k(\tilde{x}_k) \to \phi(x_0) = u^*(x_0)$. 断言证毕.
    \par 由于$u_k$是粘性下解且$u_k-\phi_k$在$\tilde{x}_k$处取到局部最大值, 则有
    \begin{equation}
        \aij \dij \phi(\tilde{x}_k) \ge f(\tilde{x}_k)
    \end{equation}
    令$k \to \infty$即可.
\end{proof}
\begin{lemma} \label{vis_l2}
    设$u$是方程\eqref{vis_d}的粘性下解. 设在$x_0 \in \O$处, $u_*$不是方程\eqref{vis_d}的粘性上解. 则存在函数 $\tilde{u}(x)$及$x_0$的邻域$B_r(x_0)$使得
    \begin{enumerate}
        \item $\tilde{u} \in \usc{\O}$用$\tilde{u}$是方程\eqref{vis_d}粘性下解.
        \item $\tilde{u} \ge u$且在$\O-B_r(x_0)$上, $\tilde{u}\eq u$.
        \item $\{\tilde{u}>u\}$是非空的.
    \end{enumerate}
\end{lemma}
\begin{proof}
    不失一般性, 设$x_0=0$. 由于$u_*$在$x_0$处不是粘性上解, 则存在$C^2$函数$\phi$使得 $\phi \le u_*$, 在$x_0$处等号成立且$\L \phi(0)-f(0)>0$. 记
    \begin{equation}
        P_\epsilon=\phi(0)+D\phi(0)x+\rec{2}D^2\phi(0)x^2-\frac{\epsilon}{2}\abs{x}^2
    \end{equation}
    取$\epsilon$及$r$足够小, 使得在$B_r$中,
    \begin{equation}
        \L P_\epsilon(x) -f(x) \ge \L \phi(x) - f(x) - \frac{\epsilon}{2} \lambda >0
    \end{equation}
    于是$P_\epsilon$在$B_r(0)$中是方程\eqref{vis_d}的经典下解.  另外, 显然有
    \begin{equation}
        P_\epsilon(x) \le \phi(x) \le u_*(x)
    \end{equation}
    在$B_r-B_{\frac{r}{2}}$上, 有严格不等式$P_\epsilon(x) < u_*(x)$. 取$\delta >0$使得 在$B_r-B_{\frac{r}{2}}$, 
    \begin{equation} \label{perron_strict_cc}
        P_\epsilon(x)+\delta < u_*(x) \le u(x)
    \end{equation}
    定义
    \begin{equation}
        \tilde{u}(x)=\left\{
            \begin{aligned}
                & \max (u, P_\epsilon+\delta) \s x \in B_r(x) \\
                & u(x) \s x \in \O-B_r
            \end{aligned}
        \right.
    \end{equation}
    由于$u(x), P_\epsilon+\delta$都是上半连续的且是方程\eqref{vis_d}的粘性下解, 再由不等式\eqref{perron_strict_cc}知, $\tilde{u}(x)$是上半连续的且是方程\eqref{vis_d}的粘性下解. 另外, 
    \begin{equation}
        \begin{split}
            \limsup_{x\to 0} (\tilde{u}(x)-u(x)) = & \tilde{u}(0)-\liminf_{x\to 0} u(x) \\
            =& \delta + \phi(0)-u_*(0)\\
            =&\delta >0
        \end{split}
    \end{equation}
\end{proof}
\begin{proof}[定理\eqref{vis_e}的证明]
    记
    \begin{equation}
        \S_g=\{v \mid v \text{是粘性下解且}v \le g\}
    \end{equation}
    定义 $u(x)=\sup_{v \in S_g} v(x)$. 根据引理\eqref{vis_l1}, $f \le u\le u^* \le g$且$u^*$是粘性下解. 则$u^* \in S_g$. 于是有$u=u^*$.
    另外, 显然有
    \begin{equation}
        f_* \le u_* \le u \le u^* \le g^*
    \end{equation}
    则在$\partial \O$上, 有 $u \eq \phi$. 现在,只需要说明$u$也是粘性上解即可. 
    \par 若$u_*$在$x_0 \in \O$处不是粘性下解,则根据引理\eqref{vis_l2}, 存在粘性下解$\tilde{u}$使得在$B_r(x_0)$处, $\tilde{u}=u$且$\{\tilde{u} >u\}$非空. 而由比较定理, $\tilde{u} \le g$. 即有$\tilde{u} \in S_g$. 则$\tilde{u} \le u$, 矛盾.  由比较原理可知, $u_* \ge u^*$. 则$u=u_*=u^*$是粘性解.
\end{proof}
\fi%noperron
\section{连续性方法}
连续性方法依赖于全局Schauder估计. 
\begin{theorem}
    设$\O \subset \R^n$是有界$C^{2,\alpha}$区域. 设$u \in C^{2,\alpha}(\overline{\O})$是下列方程的解
    \begin{equation}
        \left\{
        \begin{aligned}
            &\L u=\aij \dij u + b^iD_iu+cu=f \s x \in \O \\
            &u|_{\P \O}=g \s x \in \P \O
        \end{aligned}
        \right.
    \end{equation}
    设$g \in C^{2,\alpha}(\overline{\O})$. $\aij, b^i, c, f \in C^\alpha(\overline{\O})$. 设$\aij$满足一致椭圆条件. 设$\norm{\aij}_{C^\alpha}+\norm{b^i}_{C^\alpha}+\norm{c}_{C^\alpha} \le \Lambda$ . 则存在常数$C=C(n,\lambda,\Lambda,\alpha,\O)$使得
    \begin{equation}
        \norm{u}_{C^{2,\alpha}(\overline{\O})} \le C(\norm{u}_{L^\infty}+ \norm{f}_{C^\alpha(\overline{\O})}+\norm{g}_{C^{2,\alpha}(\overline{\O})})
    \end{equation}
\end{theorem}
\renewcommand{\B}{\mathcal{B}}
\renewcommand{\V}{\mathcal{V}}
\begin{theorem}\label{function_a_continuous}
    设$\B$是Banach空间, $\V$是赋范向量空间. 设$L_0, L_1: \B\to \V$是有界算子. $\forall t \in [0,1]$, 令
    \begin{equation}
        L_t=(1-t)L_0+tL_1
    \end{equation}
    设存在常数$C$使得$\forall t \in [0,1], x \in \B$, 成立
    \begin{equation}
        \norm{x} \le C\norm{L_tx}
    \end{equation}
    则$L_1$是满射当且仅当$L_0$是满射.
\end{theorem}
\begin{proof}
    设存在$s \in [0,1]$使得$L_s$是满射. 显然, $L_s$可逆, $L_s^{-1}$连续且$\norm{L_s^{-1}} \le C$.  $L_t$的满射性等价于
    \begin{equation}
        \forall y \in \V, \exists x \in \B \text{使得} L_tx=y.
    \end{equation}
    $\iff$
    \begin{equation}
        L_sx+(L_t-L_s)x=y
    \end{equation}
    $\iff$
    \begin{equation}
        x=L_s^{-1}y-L_s^{-1}(L_t-L_s)x
    \end{equation}
    $\iff$
    $x$是映射$Tx=L_s^{-1}y-L_s^{-1}(L_t-L_s)x$的不动点. 而对于映射$T$,
    \begin{equation}
        \begin{split}
            \norm{Tx_1-Tx_2} \le& \norm{L_s^{-1}}\norm{L_t-L_s}\norm{x_2-x_1}\\
            \le & (t-s)C\norm{L_1-L_0}
        \end{split}
    \end{equation}
    因此, 存在$\delta >0$ 使得 当 $s-t\le \delta$时,$T$是压缩映射. 此时$T$有唯一的不动点. 这也就意味着, 当$\abs{t-s} < \delta$时, $L_t$是满射.  通过将$[0,1]$分割 成若干长度为$\delta$的子区间可知, $\forall t \in [0,1]$, $L_t$是满射.
\end{proof}
\newcommand{\ctwoa}{{C^{2,\alpha}}}
\newcommand{\OC}{\overline{\Omega}}
\begin{theorem}
    设$\O \subset \R^n$是$\ctwoa$区域. 设$\L u = \aij \dij u + b^iD_iu +cu$是严格椭圆型的且具有$C^\alpha(\OC)$系数. 设$c \le 0$. 如果Poission方程
    \begin{equation}
        \left\{
            \begin{aligned}
                &\Delta u=f  \s x \in \O\\
                &u|_{\P \O}= g \s x \in \P \O
            \end{aligned}
        \right.
    \end{equation}
    对于$\forall f \in C^\alpha(\OC), g \in \ctwoa(\OC)$可解,那么方程
    \begin{equation}
        \left\{
            \begin{aligned}
                &\L u=f \s x \in \O\\
                &u|_{\P \O}=g \s x \in \P \O
            \end{aligned}
        \right.
    \end{equation}
    对于$\forall f \in C^\alpha(\OC), g \in \ctwoa(\OC)$有唯一解.
\end{theorem}
\begin{proof}
    通过考虑函数$u_g=u-g$, 可设$u|_{\P \O}=0$. 设$L_0=\Delta$, $L_1=L$. 记
    \begin{equation}
        L_t=(1-t)L_0+tL_1.
    \end{equation}
    显然, $L_t$的系数满足一致椭条件. 记
    \begin{equation}
        \V= C^\alpha(\OC), \s \B=\{u \mid u \in \ctwoa(\OC), u|_{\P \O}=0\}
    \end{equation}
    设$u \in \B$. $L_tu \in \V$. 由最大值原理可知
    \begin{equation}
        \norm{u}_{L^\infty} \le C(n,\lambda,\Lambda,\O)\norm{f}_{L^\infty}
    \end{equation}
    再由Schauder估计可知,
    \begin{equation}
        \norm{u}_{\ctwoa(\OC)} \le C\norm{f}_{C^\alpha(\OC)} = C\norm{L_tu}_{C^\alpha(\OC)}
    \end{equation}
    由定理\eqref{function_a_continuous}可知, $L_1$是满射. 
\end{proof}
\begin{remark}
    Poisson方程的可解性可由位势理论得出.参考[xxx].
\end{remark}
\section{不动点方法}
%\appendix
%%\addcontentsline{toc}{chapter}{附录}
\appendix
\appendix
\chapter{附录}
\section{复分析基础}
\subsection{等温坐标的存在性}
\begin{theorem}
    设$(M^2, g)$是二维黎曼流形, 则存在$M^2$上的一组坐标卡$\{z=x+iy\}$, 使得在这组坐标下, $g=\lambda^2\abs{dz}^2$, 并且这组坐标下的坐标变换是全纯的.
\end{theorem}
\begin{proof}
    固定点$p \in M$. 取$p$点的足够小的邻域 $\Omega$, 在$\Omega$中, 取函数$u$ 使得
    \begin{equation}
        \Delta_g u= \Delta u + \Gamma^j_{ii}u_j=0.
    \end{equation}
    根据椭圆方程的理论, 这个方程总是局部可解的. 令$\omega= \sqrt{\det g}dx$为$M$上的体积形式. 定义$1-$形式
    \begin{equation}
        \alpha= \Tr(\nabla u \otimes \omega)= i^*_{\nabla_u } \omega
    \end{equation}
    则 $d\alpha= \div(\nabla u)\omega =0$. 即, $\alpha$是闭的$1-$形式. 由Poincare引理可知, 存在函数$v$使得$\alpha= dv$.  现在,我们证明函数$u,v$具有如下关系:
    \begin{enumerate}
        \item $\inner{\nabla u}{\nabla v} =0$.
        \item $\abs{\nabla u}= \abs{\nabla v}$.
    \end{enumerate}
    对于性质(1), 我们有
    \begin{equation}
        \inner{\nabla u}{\nabla v} = dv(\nabla u) = \alpha(\nabla u)= (i^*_{\nabla u} w)\nabla u =0
    \end{equation}
    对于性质(2), 我们有
    \begin{equation}
        \inner{\nabla v}{\nabla v} = \inner{dv}{dv} = \inner{i^*_{\nabla u}\omega }{ i^*_{\nabla u}\omega}= \inner{\nabla u}{\nabla u}.
    \end{equation}
    上面的最后一个等式在测地坐标下是容易验证的. 
    \par 现在, 设$F(x,y)=(u,v)$. 在点$p$处, 由性质(1)可知, $\det dF(p) \ne 0$. 由反函数定理可知, $(u,v)$可以作为一组局部坐标. 在这组坐标下, 度量$g$的局部表示为
    \begin{equation}
        g=g_{uu}du^2+2g_{uv}dudv+g_{vv}dv^2
    \end{equation}
    由于
    \begin{align}
        &g_{uu}= \abs{\nabla u} = \abs{\nabla v}= g_{vv} \\
        &g_{uv}=\inner{\nabla u}{\nabla v}=0. 
    \end{align}
    只需要取$z=u+iv$, $\lambda= \abs{\nabla u}$即可.
\end{proof}
\subsection{调和共轭}
\begin{proposition}
    设$\Omega$是单连通区域. $u: \Omega \to \R$是调和函数. 则存在全纯函数$F$使得$u=\Re F$.
\end{proposition}
\begin{proof}
    定义
    \begin{equation}
        \tilde{v}(x,y)= \int^y_0 u_x(x,t)dt. 
    \end{equation}
    那么
    \begin{equation}
        \begin{split}
            \tilde{v}_x  =\int^y_0 u_{xx}(x,t)dt &=- \int^y_0 u_{tt}(x,t)dt \\
            &= -u_y(x,y)+u_y(x,0)
        \end{split}
    \end{equation}
    取$v(x,y)=\tilde{v}(x,y)- \int^x_0 u_y(t,0)dt$, 则$F(x,y)=u+iv$满足Cauchy-Riemann方程.
\end{proof}

%%\addcontentsline{toc}{chapter}{附录}
\appendix
\appendix
\chapter{附录}
\section{复分析基础}
\subsection{等温坐标的存在性}
\begin{theorem}
    设$(M^2, g)$是二维黎曼流形, 则存在$M^2$上的一组坐标卡$\{z=x+iy\}$, 使得在这组坐标下, $g=\lambda^2\abs{dz}^2$, 并且这组坐标下的坐标变换是全纯的.
\end{theorem}
\begin{proof}
    固定点$p \in M$. 取$p$点的足够小的邻域 $\Omega$, 在$\Omega$中, 取函数$u$ 使得
    \begin{equation}
        \Delta_g u= \Delta u + \Gamma^j_{ii}u_j=0.
    \end{equation}
    根据椭圆方程的理论, 这个方程总是局部可解的. 令$\omega= \sqrt{\det g}dx$为$M$上的体积形式. 定义$1-$形式
    \begin{equation}
        \alpha= \Tr(\nabla u \otimes \omega)= i^*_{\nabla_u } \omega
    \end{equation}
    则 $d\alpha= \div(\nabla u)\omega =0$. 即, $\alpha$是闭的$1-$形式. 由Poincare引理可知, 存在函数$v$使得$\alpha= dv$.  现在,我们证明函数$u,v$具有如下关系:
    \begin{enumerate}
        \item $\inner{\nabla u}{\nabla v} =0$.
        \item $\abs{\nabla u}= \abs{\nabla v}$.
    \end{enumerate}
    对于性质(1), 我们有
    \begin{equation}
        \inner{\nabla u}{\nabla v} = dv(\nabla u) = \alpha(\nabla u)= (i^*_{\nabla u} w)\nabla u =0
    \end{equation}
    对于性质(2), 我们有
    \begin{equation}
        \inner{\nabla v}{\nabla v} = \inner{dv}{dv} = \inner{i^*_{\nabla u}\omega }{ i^*_{\nabla u}\omega}= \inner{\nabla u}{\nabla u}.
    \end{equation}
    上面的最后一个等式在测地坐标下是容易验证的. 
    \par 现在, 设$F(x,y)=(u,v)$. 在点$p$处, 由性质(1)可知, $\det dF(p) \ne 0$. 由反函数定理可知, $(u,v)$可以作为一组局部坐标. 在这组坐标下, 度量$g$的局部表示为
    \begin{equation}
        g=g_{uu}du^2+2g_{uv}dudv+g_{vv}dv^2
    \end{equation}
    由于
    \begin{align}
        &g_{uu}= \abs{\nabla u} = \abs{\nabla v}= g_{vv} \\
        &g_{uv}=\inner{\nabla u}{\nabla v}=0. 
    \end{align}
    只需要取$z=u+iv$, $\lambda= \abs{\nabla u}$即可.
\end{proof}
\subsection{调和共轭}
\begin{proposition}
    设$\Omega$是单连通区域. $u: \Omega \to \R$是调和函数. 则存在全纯函数$F$使得$u=\Re F$.
\end{proposition}
\begin{proof}
    定义
    \begin{equation}
        \tilde{v}(x,y)= \int^y_0 u_x(x,t)dt. 
    \end{equation}
    那么
    \begin{equation}
        \begin{split}
            \tilde{v}_x  =\int^y_0 u_{xx}(x,t)dt &=- \int^y_0 u_{tt}(x,t)dt \\
            &= -u_y(x,y)+u_y(x,0)
        \end{split}
    \end{equation}
    取$v(x,y)=\tilde{v}(x,y)- \int^x_0 u_y(t,0)dt$, 则$F(x,y)=u+iv$满足Cauchy-Riemann方程.
\end{proof}

%\input{appendix3}
%%\appendix
\chapter{习题}
\begin{enumerate}
    \item 设$\Omega\subset \R^n$是有界区域. 设$u \in W^{1,1}(\Omega)$.  设$\Area(u)=\int_\Omega \sqrt{1+\abs{Du}^2}$. 证明: $\Area(u)$在$W^{1,1}(\Omega)$上是严格凸的.
    \item 证明: $\R^n$中不存在紧致无边的极小曲面.
    \item 设 $\Sigma \subset \R^3$是极小曲面. 设存在平面$P$使得$\Sigma \perp P$. 设$\Sigma$关于平面$P$的反射后的像为$\Sigma'$. 证明: $\Sigma \cup \Sigma'$是极小曲面.
%    \item 设$f: \Omega \to \mathbb{C}$是光滑函数. 证明:
%    \begin{enumerate}
%        \item $f$全纯当且仅当 $\barpartial f=0$.
%        \item $f$调和当且仅当$\partial \barpartial f=0$.
%    \end{enumerate}
    \item 设$\Sigma$为Catenoid. $\Gamma_a=\Sigma \cap \{z=\pm a\}$. 试比较以下三个以$\Gamma_a$为边界的曲面的面积大小.
    \begin{enumerate}
        \item $\Gamma_a$所围的两个圆盘之并.  
        \begin{equation*}
            \Sigma_1=\{(x,y,z)\mid \sqrt{x^2+y^2} \le \cosh a, z=\pm a\}.
        \end{equation*}
        \item $\Gamma_a$的两个分支之间的柱面. 
        \begin{equation*}
            \Sigma_2=\{(x,y,z)\mid \sqrt{x^2+y^2} = \cosh a, \abs{z} \le a\}.
        \end{equation*}
        \item $\Gamma_a$之间Catenoid的部分. 
        \begin{equation*}
            \Sigma_3=\{(x,y,z)\mid  \sqrt{x^2+y^2}=\cosh z, \abs{z} \le a \}.
        \end{equation*}
    \end{enumerate}
    \item 设$\gamma\subset \R^3$是 $xz$平面上的曲线. $\gamma= \{(x,0,z)\mid x=f(z) >0 \}$. 由$\gamma$ 绕$z$轴旋转所得到的曲面记为$\Sigma$. 若$\Sigma$是极小曲面, 试求$f$的表达式.
    \item 设$\Sigma^{n-1} \subset \M^n$是极小曲面. $F(x,t): \Sigma \times (-\epsilon, \epsilon)\to \M$是固定边界的变分. 设$H_t$为 $\Sigma_t$的平均曲率. 计算 $\ddt H_t \mid _{t=0}$.
    \item 设$u$是Catenoid上的有界调和函数, 证明$u$是常数.
    \item 设$\M^3$可定向且$\Sigma^2 \subset \M^3$是定向的稳定极小曲面. 设$\tilde{\Sigma}$是$\Sigma$的覆盖空间. 则覆盖映射$f:\tilde{\Sigma}\to \Sigma \hookrightarrow \M$是稳定极小曲面.
    \item 设$\Sigma^{n-1} \subset \M^n$是极小曲面. 证明: $\forall p \in \Sigma$, 存在$p$点的足够小的邻域$\Omega$使得$\Omega \cap \Sigma$是稳定的.
    %\item 补充定理\eqref{compactness}的证明细节.
    \item 补充定理\eqref{compactness}的证明细节, 并说明``第二基本型一致有界"可以替换为``局部一致有界".
    \item 通过等式\eqref{total_curvature_degree}计算Catenoid的全曲率, 并验证定理\eqref{curvature_estimate_4pi}在$C=8\pi$时是否正确.
    \item 设$\Sigma \subset \R^3$是极小曲面, $\II$是其第二基本型. 将$\Sigma$赋予度量$\abs{\II}\inner{\cdot}{\cdot}$, 计算该度量的曲率.
\end{enumerate}
%\nocite{*}
%\clearpage
%%\phantomsection
%%\addcontentsline{toc}{section}{参考文献}
%\tolerance=500
%\printbibliography[heading=bibintoc, title=参考文献]
\end{document}